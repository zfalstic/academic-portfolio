\documentclass[11pt]{article}

\usepackage[utf8]{inputenc}
\usepackage[T1]{fontenc}
\usepackage{lmodern}
\usepackage{microtype}

\usepackage[sexy]{evan}

\usepackage{fancyhdr}
\pagestyle{fancy}
\fancyhf{}

\usepackage{amsmath, amssymb, amsthm, mathtools}
\usepackage{siunitx}
\usepackage{graphicx}
\usepackage{float}

\usepackage{pgfplots}
\pgfplotsset{compat=1.18}
\usepgfplotslibrary{statistics}

\newcommand{\coursename}{\textbf{Introduction to Computing}}
\newcommand{\coursecode}{ECE 306H}
\newcommand{\term}{Fall 2025}
\newcommand{\instructor}{Dr.\ Yerraballi}
\newcommand{\notetaker}{Dawson Zhang}
\newcommand{\lecturetitle}{Homework 1}

\fancyhead[L]{\coursecode}
\fancyhead[C]{\lecturetitle}
\fancyhead[R]{\term}
\fancyfoot[R]{\thepage}

\title{\coursename~(\coursecode) -- \lecturetitle}
\author{\notetaker~~|~~Instructor: \instructor}
\date{\term}

\begin{document}
  
\maketitle
\pagebreak

\section{Problems}

\begin{problem}
Say we had a ``black box,'' which takes two numbers as input and outputs their sum.
See Figure 1.7a in the Textbook. Say we had another box capable of multiplying two
numbers together. See figure 1.7b. We can connect these boxes together to calculate $p * (m
+ n)$. See Figure 1.7c. Assume we have an unlimited number of these boxes. Show how to
connect them together to calculate:

\begin{enumerate}[\alph*)]
\item
$ax - by$
\item
$ijk$
\item
$a^2 - b^2$
\item
$a^3 + 3a^2b + 3ab^2 + b^3$ Try to do it with only one add box and two
multiply boxes
\item
$b^4$
\end{enumerate}
\end{problem}

\begin{soln}
For the purposes of this problem, I will represent an addition black box
with $A(x, y)$ and multiplication black box as $M(x, y)$ where $x$ and
$y$ are the two inputs.

\begin{enumerate}[\alph*)]
\item
\begin{equation*}
ax - by = A(M(a,x), M(-1,M(b,y))) 
\end{equation*}
\begin{figure}[H]
\centering
\includegraphics[width=0.4\textwidth]{1a.png}
\end{figure}
\item
\begin{equation*}
ijk = M(i,M(j,k))  
\end{equation*}
\begin{figure}[H]
\centering
\includegraphics[width=0.35\textwidth]{1b.png}
\end{figure}
\item
\begin{equation*}
a^2 - b^2 = A(M(a,a), M(-1,M(b,b)))  
\end{equation*}
\begin{figure}[H]
\centering
\includegraphics[width=0.45\textwidth]{1c.png}
\end{figure}
\item
\begin{align*}
  a^3 + 3a^2b + 3ab^2 + b^3 &= (a + b)^3 \\
                            &= M(A_1(a, b), M(A_1, A_1))
\end{align*}
\begin{remark*}
In this scenario, $A(a,b)$ is represented as $A_1$ because the black box's
output is used multiple times which on paper is just represented as a connection
and not a new black box.
\end{remark*}
\begin{figure}[H]
\centering
\includegraphics[width=\textwidth]{1d.png}
\end{figure}
\item
\begin{equation*}
b^4 = M(M_1(b,b),M_1)  
\end{equation*}
\begin{figure}[H]
\centering
\includegraphics[width=0.4\textwidth]{1e.png}
\end{figure}
\end{enumerate}
\end{soln}

\pagebreak
\begin{problem}
Perform the following conversions, assuming unsigned numbers.
\begin{enumerate}[\alph*)]
  \item $(1432)_{16} = (\quad )_2$
  \item $(110100100101)_2 = (\quad )_{16}$
  \item $(1010001011010011)_2 = (\quad )_8$
  \item $(1100)_8 = (\quad )_2$
  \item $(1432)_8 = (\quad )_{16}$
  \item $(1432)_{16} = (\quad )_8$
  \item $(0100010101)_2 = (\quad )_{12}$
  \item $(1234)_5 = (\quad )_7$
\end{enumerate}
\end{problem}

\begin{soln}
\end{soln}

\begin{enumerate}[\alph*)]

\item $(1010000110010)_{2}$
\item $(D25)_{16}$
\item $(121323)_{8}$
\item $(1001000000)_{2}$
\item
First convert it to base-2 before converting to base-16 \\
base-2: $(1100011010)_{2}$ \\
base-16: $(31A)_{16}$
\item
Same thing again \\
base-2: $(1010000110010)_{2}$ \\
base-8: $(12062)_{8}$ 
\item
Convert to base-10 then to base-12 \\
base-10: $256 + 16 + 4 + 1 = (277)_{10}$ \\
base-12: $(1B1)_{12}$
\item
Convert to base-10 then to base-7 \\
base-10: $4 + 15 + 50 + 125 = (194)_{10}$ \\
base-7: $(365)_{7}$

\end{enumerate}

\pagebreak
\begin{problem}
Show the binary representation of the following signed decimal numbers in 8-bit 2’s
complement. If they cannot be represented in 8 bits, write Overflow.
\begin{enumerate}[\alph*)]
\item $100$
\item $128$
\item $-101$
\item $-128$
\item $-1$
\item $42$
\end{enumerate}
\end{problem}

\begin{soln}
\end{soln}

\begin{enumerate}[\alph*)]
\item
$(01100100)_{\text{2's comp}}$
\item
Overflow
\item
$(10011011)_{\text{2's comp}}$
\item
$(10000000)_{\text{2's comp}}$
\item
$(11111111)_{\text{2's comp}}$
\item
$(00101010)_{\text{2's comp}}$
\end{enumerate}

\pagebreak
\begin{problem}
Given a number in base 16, find the corresponding decimal value by interpreting it
as an unsigned number, signed magnitude, one’s complement, and two’s complement.
\begin{enumerate}[\alph*)]
  \item $(22)_{16}$
  \item $(4B)_{16}$
  \item $(7F)_{16}$
\end{enumerate}
\end{problem}

\begin{soln}
\end{soln}

\begin{enumerate}[\alph*)]
  \item 
  $2 + 2(16) = (34)_{10}$ \\
  $(100010)_{2}$ \\
  $(0100010)_{\text{signmag}}$ \\
  $(00100010)_{\text{1's comp}}$ \\
  $(00100010)_{\text{2's comp}}$
  \item
  $11 + 4(16) = (75) _{10}$ \\
  $(1001011) _{2}$ \\
  $(01001011) _{\text{signmag}}$ \\
  $(01001011) _{\text{1's comp}}$ \\
  $(01001011) _{\text{2's comp}}$
  \item
  $15 + 7(16) = (127) _{10}$ \\
  $(1111111) _{2}$ \\
  $(01111111) _{\text{signmag}}$ \\
  $(01111111) _{\text{1's comp}}$ \\
  $(01111111) _{\text{2's comp}}$
\end{enumerate}

\pagebreak
\begin{problem}
Without changing their values, convert the following signed numbers given in 2's complement representation into 8-bit signed numbers in 2's complement representation, if possible. Give their decimal values and 8-bit 2’s complement representation. If not possible, write the decimal value.
\begin{enumerate}[\alph*)]
  \item $1$
  \item $101$
  \item $1011101010$
  \item $111110000000$
  \item $00101$
  \item $011111111$
\end{enumerate}
\end{problem}

\begin{soln}
\end{soln}

\begin{enumerate}[\alph*)]
  \item $(11111111)_{\text{2's comp}} = (-1)_{\text{decimal}}$
  \item $(11111101)_{\text{2's comp}} = (-3)_{\text{decimal}}$
  \item $(-278)_{\text{decimal}}$
  \item $(10000000)_{\text{2's comp}} = (-128)_{\text{decimal}}$
  \item $(00000101)_{\text{2's comp}} = (5)_{\text{decimal}}$
  \item $(255)_{\text{decimal}}$
\end{enumerate}

\pagebreak
\begin{problem}
Add the following signed numbers given in 2’s complement representation. Express
your final answer in 8 bits and in decimal. If the result of the addition cannot fit inside 8 bits,
write Overflow.
\begin{enumerate}[\alph*)]
  \item $1 + 101$
  \item $01101111 + 01$
  \item $0010 + 1100$
  \item $0101 + 001001$
\end{enumerate}
\end{problem}

\begin{soln}
\end{soln}

\begin{enumerate}[\alph*)]
\item
\[
\left\{
\begin{aligned}
11111111 \\
11111101
\end{aligned}
\right.
\quad \Rightarrow \quad
(11111100)_{\text{2's comp}} = (-4)_{\text{decimal}}
\]
\item
\[
\left\{
\begin{aligned}
01101111 \\
00000001
\end{aligned}
\right.
\quad \Rightarrow \quad
(01110000)_{\text{2's comp}} = (112)_{\text{decimal}}
\]
\item
\[
\left\{
\begin{aligned}
00000010 \\
11111100
\end{aligned}
\right.
\quad \Rightarrow \quad
(11111110)_{\text{2's comp}} = (-2)_{\text{decimal}}
\]
\item
\[
\left\{
\begin{aligned}
00000101 \\
00001001
\end{aligned}
\right.
\quad \Rightarrow \quad
(00001110)_{\text{2's comp}} = (14)_{\text{decimal}}
\]
\end{enumerate}

\pagebreak
\begin{problem}
You wish to express -128 as a 2's complement number.
\begin{enumerate}[\alph*)]
  \item How many bits do you need? (the minimum number)
  \item With this number of bits, what is the largest positive number you can represent? (Please give answers both in decimal and binary.)
  \item With this number of bits, what is the largest unsigned number you can represent? (Please give answers both in decimal and binary.)
\end{enumerate}
\end{problem}

\begin{soln}
\end{soln}

\begin{enumerate}[\alph*)]
  \item $8$ bits
  \item $127$ \\
    $(1111111)_{2}$
  \item $255$ \\
    $(11111111)_{2}$
\end{enumerate}

\pagebreak
\begin{problem}
We have represented numbers in base-2 (binary) and in base-16 (hex). We are now
ready for unsigned base-4, which we will call quad numbers. A quad digit can be 0, 1, 2, or 3.
\begin{enumerate}[\alph*)]
\item What is the maximum unsigned decimal value that one can represent with 5 quad digits?
\item What is the maximum unsigned decimal value that one can represent with n quad digits? (Hint: your answer should be a function of n.)
\item Add the two unsigned quad numbers: 123 and 321.
\item What is the quad representation of the decimal number 333 using 5 quad digits?
\end{enumerate}
\end{problem}

\begin{soln}
\end{soln}

\begin{enumerate}[\alph*)]
  \item $3(4)^{0} + 3(4)^{1} + 3(4)^{2} + 3(4)^{3} + 3(4)^{4} = \boxed{1023}$
  \item $4^n - 1$
  \item
  \[
  \left\{
  \begin{aligned}
  123 \\
  321
  \end{aligned}
  \right.
  \quad \Rightarrow \quad
  \boxed{(1110)_{2}}
  \]
  \item $(11031)_{\text{4}}$
\end{enumerate}

\end{document}
