\documentclass[11pt]{article}

\usepackage[utf8]{inputenc}
\usepackage[T1]{fontenc}
\usepackage{lmodern}
\usepackage{microtype}

\usepackage[sexy]{evan}

\usepackage{fancyhdr}
\pagestyle{fancy}
\fancyhf{}

\usepackage{amsmath, amssymb, amsthm, mathtools}
\usepackage{graphicx}
\usepackage{tikz}

\newcommand{\coursename}{\textbf{Introduction to Electrical Engineering}}
\newcommand{\coursecode}{ECE 302H}
\newcommand{\term}{Fall 2025}
\newcommand{\instructor}{Dr.\ Hanson}
\newcommand{\notetaker}{Dawson Zhang}
\newcommand{\lecturetitle}{Fundamentals of Electricity}

\fancyhead[L]{\coursecode}
\fancyhead[C]{\lecturetitle}
\fancyhead[R]{\term}
\fancyfoot[C]{\leftmark}
\fancyfoot[R]{\thepage}

\title{\coursename~(\coursecode) -- \lecturetitle}
\author{\notetaker~~|~~Instructor: \instructor}
\date{\term}

\begin{document}
  
\maketitle
\pagebreak

\section{Charge}
\begin{itemize}
  \item Proton $= 1e$
  \item Electron $= -1e$
  \item \textbf{Coulomb} $= 6.24*10^{18}e$
\end{itemize}

\section{Electricity}
\begin{itemize}
  \item Imbalance of Charge: \textbf{Static Electricity}
  \item Imbalance in Flow of Charge: \textbf{Current}
  \begin{remark}
  Dr. Hanson drew two examples of charge flowing. In the first example,
  there are 2 positive charges moving right and 2 negative charges
  moving right. In the second example, there are 2 positive charges
  moving right, \textbf{but} 2 negative charges moving left.

  The first example experiences no current while the second example does.
  \end{remark}
  \begin{remark}
  I want to clarify here that conventional current follows the direction
  that the positive charges move. Dr. Hanson's second example would
  therefore imply that the conventional current moves towards the right.
  \end{remark}
\end{itemize}

\section{Current}
\begin{definition}
\textbf{Current} is definied as a coulomb per second also known as an
amp or ampere and is represented with $i$.
\end{definition}
insert circuit diagram

\section{Potential (Energy)}
\begin{definition}
\textbf{Electric potential} is defined as a joule per coulomb also known as a volt.
Because of this, electric potential is often refered to as voltage.
\end{definition}

\section{Power}
\begin{remark}
Recall that power in mechanics comes from a joule per second. If we 
multiply current and electric potential (voltage), coulombs cancel out
and we end up with joules per second or watts
\begin{equation*}
\frac{J}{C} \times \frac{C}{s} = \frac{J}{s} = W 
\end{equation*}
\end{remark}

\section{Circuits}
\begin{remark}
Dr. Hanson breaks down ``solving'' a circuit into two steps:
\begin{enumerate}
  \item Laws of Physics
  \item Solving Systems of Equations
\end{enumerate}
\end{remark}
\subsection{Relevant Laws of Physics}
\begin{itemize}
  \item Conservation of Charge
  \begin{equation*}
    \sum_{\text{into node region}} i=0  
  \end{equation*}
  also known as Kirchoff's Current Law \textbf{(KCL)}
  \item Conservation of Energy
  \begin{equation*}
    \sum_{\text{loop}} V=0
  \end{equation*}
  also known as Kirchoff's Voltage Law \textbf{(KVL)}
  \item Component $i$-$V$ Relationship
  \begin{itemize}
    \item Voltage Source
    \begin{equation*}
    V(i) = V_{\text{value}}  
    \end{equation*}
    \begin{remark}
    Note that just because current is not in the equation, it is not zero
    \end{remark}
    \item Resistor
    \begin{equation*}
    V(i) = Ri  
    \end{equation*}
    where $R$ is resistance measured in $\frac{V}{A}=\Omega$.
  \end{itemize}
\end{itemize}

insert example of circuit with 3V source and 1.5ohm resistor

\subsection{Important Circuit Terminology}

\begin{itemize}
\item Node - Wire
\item Loop - Think of it as anywhere you can draw a loop
\item Series - Two components experience same current by KCL
\item Parallel - Two components experience same voltage by KVL
\end{itemize}

\subsection{Passive Sign Convention}

\begin{definition}
an electrical engineering standard where the current is assumed to flow from the positive (+) terminal of a component to the negative (-) terminal, and the power absorbed by the component is considered positive (a sink of power)
\end{definition}

\begin{remark}
Basically just label current going from the positive to negative terminal.
\end{remark}

\subsection{Equivalent Circuits}

\begin{definition}
If two circuits have the same $i$-$v$ relationship, they are \textbf{equivalent}
\end{definition}

\end{document}
