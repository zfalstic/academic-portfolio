\documentclass[11pt]{article}

\usepackage[utf8]{inputenc}
\usepackage[T1]{fontenc}
\usepackage{lmodern}
\usepackage{microtype}

\usepackage[sexy]{evan}

\usepackage{fancyhdr}
\pagestyle{fancy}
\fancyhf{}

\usepackage{amsmath, amssymb, amsthm, mathtools}
\usepackage[siunitx]{circuitikz}
\usepackage{graphicx}
\usepackage{float}
\usepackage{chemformula}
\usepackage{booktabs}

\usepackage{pgfplots}
\pgfplotsset{compat=1.18}
\usepgfplotslibrary{statistics}

\usepackage{listings}
\usepackage{xcolor}

\lstset{
  basicstyle=\ttfamily\small,
  keywordstyle=\color{blue},
  commentstyle=\color{gray},
  stringstyle=\color{teal},
  frame=single,
  breaklines=true
}

\newcommand{\coursename}{\textbf{Introduction to Electrical Engineering}}
\newcommand{\coursecode}{ECE 302H}
\newcommand{\term}{Fall 2025}
\newcommand{\instructor}{Dr.\ Hanson}
\newcommand{\notetaker}{Dawson Zhang}
\newcommand{\lecturetitle}{Homework 10}

\fancyhead[L]{\coursecode}
\fancyhead[C]{\lecturetitle}
\fancyhead[R]{\term}
\fancyfoot[R]{\thepage}

\title{\coursename~(\coursecode) -- \lecturetitle}
\author{\notetaker~~|~~Instructor: \instructor}
\date{\term}

\begin{document}
  
\maketitle
\pagebreak

\setcounter{section}{10}

\begin{problem}

\textbf{Poles and Zeros}

Transfer functions of linear systems always look like,

\begin{align*}
T = \frac{\text{polynomial of } \omega}{\text{other polynomial of } \omega}
\end{align*}

If those polynomials are factored, then the transfer function looks like,

\begin{align*}
T = A \frac{\left( 1 + \frac{j \omega}{\omega _{z1}} \right)\left( 1 + \frac{j \omega}{\omega _{z2}} \right)\left( 1 + \frac{j \omega}{\omega _{z3}} \right) \cdots}{\left( 1 + \frac{j \omega}{\omega _{p1}} \right)\left( 1 + \frac{j \omega}{\omega _{p2}} \right)\left( 1 + \frac{j \omega}{\omega _{p3}} \right) \cdots}
\end{align*}

where $A$ is a constant scaling factor. The known frequency elements $\omega_z$ (they come from factoring the
polynomial in the numerator) are called zeros, while the elements labeled $\omega_p$ are called poles. Note that
poles and zeros can be real, imaginary, or complex, and can have positive or negative values.
(Throughout this problem, be careful to distinguish the complex-valued $T$ from the real-valued $\abs{T}$ and
$\angle T$)

\begin{enumerate}[\alph*)]

\item

What are the poles and zeros of $T = 3 \frac{\left( 1 - \frac{j \omega}{100} \right)\left( 1 - \frac{j \omega}{1,000} \right)}{\left( 1 - \frac{j \omega}{10} \right)\left( 1 - \frac{j \omega}{10,000} \right)}$ , assuming that the numerical frequencies
have units of \SI{}{\radian / \second}? Are they real, imaginary, or complex? Are they positive or negative?

\item

The scaling factor $A$ is often called the ``dc gain.'' Explain why by letting $\omega \rightarrow 0$. Are there any
values of $\omega_z$ or $\omega_p$ that would ruin this interpretation?

\item

Sketch the magnitude of $T$ on a log-log plot (y axis in \SI{}{\decibel}) as follows:

\begin{itemize}

\item

Start from low frequency. What is $T$ approximately equal to? What is $\abs{T}$ approximately
equal to in this region? At what frequency does this approximation break down? Sketch
$\abs{T}$ up until this frequency.

\item

Above the frequency you just found, what is $T$ approximately equal to? What should
the slope of $\abs{T}$ be in this region? At what frequency does this approximation break
down? Continue sketching $\abs{T}$, starting where you left off and ending at the frequency
you just found.

\item

Continue increasing frequency, identifying the slope of $\abs{T}$ in each region until you have
a complete graph

\end{itemize}

\item

As frequency increases, what do you notice about the slope of $\abs{T}$ each time you hit a pole?
Each time you hit a zero?

\item

Sketch the phase of $T$ on a semilog-x plot in a similar way:

\begin{itemize}

\item

Start from low frequency. What is $T$ approximately equal to? What is the phase of $T$ in
this region? At what frequency does this approximation break down? Sketch the phase
of $T$ up until this frequency.

\item

Above the frequency you just found, what is $T$ approximately equal to? What is the
phase of $T$ in this region? At what frequency does this approximation break down?
Sketch the phase of $T$ in this frequency region. Roughly sketch a smooth connection
from the phase in the last region to the phase in this region.

\item

Continue increasing frequency, identifying the phase of $T$ in each region until you have a
complete graph

\end{itemize}

\item

As frequency increases, what do you notice about the phase of $T$ each time you hit a pole? Each
time you hit a zero?

\item

Suppose $A$ were changed from $3$ to $30$. Describe in words how the magnitude and phase plots
would change.

\end{enumerate}

\end{problem}

\pagebreak

\begin{soln}
\end{soln}

\begin{enumerate}[\alph*)]

\item

This transfer function has poles ($-10, -10,000$) and zeros ($-100, -1,000$) that only have a real part and are negative.

\item

By letting $\omega \rightarrow 0$, the terms with any poles $\frac{j \omega}{\omega _{p}}$ or zeros $\frac{j \omega}{\omega _{z}}$
zero out, letting $T = A$. This makes sense as an interpretation for DC gain.

This breaks down for $\omega_z = 0$ or $\omega_p = 0$. Since there would be
an indeterminate term of $\frac{0}{0}$ as $\omega \rightarrow 0$.

\item

Graph:

\begin{figure}[H]
\centering
\includegraphics[width=0.4\textwidth]{1c1.jpg}
\end{figure}

\begin{itemize}

\item

$0 < \omega < 10$: $\omega \rightarrow 0 \quad \Rightarrow \quad T \approx \abs{T} \approx 3$.
This approximation starts to break down as $\omega \rightarrow 10$

\item

$10 < \omega < 100$: $\abs{T} \propto \frac{1}{\omega}$. Since $\omega$
increases by a factor of $10$, there is a decrease of $\SI{20}{\decibel}$
in $\abs{T}$

\end{itemize}

\item

As frequency increases, the slope of $\abs{T}$ decreases each time a pole
is hit, and the slope increases each time a zero is hit.

\item

Graph:

\begin{figure}[H]
\centering
\includegraphics[width=0.4\textwidth]{1e1.jpg}
\end{figure}

\item

As frequency increases, the slope of $\abs{T}$ increases each time a pole
is hit and decreases each time a zero is hit.

\item

Since $A$ increased by a factor of $10$, the magnitude plot would be
shifted up by $\SI{20}{\decibel}$. However, since $A$ is still real and
positive, there is no affect on the phase.

\end{enumerate}

\pagebreak

\begin{problem}

\textbf{How Oscilloscope Probes Work and the ``Doublet'' Problem}

An oscilloscope probe can be modeled as one resistor $R_1$ in parallel with a capacitor $C_1$. This parallel
combination appears in series with the scope itself, which is modeled as another resistor $R_2$ in parallel
with a capacitor $C_2$. The bottom of $R_2$ and $C_2$ is ground, which connects back to the device under test,
completing the circuit. The midpoint voltage between $R_1$ and $R_2$ is what the oscilloscope itself
measures.

\begin{enumerate}[\alph*)]

\item

Use circuit analysis to determine the differential equation that relates the input (the voltage of
the device under test) and the output (the midpoint voltage) in the time domain.

\item

Let the input be a cosine wave, $v _{in}(t) = V _{in} \cos(\omega t)$. What form must the output $v _{out}(t)$ take,
and which parameters of $v _{out}(t)$ are known or unknown at this stage?

\item

Write the input and output voltages as the real part of complex exponentials.

\item

Plug your answers into the differential equation; perform any derivatives; reduce the problem
to $Re\{ A e ^{j \omega t} \} = Re\{ B e ^{j \omega t} \}$ and use the theorem we learned in class to conclude that $A = B$. In
the resulting equation, identify the input voltage phasor $\vec{V} _{in}$ and the output voltage phasor $\vec{V} _{out}$.

\item

Solve for the complex transfer function and calculate its magnitude and phase as functions of $\omega$.

\item

Prove that the magnitude and phase of the transfer function are independent of frequency if $\frac{R_1}{R_2} = \frac{C_1}{C_2}$.
(We call this ``pole-zero cancellation'').

\item

Suppose you have a square wave voltage source applied to the circuit input. If the above
condition is met, what do you expect the output (midpoint) voltage to look like? Why? (Hint:
How does the transfer function affect the different frequency components of a square wave?)

\item

Let $R_1 = \SI{90.9}{\kilo \ohm}$, $R_2 = \SI{9.42}{\kilo \ohm}$, $C_2 = \SI{145}{\pico \farad}$, $C_1 = \SI{15}{\pico \farad}$ and simulate the circuit in LTSpice.
Let the voltage source be a PULSE from $\SI{0}{\volt}$ to \SI{10}{\volt} with rise and fall times of $\SI{1}{\nano \second}$ at a frequency
of $\SI{10}{\kilo \hertz}$ and 50\% duty cycle (also set its AC amplitude to 1V for ac analysis simulations). Run
transient simulation from $0$ to $\SI{300}{\micro \second}$ and an ac simulation from $\SI{10}{\hertz}$ to $\SI{10}{\mega \hertz}$ (on the ac
simulation, adjust the magnitude scale to be from $\SI{-40}{\decibel}$ to $\SI{0}{\decibel}$ and the phase scale to be from
$\SI{-90}{\degree}$ to $\SI{+90}{\degree}$). Include both plots.

\item

Repeat (h) for $C_1 = \SI{20}{\pico \farad}$ and $C_1 = \SI{10}{\pico \farad}$. Include the plots and describe what you see in the
time domain simulation. (In the transfer function, the close-but-imperfect cancellation of poles
and zeros is called a ``doublet'')

\item

Suppose $C_1$ is adjustable (e.g., with a screwdriver), describe how you could properly tune an
oscilloscope probe.

\end{enumerate}

\begin{figure}[H]
\centering
\includegraphics[width=0.8\textwidth]{2.png}
\end{figure}

\end{problem}

\pagebreak

\begin{soln}
\end{soln}

\begin{enumerate}[\alph*)]

\item

Node analysis KCL:

\begin{align*}
\frac{v _{in} - v _{out}}{R_1} + C_1 \frac{d}{dt}(v _{in} - v _{out}) - \frac{v _{out}}{R_2} - C_2 \frac{d v _{out}}{dt} = 0\\
\end{align*}

\item

\begin{align*}
v _{out} (t) = V _{out} \cos (\omega t + \phi)
\end{align*}

We don't know the magnitude and the phase.

\item

\begin{align*}
v _{in} = Re \{ V _{in} e ^{j \omega t} \} \\
v _{out} = Re \{ V _{out} e ^{j (\omega t + \phi)} \}
\end{align*}

\item

\begin{align*}
\frac{1}{R_1} Re\{ V _{in} e ^{j \omega t} \} - \frac{1}{R_1} Re \{ V _{out} e ^{j (\omega t + \phi)} \} + C_1 \frac{d}{dt} (Re\{ V _{in} e ^{j \omega t} \} - Re \{ V _{out} e ^{j (\omega t + \phi)} \}) \\ 
- \frac{1}{R_2} Re \{ V _{out} e ^{j (\omega t + \phi)} \} - C_2 \frac{d}{dt} Re \{ V _{out} e ^{j (\omega t + \phi)} \} = 0
\end{align*}

\begin{align*}
\frac{1}{R_1} V _{in} e ^{j \omega t} - \frac{1}{R_1} V _{out} e ^{j \omega t} e ^{j \phi} + C_1 j \omega (V _{in} e ^{j \omega t} - V _{out} e ^{j \omega t} e ^{\phi}) - \frac{1}{R_2} V _{out} e ^{j \omega t} e ^{j \phi} - C_2 j \omega V _{out} e ^{j \omega t} e ^{j \phi} &= 0 \\
\frac{1}{R_1} V _{in} - \frac{1}{R_1} V _{out} e ^{j \phi} + C_1 j \omega V _{in} - C_1 j \omega V _{out} e ^{j \phi} - \frac{1}{R_2} V _{out} e ^{j \phi} - C_2 j \omega V _{out} e ^{j \phi} &= 0 \\
V _{in} e ^{j 0} (\frac{1}{R_1} + C_1 j \omega) = V _{out} e ^{j \phi} (\frac{1}{R_1} + \frac{1}{R_2} + C_1 j \omega + C_2 j \omega) \\
\end{align*}

\item

\begin{align*}
\frac{\vec{V _{out}}}{\vec{V _{in}}} = \frac{V _{out} e ^{j \phi}}{V _{in} e ^{j 0}} = \boxed{\frac{\frac{1}{R_1} + \frac{1}{R_2} + C_1 j \omega}{\frac{1}{R_1} + C_1 j \omega + C_2 j \omega}}
\end{align*}

\begin{align*}
\frac{V _{out}}{V _{in}} &= \abs{ \frac{\frac{1}{R_1} + C_1 j \omega}{\frac{1}{R_1} + \frac{1}{R_2} + C_1 j \omega + C_2 j \omega} } \\
\frac{V _{out}}{V _{in}} &= \abs{ \frac{\frac{1}{R_1} + j C_1 \omega}{(\frac{1}{R_1} + \frac{1}{R_2}) + j (C_1 \omega + C_2 \omega)} } \\
\frac{V _{out}}{V _{in}} &= \boxed{\frac{\sqrt{ \left( \frac{1}{R_1} \right)^{2} + (C_1 \omega)^{2} }}{\sqrt{ \left( \frac{1}{R_1} + \frac{1}{R_2} \right)^{2} + (C_1 \omega + C_2 \omega)^{2} }}}
\end{align*}

\begin{align*}
\frac{\vec{V _{out}}}{\vec{V _{in}}} &= \frac{\frac{1}{R_1} + \frac{1}{R_2} + C_1 j \omega}{\frac{1}{R_1} + C_1 j \omega + C_2 j \omega} \\
\frac{\vec{V _{out}}}{\vec{V _{in}}} &= \frac{M_1 e ^{j \arctan (C_1 \omega R_1)}}{M_2 e ^{j \arctan \left( \frac{C_1 \omega + C_2 \omega}{\frac{1}{R_1} + \frac{1}{R_2}} \right)}} = \frac{M_1}{M_2} e ^{j \left( \arctan (C_1 \omega R_1) - \arctan \left( \frac{C_1 \omega + C_2 \omega}{\frac{1}{R_1} + \frac{1}{R_2}} \right) \right)} \\
\phi &= \boxed{\arctan (C_1 \omega R_1) - \arctan \left( \frac{C_1 \omega + C_2 \omega}{\frac{1}{R_1} + \frac{1}{R_2}} \right)}
\end{align*}

\item

Let $R_1 C_1 = R_2 C_2$,

\begin{align*}
\frac{\vec{V _{out}}}{\vec{V _{in}}} = \boxed{\frac{R_2}{R_1 + R_2}}
\end{align*}

\item

\begin{align*}
\frac{V _{out}}{V _{in}} &= \frac{R_2}{R_1 + R_2} \\
\multicolumn{2}{c}{
\boxed{V _{out} = V _{in} \frac{R_2}{R_1 + R_2}}
}
\end{align*}

\item

Images (credit to Anh-Vu)

\begin{figure}[H]
\centering
\includegraphics[width=0.4\textwidth]{2h1.png}
\end{figure}

\begin{figure}[H]
\centering
\includegraphics[width=0.6\textwidth]{2h2.png}
\end{figure}

\begin{figure}[H]
\centering
\includegraphics[width=0.6\textwidth]{2h3.png}
\end{figure}

\item

When the ratio between $R_1$ and $R_2$ is similar to $C_1$ and $C_2$,
the frequency and output magnitude is almost constant. The more you deviate
from the ratio, the messier your plots get.

\begin{figure}[H]
\centering
\includegraphics[width=0.4\textwidth]{2i1.png}
\end{figure}

\begin{figure}[H]
\centering
\includegraphics[width=0.6\textwidth]{2i2.png}
\end{figure}

\begin{figure}[H]
\centering
\includegraphics[width=0.6\textwidth]{2i3.png}
\end{figure}

\item

Input a square wave from a signal source. Then tune $C_1$ until your
output and phase are both constant and your square wave is normal.

\end{enumerate}

\pagebreak

\begin{problem}

\textbf{Inductors and RLC Circuits}

An inductor is a coil of wire, sometimes wrapped around a magnetic core. Its component law is the dual
of the capacitor’s, namely $v_L = L \frac{di_L}{dt}$. Imagine an LCR circuit like the one below (which happens to be the
same as the circuit in your final project):

\begin{figure}[H]
\centering
\includegraphics[width=0.5\textwidth]{3.png}
\end{figure}

\begin{enumerate}[\alph*)]

\item

Find the differential equation that relates the input voltage to the output voltage.

\item

If the input is a sine wave, we needn't go through the whole rigmarole of writing it as the real
part of a complex exponential, plugging it in, etc. We've done this enough times to know what
will happen. Instead, look directly at the differential equation and replace $v _{in}$ and $v _{out}$ with
their phasors, recognizing that $\frac{d}{dt}v _{out}$ will turn into $j\omega \vec{v}_{out}$ and $\frac{d}{dt} v _{in}$ will turn into $j \omega \vec{v}_{in}$ in the
frequency domain.

\item

Solve for the complex transfer function of the circuit, namely $\frac{\vec{v}_{out}}{\vec{v}_{in}}$. Calculate the magnitude and
phase as a function of frequency $\omega$.

\item

On a log-log plot (Bode Plot), what is the slope of the magnitude at low frequencies (far below
the LC resonance, $\omega_r = 1 / \sqrt{LC}$) and high frequencies (far above resonance)?

\item

Create this circuit in LTSpice. Right click on the voltage source, choose ``advanced,'' and set the
AC amplitude to 1. Then, run a simulation but, in the dialog box, change from ``Transient'' to
``AC Analysis.'' Select ``Decade'' with 100 points per decade with a start frequency of $\SI{100}{\hertz}$ and
a stop frequency of $\SI{1}{\mega \hertz}$. Submit the plot of $v _{out}$ (since $v _{in}$ is a cosine wave with a
magnitude of 1 and a phase of 0, then plotting $v _{out}$ is the same as plotting $v _{out}/v _{in}$).

\item

A square wave, like all waves in the frequency domain, consists of a dc component, a
fundamental component at the square wave frequency, and a sequence of harmonics at twice,
thrice, and so forth of the fundamental frequency, of generally decreasing magnitude. If a
square wave of frequency much less than resonance is applied to the above circuit, what do you
expect the output to approximately look like? If a square wave of frequency much greater than
resonance is applied to the above circuit, what do you expect the output to approximately look
like?

\item

Verify your answer in LTSpice by right-clicking on the voltage source -> ``Pulse'' and assigning
the following parameters to create a $\SI{100}{\hertz}$ square wave that goes from $\SI{0}{\volt}$ to $\SI{20}{\volt}$. Change the
simulation type back to Transient and run the simulation for $\SI{50}{\milli \second}$. Submit the time-domain
plot of $v _{in}$ and $v _{out}$. Does this match your intuition about this case?

\begin{figure}[H]
\centering
\includegraphics[width=0.5\textwidth]{3g.png}
\end{figure}

\item

Further verify your intuition from the frequency domain by adjusting the voltage source to have
a frequency of $\SI{100}{\kilo \hertz}$. Run the simulation for $\SI{300}{\micro \second}$ but only saving the last $\SI{150}{\micro \second}$. Submit
the time-domain plot of Vin and Vout. Does this match your intuition about this case?

\item

For the $\SI{100}{\kilo \hertz}$ case, right-click on the plot and select view -> FFT (Fast Fourier Transform, i.e.
the frequency-domain representation of the signals). Make sure both $v _{in}$ and $v _{out}$ are selected
and click ok. Make sure $v _{in}$ and $v _{out}$ are selected for visibility and then click ok again. Submit
the plot.

\item

Finally, note that the dc value of the output is $\SI{10}{\volt}$ in the $\SI{100}{\kilo \hertz}$ case because the duty cycle of
the input, i.e. the fraction of time that it is $\SI{20}{\volt}$ instead of $\SI{0}{\volt}$, is 50\%. If the duty cycle were
25\%, what would you expect the dc value of $v _{out}$ to be?

\end{enumerate}

\end{problem}

\pagebreak

\begin{soln}
\end{soln}

\begin{enumerate}[\alph*)]

\item

Node Analysis KCL:

\begin{align*}
i_L = \frac{v _{out}}{R_1} + C_1 \frac{d}{dt} v _{out} \\
\frac{d}{dt} i_L = \frac{1}{R_1} \frac{d}{dt} v _{out} + C_1 \frac{d^2}{dt^2} v _{out}
\end{align*}

Inductor component law:

\begin{align*}
v _{in} - v _{out} &= L_1 \frac{d}{dt} i_L \\
v _{in} - v _{out} &= \frac{L_1}{R_1} \frac{d}{dt} v _{out} + L_1 C_1 \frac{d^2}{dt^2} v _{out} \\
\multicolumn{2}{c}{
\boxed{v _{in} = \frac{L_1}{R_1} \frac{d}{dt} v _{out} + L_1 C_1 \frac{d^2}{dt^2} v _{out} + v _{out}}
} \\
\end{align*}

\item

\begin{align*}
v _{in} &= \frac{L_1}{R_1} \frac{d}{dt} v _{out} + L_1 C_1 \frac{d^2}{dt^2} v _{out} + v _{out} \\
\multicolumn{2}{c}{
\boxed{V _{in} e ^{j 0} = V _{out} e ^{j \phi} \left( \frac{L_1}{R_1} j \omega + L_1 C_1 j^2 \omega^2 + 1 \right)}
}
\end{align*}

\item

\begin{align*}
\frac{\vec{V _{out}}}{\vec{V _{in}}} = \frac{V _{out} e ^{j \phi}}{V _{in} e ^{j 0}} = \boxed{\frac{1}{\frac{L_1}{R_1} j \omega - L_1 C_1 \omega^2 + 1}}
\end{align*}

\begin{align*}
\frac{V _{out}}{V _{in}} = \frac{1}{\sqrt{(1 - L_1 C_1 \omega ^{2})^{2} + \left( \frac{L_1 \omega}{R_1} \right)^{2}}}
\end{align*}

\begin{align*}
\boxed{\phi = -\arctan \left( \frac{L_1 \omega}{R_1 (1 - L_1 C_1 \omega^2)} \right)}
\end{align*}

\item

At low frequencies $\omega \rightarrow 0$, $\frac{V _{out}}{V _{in}} \rightarrow 1$,
there is no change in magnitude, $\boxed{\text{slope} = 0}$.

At high frequencies, $a \omega ^{2} >> b$, and $c \omega ^{4} >> d \omega^2$.

\begin{align*}
\frac{V _{out}}{V _{in}} &= \frac{1}{\sqrt{ (1 - L_1 C_1 \omega^2)^{2} + \left( \frac{L_1 \omega}{R_1} \right)^{2} }} \\
\frac{V _{out}}{V _{in}} &= \frac{1}{\sqrt{ (-L_1 C_1 \omega^2)^{2} + \left( \frac{L_1 \omega}{R_1} \right)^{2} }} \\
\frac{V _{out}}{V _{in}} &= \frac{1}{\sqrt{ L_1^2 C_1^2 \omega^4 + \frac{L_1^2}{R_1^2} \omega^2 }} \\
\frac{V _{out}}{V _{in}} &= \frac{1}{\sqrt{ L_1^2 C_1^2 \omega^4}} \\
\frac{V _{out}}{V _{in}} &= \frac{1}{L_1 C_1 \omega^2} \\
\frac{V _{out}}{V _{in}} &\propto \omega ^{-2}
\end{align*}

Slope is $\boxed{\SI{-40}{\decibel / decade}}$

\item

Images (credit to Anh-Vu)

\begin{figure}[H]
\centering
\includegraphics[width=0.4\textwidth]{3e1.png}
\end{figure}

\begin{figure}[H]
\centering
\includegraphics[width=0.6\textwidth]{3e2.png}
\end{figure}

\item

Much less than resonance: smaller magnitude than $V _{in}$

Much greater than resonance: larger magnitude

\item

\begin{figure}[H]
\centering
\includegraphics[width=0.6\textwidth]{3g1.png}
\end{figure}

\begin{figure}[H]
\centering
\includegraphics[width=0.6\textwidth]{3g2.png}
\end{figure}

Yes, my intuition is correct because $V _{in} < V _{out}$

\item

Image credit to (Anh-Vu)

\begin{figure}[H]
\centering
\includegraphics[width=0.6\textwidth]{3h1.png}
\end{figure}

\item

Image credit to (Anh-Vu)

\begin{figure}[H]
\centering
\includegraphics[width=0.6\textwidth]{3i1.png}
\end{figure}

\item

\begin{align*}
\boxed{\SI{5}{\volt}}
\end{align*}

\end{enumerate}

\pagebreak

\begin{problem}

\textbf{Rapid AC Solutions}

Having multiple times gone through the process of writing out the differential equations in the time
domain and then converting to the frequency domain, we are now prepared to do direct ac analysis in
the frequency domain. Treating all resistors, inductors, and capacitors as \textit{impedances} and using all the
same circuit analysis tools that we used for resistors, answer the following questions:

\begin{enumerate}[\alph*)]

\item

Find the total \textit{impedance} to the right of the voltage source and use your answer to solve for $i$.

\begin{figure}[H]
\centering
\includegraphics[width=0.7\textwidth]{4a.png}
\end{figure}

\item

Use node analysis to solve for $v_o$ if $\omega = \SI{2}{\kilo \radian / \second}$

\begin{figure}[H]
\centering
\includegraphics[width=0.7\textwidth]{4b.png}
\end{figure}

\item

Find the Thevenin equivalent for this circuit:

\begin{figure}[H]
\centering
\includegraphics[width=0.7\textwidth]{4c.png}
\end{figure}

\end{enumerate}

\end{problem}

\pagebreak

\begin{soln}
\end{soln}

\begin{enumerate}[\alph*)]

\item

\begin{align*}
Z _{eq} &= 1 + \frac{1}{1 + \frac{1}{j \omega} + j \omega} \\
Z _{eq} &= 1 + \frac{1}{1 + 0.1j + 10j} \\
Z _{eq} &= 1 + \frac{1}{10.149 e ^{1.472 j}} \\
Z _{eq} &= 1 + 0.099 e ^{-1.472 j} \\
Z _{eq} &= 1 + 0.010 - 0.099j \\
Z _{eq} &= 1.010 - 0.099j \\
Z _{eq} &= 1.015 e ^{-0.098 j}
\end{align*}

\begin{align*}
\vec{V} &= Z_{eq} \vec{i} \\
\vec{i} &= \frac{\vec{V}}{Z _{eq}} \\
\vec{i} &= \frac{2 e ^{j 0}}{1.015 e ^{-0.098 j}} \\
\vec{i} &= 1.970 e ^{0.098 j} \\
i (t) &= \boxed{1.970 \cos (10t + 0.098)}
\end{align*}

\item

Node analysis:

\begin{align*}
36 = \frac{\vec{v_x}}{2000} + \frac{\vec{v_x}}{j (2000)(0.050)} + j (2000) (0.000002) (\vec{v_x} - \vec{v_0}) \\
j (2000) (0.000002) (\vec{v_x} - \vec{v_0}) = 0.1 \vec{v_x} + \frac{\vec{v_0}}{4000}
\end{align*}

\begin{align*}
0.004 j \vec{v_x} - 0.004 j \vec{v_o} &= 0.1 \vec{v_x} + \frac{1}{4000} \vec{v_o} \\
\vec{v_x} (-0.1 + 0.004j) &= (\frac{1}{4000} + 0.004j) \vec{v_o} \\
\vec{v_x} &= \frac{\frac{1}{4000} + 0.004j}{-0.1 + 0.004j} \vec{v_o} \\
\vec{v_x} &= \frac{0.004 e ^{1.508 j}}{0.100 e ^{3.102 j}} \vec{v_o} \\
\vec{v_x} &= 0.04 e ^{1.594 j} \vec{v_o}
\end{align*}

\begin{align*}
36 e ^{1.571j} &= \vec{v_o} 0.00002 e ^{1.594j} - 0.0004 \vec{v_o} j e ^{1.594j} + 0.004 \vec{v_o} j ( 0.04 e ^{1.594j} - 1 ) \\
36 e ^{1.571j} &= \vec{v_o} 0.00002 e ^{1.594j} - 0.0004 \vec{v_o} j e ^{1.594j} + 0.004 \vec{v_o} ( - 1.0009 j - 0.040) \\
36 e ^{1.571j} &= \vec{v_o} 0.00002 e ^{1.594j} - 0.0004 \vec{v_o} j e ^{1.594j} - 0.004 \vec{v_o} j - 0.00016 \vec{v_o} \\
\vec{v_o} &= \frac{36 e ^{1.571j}}{0.00002 e ^{1.594j} - 0.0004j e ^{1.594j} - 0.004j - 0.00016} \\
\vec{v_o} &= \frac{36 e ^{1.571j}}{0.00398 e ^{1.511j}} \\
\vec{v_o} &= 9045 e ^{0.06j} \\
v_o (t) &= \boxed{9045 \cos (2000t + 0.06)}
\end{align*}

\item

Node analysis:

\begin{align*}
5 - \vec{v_1} &= - j \vec{v_1} + j \vec{v _{test}} + j \vec{v_1} \\
- j \vec{v_1} + j \vec{v _{test}} - \vec{i _{test}} &= 0
\end{align*}

\begin{align*}
\vec{v_1} &=  j \vec{i _{test}} + \vec{v _{test}} \\
5 - j \vec{i _{test}} - \vec{v _{test}} &= j \vec{ v _{test}} \\
\vec{v _{test}} (1 + j) &= 5 - j \vec{i _{test}} \\
\vec{v _{test}} &= \boxed{3.536 e ^{-0.785j} + 0.707 e ^{0.786j} \vec{i _{test}}}
\end{align*}

\end{enumerate}

\pagebreak

\begin{problem}

\textbf{Extra Credit: Degree Plan}

Choose the tech core that you are most likely to choose and craft an accelerated degree plan for that
tech core. You can use whatever criteria you want to craft your plan – including early and deep
concentration, intentional hybridization, maintaining the opportunity to pivot, and so forth – but your
plan must not be the same as the 4-year plan put forward by the ECE department. You can leave the
gen-ed classes unspecified, but you do have to say which semesters they will occur in. Choose the
elective ECE classes specifically, though. Write a short 1-2 paragraphs about the key features of your
degree plan and your reasoning for some of the key choices.

You are in no way committing to that tech core or the accelerated degree plan, but going through the
exercise will help you think creatively about your degree plan.

Only complete plans with sensible rationale will be awarded credit.

\end{problem}

\pagebreak

\begin{soln}
\end{soln}

The tech-core that interests me the most right now is Computer Architechture
and Embedded Systems. An accelerated plan I have for this tech-core involves
taking ECE 316 via study abroad through the Tokyo Tech program, completing
M 427J ECE 312H and ECE 319H my freshman fall, and jumping straight into
ECE 460N during my sophomore fall. This opens me up to take ECE 411 during
this semester as well, and then being able to take the core lab ECE 445L
during my sophomore spring. As for electives, I'm looking at ECE 361E,
ECE 445S, and ECE 461S.

\end{document}
