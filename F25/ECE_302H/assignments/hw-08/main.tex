\documentclass[11pt]{article}

\usepackage[utf8]{inputenc}
\usepackage[T1]{fontenc}
\usepackage{lmodern}
\usepackage{microtype}

\usepackage[sexy]{evan}

\usepackage{fancyhdr}
\pagestyle{fancy}
\fancyhf{}

\usepackage{amsmath, amssymb, amsthm, mathtools}
\usepackage[siunitx]{circuitikz}
\usepackage{graphicx}
\usepackage{float}
\usepackage{chemformula}

\usepackage{pgfplots}
\pgfplotsset{compat=1.18}
\usepgfplotslibrary{statistics}

\newcommand{\coursename}{\textbf{Introduction to Electrical Engineering}}
\newcommand{\coursecode}{ECE 302H}
\newcommand{\term}{Fall 2025}
\newcommand{\instructor}{Dr.\ Hanson}
\newcommand{\notetaker}{Dawson Zhang}
\newcommand{\lecturetitle}{Homework 8}

\fancyhead[L]{\coursecode}
\fancyhead[C]{\lecturetitle}
\fancyhead[R]{\term}
\fancyfoot[R]{\thepage}

\title{\coursename~(\coursecode) -- \lecturetitle}
\author{\notetaker~~|~~Instructor: \instructor}
\date{\term}

\begin{document}
  
\maketitle
\pagebreak

\setcounter{section}{8}

\begin{problem}

\textbf{Log Scale}

\begin{enumerate}[\alph*)]

\item

You read that a function of frequency ``rolls off at \SI{20}{\decibel} per decade.'' Describe what this means.

\item

Convert the ratio 40:1 to \SI{}{\decibel}

\item

Convert the ratio 3710:1 to \SI{}{\decibel}

\item

Convert \SI{65}{\decibel} to a ratio

\item

Convert \SI{33}{\decibel} to a ratio

\item

How many volts is \SI{65}{\decibel \micro \volt}? How many volts is \SI{33}{\decibel \milli \volt}? 

\item

A paper reports that an input signal has been attenuated by \SI{60}{\decibel}. Has the signal become
bigger or smaller, and by what factor?

\item

On a log plot, you want to plot a number between $X$ and $10X$ (e.g., between $10$ and $100$). If the
number you want to plot is $2.5X$, where will it appear on the plot (i.e., what percentage of the
distance from the $X$ tick mark to $10X$ tick mark). Does it matter what $X$ is?

\item

On a linear plot, a data point halfway between tick marks is the ``arithmetic mean'' of the tick
mark values (e.g., $2.5$ is the average of $2$ and $3$ and it appears halfway between the tick mark for
$2$ and the tick mark for $3$). Explain why a data point halfway between two tick marks on a log
plot represents the ``geometric mean'' of the tick mark values, i.e. $\sqrt{Tick_1 \times Tick_2}$.

\end{enumerate}

\end{problem}

\pagebreak

\begin{soln}
\end{soln}

\begin{enumerate}[\alph*)]

\item

The statement of a frequency ``rolling off at $\SI{20}{\decibel}$ per decade''
means that the function decreases by \SI{20}{\decibel} when the frequency
increases by a factor of $10$.

\item

\begin{align*}
20 \times \log{40} \\
\boxed{\SI{32.041}{\decibel}}
\end{align*}

\item

\begin{align*}
20 \times \log{3710} \\
\boxed{\SI{71.387}{\decibel}}
\end{align*}

\item

\begin{align*}
10 ^{\frac{65}{20}} \\
\boxed{1778:1}
\end{align*}

\item

\begin{align*}
10 ^{\frac{33}{20}} \\
\boxed{45:1}
\end{align*}

\item

\begin{align*}
\SI{65}{\decibel \micro \volt} &= 10^{\frac{65}{20}} \SI{}{\micro \volt} \\
\SI{65}{\decibel \micro \volt} &= \SI{1778}{\micro \volt} \\
\multicolumn{2}{c}{
\boxed{\SI{65}{\decibel \micro \volt} = \SI{1.778}{\milli \volt}}
}
\end{align*}

\begin{align*}
\SI{33}{\decibel \milli \volt} &= 10^{\frac{33}{20}} \SI{}{\milli \volt} \\
\multicolumn{2}{c}{
\boxed{\SI{33}{\decibel \milli \volt} = \SI{44.7}{\milli \volt}}
}
\end{align*}

\item

The signal would become $\boxed{\text{smaller}}$ by a factor of,

\begin{align*}
\boxed{10 ^{\frac{-60}{20}} = 0.001}
\end{align*}

\item

\begin{align*}
\text{tick} \% &= \log{\frac{2.5}{1}} \times 100 \% \\
\multicolumn{2}{c}{
\boxed{\text{tick} \% = 39.79 \%}
} 
\end{align*}

It does not depend on $X$.

\item

Consider the two tick marks $T_1$ and $T_2$, where $T_1 < T_2$ the value $x$ $50 \%$ between
the two ticks is,

\begin{align*}
0.5 &= \frac{\log{\frac{x}{T_1}}}{\log{\frac{T_2}{T_1}}} \\
0.5 \log{\frac{T_2}{T_1}} &= \log{\frac{x}{T_1}} \\
\log{\left( \frac{T_2}{T_1} \right)^{0.5}} &= \log \frac{x}{T_1} \\
\sqrt{\frac{T_2}{T_1}} &= \frac{x}{T_1} \\
\multicolumn{2}{c}{
\boxed{x = \sqrt{T_1 T_2}}
}
\end{align*}

\end{enumerate}

\pagebreak

\begin{problem}

\textbf{Sine Wave Technology}

\begin{enumerate}[\alph*)]

\item

For any function $f(x)$, we know that $f(x - a)$ represents a horizontal translation of the graph. If
$a > 0$, does the graph translate left or right?

\item

Consider a sine wave $A \cos (\omega t)$. What does the function $A \cos (\omega t + \phi)$ look like if $\phi > 0$? If
$\phi < 0$? Sketch both scenarios relative to the base sine wave.

\item

When $\phi > 0$, do we say that the new wave lags or leads the original? How about when $\phi < 0$?
In two or three sentences, explain why this terminology makes sense.

\item

It’s often easier to measure the size of a sine wave from ``peak to peak,'' meaning from its highest
point to its lowest point. How does the peak-to-peak size of the sine wave relate to the amplitude
$A$?

\item

If we look at a sine wave, we can’t measure $\omega$ directly. The easiest thing we can measure is the
period $T$. How would we calculate $\omega$ from $T$?

\item

Explain the difference between frequency $f$ and angular frequency $\omega$? (Not just the formulas)

\end{enumerate}

\end{problem}

\pagebreak

\begin{soln}
\end{soln}

\begin{enumerate}[\alph*)]

\item

The translation $f(x - a)$, where $a > 0$, translates the graph to the $\boxed{\text{right}}$ by $a$ units.

\item

Based on the answer to the previous question, we can conclude that for a
function $f(t) = A \cos (\omega t + \phi)$, 

\begin{align*}
\phi < 0 \quad &\Rightarrow \quad \text{right shift} \\
\phi > 0 \quad &\Rightarrow \quad \text{left shift}
\end{align*}

\begin{figure}[H]
\centering
\includegraphics[width=0.7\textwidth]{2b1.png}
\end{figure}

\begin{align*}
A &= 1.5 \\
\omega &= 1.5 \\
\textcolor{red}{\phi} &= 0 \\
\textcolor{green}{\phi} &= -1 \\
\textcolor{blue}{\phi} &= 1
\end{align*}

\item

When $\phi < 0$, the wave can be closely modeled by the green line on 
the previous graph where $\phi = -1$. As you can see, the green line lags behind the
base red line. Conversely, $\phi > 0$, which can be modeled by the blue
line $\phi = 1$, leads the base red line. This explanation makes
sense because relative to time ($x$-axis), the blue line experiences changes before
red line, which experiences changes before the green line.

\item

The highest point to its lowest point on a sine wave (peak-to-trough) is
double its amplitude. $\boxed{\text{peak-to-trough} = 2A}$

\item

\begin{align*}
\boxed{\omega = \frac{2 \pi}{T}}
\end{align*}

\item

Frequency measures the amount of cycles that something happened each second,
while angular frequency captures a full cycle of that same something
within the period of a circle ($2 \pi \SI{}{\radian}$ think $\sin$ and $\cos$). That's what
defines the relationship between the two,

\begin{align*}
\omega = 2 \pi f
\end{align*}

\end{enumerate}

\pagebreak

\begin{problem}

\textbf{Combining Sine Waves}

When two sine waves of the same frequency are added together, the result is a single sine wave of the
same frequency, even if the original sine waves had different phases. Let us derive the amplitude and
phase of the new sine wave. In particular, consider adding together two waves, $a = A \sin (\omega t)$ and
$b = B \sin (\omega t + \delta)$. The result is a new sine wave $c = C \sin (\omega t + \phi)$, where the amplitude $C$ and phase
$\phi$ are unknown.

\begin{enumerate}[\alph*)]

\item

We contend that $a + b = c$. In particular, this must be true for any time $t$. Plug in $t = 0$ and
solve for $\sin ( \phi )$. Your answer will include knowns and unknowns, but it’s a good constraint that
your final answer must obey.

\item

Now set $t = \pi / 2 \omega$ and solve for $\cos ( \phi )$, again in terms of knowns and unknowns.

\item

Use the fact that $\cos^2 (x) + \sin^2 (x) = 1$ to solve for $C$ only in terms of knowns. This is your final
answer for $C$. What trigonometric law does your answer remind you of?

\item

Plug your answer for $C$ back into part (a) or (b) and solve for $\phi$ only in terms of knowns. This is
your final answer for $\phi$.

\end{enumerate}

Note – later in the course, we will learn an easier way to add sine waves using complex numbers.

\end{problem}

\pagebreak

\begin{soln}
\end{soln}

\begin{enumerate}[\alph*)]

\item

\begin{align*}
a + b &= c \\
A \sin (\omega t) + B \sin (\omega t + \delta) &= C \sin (\omega t + \phi) \\
A \sin (0) + B \sin (\delta) &= C \sin (\phi) \\
\multicolumn{2}{c}{
\boxed{\sin (\phi) = \frac{B \sin (\delta)}{C}}
}
\end{align*}

\item

\begin{align*}
A \sin (\omega t) + B \sin (\omega t + \delta) &= C \sin (\omega t + \phi) \\
A \sin (\omega \times \frac{\pi}{2 \omega}) + B \sin (\omega \times \frac{\pi}{2 \omega} + \delta) &= C \sin (\omega \times \frac{\pi}{2 \omega} + \phi) \\
A \sin (\frac{\pi}{2}) + B \sin (\frac{\pi}{2} + \delta) &= C \sin (\frac{\pi}{2} + \phi) \\
A + B \cos (\delta) &= C \cos (\phi) \\
\multicolumn{2}{c}{
\boxed{\cos (\phi) = \frac{A + B \cos (\delta)}{C}}
}
\end{align*}

\item

\begin{align*}
\sin^2 (\phi) + \cos^2 (\phi) &= 1 \\
\left( \frac{B \sin (\delta)}{C} \right)^{2} + \left( \frac{A + B \cos (\delta)}{C} \right)^{2} &= 1 \\
B^2 \sin^2 (\delta) + A^2 + 2 A B \cos (\delta) + B^2 \cos^2 (\delta) &= C^2 \\
A^2 + B^2 (\sin^2 (\delta) + \cos^2 (\delta)) + 2 A B \cos (\delta) &= C^2 \\
A^2 + B^2 + 2 A B \cos (\delta) &= C^2 \\
\multicolumn{2}{c}{
\boxed{C = \sqrt{ A^2 + B^2 + 2 A B \cos (\delta) }}
}
\end{align*}

This equation looks a lot like the law of cosines.

\item

\begin{align*}
\sin (\phi) &= \frac{B \sin (\delta)}{C} \\
\sin (\phi) &= \frac{B \sin (\delta)}{\sqrt{ A^2 + B^2 + 2 A B \cos (\delta) }} \\
\multicolumn{2}{c}{
\boxed{\phi = \arcsin \left( \frac{B \sin (\delta)}{\sqrt{ A^2 + B^2 + 2 A B \cos (\delta) }} \right)}
}
\end{align*}

However, there is an easier way to solve for $\phi$. Notice the two
equations for $\sin (\phi)$ and $\cos (\phi)$,

\begin{align*}
\sin (\phi) &= \frac{B \sin (\delta)}{C} \\
\cos (\phi) &= \frac{A + B \cos (\delta)}{C} \\
\tan (\phi) &= \frac{\sin (\phi)}{\cos (\phi)} = \frac{B \sin (\delta)}{A + B \cos (\delta)} \\
\multicolumn{2}{c}{
\boxed{\phi = \arctan \left( \frac{B \sin (\delta)}{A + B \cos (\delta)} \right)}
}
\end{align*}

\end{enumerate}

\pagebreak

\begin{problem}

\textbf{Mixing Sine Waves}

One function that is commonly done (purposefully or not) to sine waves is to multiply them together.
This is known as ``mixing.''

\begin{enumerate}[\alph*)]

\item

Use the identity $\cos (\theta_1 \pm \theta_2) = \cos \theta_1 \cos \theta_2 \pm \sin \theta_1 \sin \theta_2$ to prove that $\cos \theta_1 + \cos \theta_2 = 2$
$\cos \left( \frac{\theta_1 - \theta_2}{2} \right) \cos \left( \frac{\theta_1 + \theta_2}{2} \right)$.

\item

When you listen to a speaker generate a sine wave, it sounds like a pure tone. If you add two
identical sine waves together, it still sounds like a pure tone. However, if you add two sine
waves together at slightly different frequencies, you hear a single tone play that slowly gets
louder and quieter. Use the above identity to explain this concept. If the sine waves are at
frequencies $f_1$ and $f_2$ which are close to each other, what tone do you expect to hear and how
long do you expect it to take for the tone to go from loud to quiet and back to loud again?

\item

In part (b), we explored what happens when you add two sine waves of different frequencies
and we gained intuition by thinking of the sum instead as a product of two sine waves. In rf
systems, we also do the opposite: we multiply two sine waves together and interpret the result
as a sum of sine waves. The most common rf receiver architecture is called a
``superheterodyne'' structure. It works by multiplying (``mixing'') the incoming rf signal at a
frequency $f _{rf}$ by another sine wave at frequency $f _{LO}$ (LO stands for local oscillator; it can be
lower or higher than $f _{rf}$). The difference between $f _{rf}$ and $f _{LO}$ is called the ``intermediate
frequency,'' $f _{IF}$. Suppose that you want to listen to KUT 90.5 which transmits at \SI{90.5}{\mega \hertz}, and
you multiply that signal with a local oscillator of the same magnitude at \SI{40.25}{\mega \hertz}. On a plot
versus frequency, sketch the rf signal, the local oscillator signal, and the frequencies of the signal
after mixing, each as impulses whose height corresponds to the magnitude of the sine wave.
Label the intermediate frequency and every other frequency both symbolically and numerically.
Make sure your magnitudes are roughly to scale.

\end{enumerate}

\end{problem}

\pagebreak

\begin{soln}
\end{soln}

\begin{enumerate}[\alph*)]

\item

\begin{align*}
2 \cos \left( \frac{\theta_1 - \theta_2}{2} \right) \cos \left( \frac{\theta_1 + \theta_2}{2} \right) \\
2 \left( \cos \left( \frac{\theta_1}{2} \right) \cos \left( \frac{\theta_2}{2} \right) + \sin \left( \frac{\theta_1}{2} \right) \sin \left( \frac{\theta_2}{2} \right) \right) \left( \cos \left( \frac{\theta_1}{2} \right) \cos \left( \frac{\theta_2}{2} \right) - \sin \left( \frac{\theta_1}{2} \right) \sin \left( \frac{\theta_2}{2} \right) \right) \\
2 \left( \left( \cos \left( \frac{\theta_1}{2} \right) \cos \left( \frac{\theta_2}{2} \right) \right)^{2} + \left( \sin \left( \frac{\theta_1}{2} \right) \sin \left( \frac{\theta_2}{2} \right) \right)^{2} \right) \\
2 \left( \cos^2 \left( \frac{\theta_1}{2} \right) \cos^2 \left( \frac{\theta_2}{2} \right) + \sin^2 \left( \frac{\theta_1}{2} \right) \sin^2 \left( \frac{\theta_2}{2} \right) \right) \\
2 \left( \frac{1 + \cos \theta_1}{2} \frac{1 + \cos \theta_2}{2} - \frac{1 - \cos \theta_1}{2} \frac{1 - \cos \theta_2}{2} \right) \\
\frac{1}{2} (1 + \cos \theta_1) (1 + \cos \theta_2) - \frac{1}{2} (1 - \cos \theta_1) (1 - \cos \theta_2) \\
\frac{1}{2} (1 + \cos \theta_1 + \cos \theta_2 + \cos \theta_1 \cos \theta_2) - \frac{1}{2} (1 - \cos \theta_1 - \cos \theta_2 + \cos \theta_1 \cos \theta_2) \\
\frac{1}{2} \cos \theta_1 + \frac{1}{2} \cos \theta_2 + \frac{1}{2} \cos \theta_1 + \frac{1}{2} \cos \theta_2 \\
\cos \theta_1 + \cos \theta_2
\end{align*}

\item

\begin{align*}
\theta_1 &= 2 \pi f_1 t \\
\theta_2 &= 2 \pi f_2 t
\end{align*}

For frequencies close togehter, $f_2 \rightarrow f_1 \quad \Rightarrow \quad f_2 = f_1$

\begin{align*}
\theta_2 &= 2 \pi f_1 t \\
\theta_1 &= \theta 2
\end{align*}

\begin{align*}
2 \cos \left( \frac{\theta_1 - \theta_2}{2} \right) \cos \left( \frac{\theta_1 + \theta_2}{2} \right) \\
2 \cos (0) \cos \left( \frac{4 \pi f_1 t}{2} \right) \\
2 \cos (2 \pi f_1 t)
\end{align*}

\begin{remark*}
Notice that the amplitude is constant, $2$.
\end{remark*}

For frequencies far apart, $f_2 \neq f_1$,

\begin{align*}
\theta_1 &= 2 \pi f_1 t \\
\theta_2 &= 2 \pi f_2 t
\end{align*}

\begin{align*}
2 \cos \left( \frac{\theta_1 - \theta_2}{2} \right) \cos \left( \frac{\theta_1 + \theta_2}{2} \right) \\
2 \cos \left( \frac{2 \pi f_1 t - 2 \pi f_2 t}{2} \right) \cos \left( \frac{2 \pi f_1 t + 2 \pi f_2 t}{2} \right) \\
2 \cos (\pi t (f_1 - f_2)) \cos (\pi t (f_1 + f_2))
\end{align*}

If we think about $\cos (\pi t (f_1 + f_2))$ as the ``base'' wave,
the amplitude of this wave can be modeled by $2 \cos (\pi t (f_1 - f_2))$,
since $A \times \cos(\omega t + \phi)$.

Notice that if $f_1 - f_2 = 0 \Rightarrow f_1 = f_2$, we would get the
previous case with an amplitude of $2 \cos (\pi t (f_1 - f_1)) = 2$

Now to find the period of this wave, AKA the time the it takes for it
to go from loud to quiet and get back loud again, conceptually this
would just be the time difference between consequtive max amplitudes.

\begin{align*}
2 \cos (\pi t (f_1 - f_2)) &= 2 \\
\pi t (f_1 - f_2) &= k \pi \\
t = \frac{k}{f_1 - f_2}&, \quad \text{for } f_1 \neq f_2
\end{align*}

\begin{align*}
T &= t _{n+1} - t_n \\
T &= \frac{n + 1}{f_1 - f_2} - \frac{n}{f_1 - f_2} \\
\multicolumn{2}{c}{
\boxed{T = \frac{1}{f_1 - f_2}}
}
\end{align*}

\item

\begin{align*}
b = 2 \pi f _{rf} t \\
a = 2 \pi f _{LO} t
\end{align*}

\begin{align*}
\cos (a) \cos(b) &= \cos \left( \frac{\theta_1 - \theta_2}{2} \right) \cos \left( \frac{\theta_1 + \theta_2}{2} \right) \\
a &= \frac{\theta_1 - \theta_2}{2} \\
b &= \frac{\theta_1 + \theta_2}{2} \\
\end{align*}

\begin{align*}
\theta_1 &= 2a + \theta_2 \\
\frac{2a + \theta_2 + \theta_2}{2} &= b \\
a + \theta_2 &= b \\
\theta_2 &= b - a = 2 \pi t (f _{rf} - f _{LO}) \\
\theta_1 &= a + b = 2 \pi t (f _{rf} + f _{LO})
\end{align*}

\begin{align*}
\cos \left( \frac{\theta_1 - \theta_2}{2} \right) \cos \left( \frac{\theta_1 + \theta_2}{2} \right) &= \frac{1}{2} \cos \theta_1 + \frac{1}{2} \cos \theta_2 \\
&= \boxed{\frac{1}{2} \cos (2 \pi t (f _{rf} + f _{LO})) + \frac{1}{2} \cos (2 \pi t (f _{LO} - f _{rf}))}
\end{align*}

\begin{align*}
f _{rf} &= \SI{90.5}{\mega \hertz} \\
f _{LO} &= \SI{40.25}{\mega \hertz} \\
f _{IF} &= \SI{50.25}{\mega \hertz}
\end{align*}

\begin{figure}[H]
\centering
\includegraphics[width=0.7\textwidth]{4c1.jpg}
\end{figure}

\end{enumerate}

\pagebreak

\begin{problem}

\textbf{Balanced 3-Phase Systems}

In power transmission and distribution systems, it is common to use ``three-phase'' power with the source
and load divided into three parts. Usually the voltages and resistances are balanced, meaning $v_a(t) + v_b(t) + v_c(t) = 0$
and $R_a = R_b = R_c$.

\begin{figure}[H]
\centering
\includegraphics[width=0.7\textwidth]{5.png}
\end{figure}

\begin{enumerate}[\alph*)]

\item

If the central node of the source (also known as the ``neutral'' voltage) is at ground, what will the
central node of the load be? Use node analysis to prove it.

\item

One way to ensure balancing ($v_a(t) + v_b(t) + v_c(t) = 0$) is to have each voltage be a sine wave,
each phase-shifted by a third of a cycle, i.e. $v_a(t) = V \cos (\omega t)$, $v_b(t) = V \cos \left( \omega t + \frac{2 \pi}{3} \right)$, and $v_c(t) = V \cos \left( \omega t - \frac{2 \pi}{3} \right)$. Show that their sum is always zero. Hint: recall the following
trigonometric identities:

\begin{align*}
\sin (\alpha + \beta) &= \sin (\alpha) \cos (\beta) + \sin (\beta) \cos (\alpha) \\
\sin (- \alpha) &= - \sin (\alpha) \\
\cos (- \alpha) &= \cos (\alpha)
\end{align*}

\end{enumerate}

\end{problem}

\pagebreak

\begin{soln}
\end{soln}

\begin{enumerate}[\alph*)]

\item

Node analysis at the load node ($V_L$),

\begin{align*}
\frac{V_a - V_L}{R_a} + \frac{V_b - V_L}{R_b} + \frac{V_c - V_L}{R_c} &= 0 \\
\frac{V_a}{R_a} + \frac{V_b}{R_b} + \frac{V_c}{R_c} &= \frac{V_L}{R_a} + \frac{V_L}{R_b} + \frac{V_L}{R_c} \\
\frac{V_a}{R_a} + \frac{V_b}{R_b} + \frac{V_c}{R_c} &= V_L \left( \frac{1}{R_a} + \frac{1}{R_b} + \frac{1}{R_c} \right) \\
\multicolumn{2}{c}{
\boxed{V_L = \left( \frac{\frac{V_a}{R_a} + \frac{V_b}{R_b} + \frac{V_c}{R_c}}{\frac{1}{R_a} + \frac{1}{R_b} + \frac{1}{R_c}} \right)}
}
\end{align*}

Balanced case,

\begin{align*}
V_L &= \left( \frac{\frac{V_a}{R_a} + \frac{V_b}{R_b} + \frac{V_c}{R_c}}{\frac{1}{R_a} + \frac{1}{R_b} + \frac{1}{R_c}} \right) \\
V_L &= \left( \frac{\frac{V_a}{R} + \frac{V_b}{R} + \frac{V_c}{R}}{\frac{1}{R} + \frac{1}{R} + \frac{1}{R}} \right) \\
V_L &= \left( \frac{V_a + V_b + V_c}{3} \right) \\
V_L &= \left( \frac{0}{3} \right) \\
\multicolumn{2}{c}{
\boxed{V_L = 0}
}
\end{align*}

\item

\begin{align*}
v_a(t) + v_b(t) + v_c(t) \\
V \cos (\omega t) + V \cos \left( \omega t + \frac{2 \pi}{3} \right) + V \cos \left( \omega t - \frac{2 \pi}{3} \right) \\
V \cos (\omega t) + V \cos (\omega t) \cos \frac{2 \pi}{3} - V \sin (\omega t) \sin \frac{2 \pi}{3} + V \cos (\omega t) \cos \frac{2 \pi}{3} + V \sin (\omega t) \sin \frac{2 \pi}{3} \\
V \cos (\omega t) + V \cos (\omega t) \cos \frac{2 \pi}{3} + V \cos (\omega t) \cos \frac{2 \pi}{3} \\
V \cos (\omega t) \left( 1 + 2 \cos \frac{2 \pi}{3} \right) \\
V \cos (\omega t) \left( 1 + 2 \left( - \frac{1}{2} \right) \right) \\
V \cos (\omega t) (0) \\
\multicolumn{2}{c}{
\boxed{V \cos (\omega t) (0) = 0}
}
\end{align*}

\end{enumerate}

\pagebreak

\end{document}
