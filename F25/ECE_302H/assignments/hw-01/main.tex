\documentclass[11pt]{article}

\usepackage[utf8]{inputenc}
\usepackage[T1]{fontenc}
\usepackage{lmodern}
\usepackage{microtype}

\usepackage[sexy]{evan}

\usepackage{fancyhdr}
\pagestyle{fancy}
\fancyhf{}

\usepackage{amsmath, amssymb, amsthm, mathtools}
\usepackage{siunitx}
\usepackage{graphicx}
\usepackage{float}

\usepackage{pgfplots}
\pgfplotsset{compat=1.18}
\usepgfplotslibrary{statistics}

\newcommand{\coursename}{\textbf{Introduction to Electrical Engineering}}
\newcommand{\coursecode}{ECE 302H}
\newcommand{\term}{Fall 2025}
\newcommand{\instructor}{Dr.\ Hanson}
\newcommand{\notetaker}{Dawson Zhang}
\newcommand{\lecturetitle}{Homework 1}

\fancyhead[L]{\coursecode}
\fancyhead[C]{\lecturetitle}
\fancyhead[R]{\term}
\fancyfoot[R]{\thepage}

\title{\coursename~(\coursecode) -- \lecturetitle}
\author{\notetaker~~|~~Instructor: \instructor}
\date{\term}

\begin{document}
  
\maketitle
\pagebreak

\section{Problems}

\begin{problem}
LTSpice Tutorial
\end{problem}

\begin{remark*}
Not copying in the entire problem statement... refer to the HW sheet
\end{remark*}

\begin{soln}
\end{soln}

\textbf{K)} Vout = \SI{1.5}{\volt}, Vout2 = \SI{1.5}{\volt}

\textbf{I)} 

\begin{tikzpicture}
\begin{axis}[
  xlabel = {Vout}, ylabel = {iR5}
]
\addplot[only marks] coordinates{
    (1.428, 0.00001428)
    (1.0, 0.00009993)
    (0.25, 0.00025)
};
\end{axis}
\end{tikzpicture}

\begin{tikzpicture}
\begin{axis}[
  xmin = 1, xmax = 2,
  xlabel = {Vout2}, ylabel = {iR6}
]
\addplot[only marks] coordinates{
    (1.5, 0.000015)
    (1.5, 0.00014997)
    (1.5, 0.0016)
};
\end{axis}
\end{tikzpicture}

\begin{figure}[H]
\centering
\includegraphics[width=0.9\textwidth]{Circuit1.png}
\end{figure}

\textbf{M)} The main difference between the two circuits is that the
second one always outputs a constant voltage. This is advantageous
because there might be electrical components that require a certain
voltage or may experience unindented behaviors when operating at lower
or higher voltages.

\pagebreak

\begin{problem}
System of Equations by Hand
\begin{enumerate}

\item Solve for $x$ and $y$:
\begin{align*}
2x + 6y &= 36 \\
7x - 3y &= 6
\end{align*}

\item Solve for $x$, $y$, and $z$:
\begin{align*}
\frac{1}{8}x + \frac{1}{2}y - \frac{1}{4}z &= -\frac{1}{2} \\
\frac{2}{3}x - \frac{1}{4}y + \frac{1}{6}z &= \frac{7}{2} \\
           \frac{1}{3}x + y - \frac{1}{3}z &= \frac{2}{3}
\end{align*}

\item Solve for $x$, $y$, and $z$ in terms of $a$:
\begin{align*}
4x - 2y + 6z &= 10 + 18a \\
 6x + 4y - z &= 22 \\
2x + 8y - 5z &= 14 - 16a \\
\end{align*}

\end{enumerate}
\end{problem}

\begin{soln}
\end{soln}

\begin{enumerate}

\item
Multiply the second equation by 2:
\begin{align*}
7x - 3y &= 6 \\
2(7x - 3y) &= 2(6) \\
14x - 6y &= 12
\end{align*}
Add both equations:
\[
\left\{
\begin{aligned}
2x + 6y &= 36 \\
14x - 6y &= 12
\end{aligned}
\right.
\quad \Rightarrow \quad
16x = 48
\]
\begin{equation*}
\boxed{x = 3}  
\end{equation*}
Substitute $x$ back into first equation:
\begin{align*}
2x + 6y &= 36 \\
2(3) + 6y &= 36 \\
6y &= 30 \\
\end{align*}
\begin{equation*}
\boxed{y = 5}  
\end{equation*}

\item
Multiply the second equation by 2:
\begin{align*}
\frac{2}{3}x - \frac{1}{4}y + \frac{1}{6}z &= \frac{7}{2} \\
2\left(\frac{2}{3}x - \frac{1}{4}y + \frac{1}{6}z\right) &= 2\left(\frac{7}{2}\right) \\
\frac{4}{3}x - \frac{1}{2}y + \frac{1}{3}z &= 7 
\end{align*}
Add with the first equations together:
\[
\left\{
\begin{aligned}
\frac{1}{8}x + \frac{1}{2}y - \frac{1}{4}z &= -\frac{1}{2} \\
\frac{4}{3}x - \frac{1}{2}y + \frac{1}{3}z &= 7 \\
\end{aligned}
\right.
\quad \Rightarrow \quad
\frac{35}{24}x + \frac{1}{12}z = \frac{13}{2}
\]
Multiply the first equation by -2:
\begin{align*}
\frac{1}{8}x + \frac{1}{2}y - \frac{1}{4}z &= -\frac{1}{2} \\
-2\left(\frac{1}{8}x + \frac{1}{2}y - \frac{1}{4}z\right) &= -2\left(-\frac{1}{2}\right) \\
-\frac{1}{4}x - y + \frac{1}{2}z &= 1
\end{align*}
Add with the third equation:
\[
\left\{
\begin{aligned}
-\frac{1}{4}x - y + \frac{1}{2}z &= 1 \\
\frac{1}{3}x + y - \frac{1}{3}z &= \frac{2}{3}
\end{aligned}
\right.
\quad \Rightarrow \quad
\frac{1}{12}x + \frac{1}{6}z = \frac{5}{3}
\]
\begin{remark*}
With these two new equations, we have converted the $x$ and $z$ parts
of the problem to a 2 equation 2 unknown system.
\end{remark*}
Solve $x$-$z$ system:
\[
\left\{
\begin{aligned}
-2\left(\frac{35}{24}x + \frac{1}{12}z\right) &= -2\left(\frac{13}{2}\right) \\
\frac{1}{12}x + \frac{1}{6}z &= \frac{5}{3}
\end{aligned}
\right.
\quad \Rightarrow \quad
-\frac{17}{6}x = -\frac{34}{3}
\]
\begin{equation*}
\boxed{x = 4} 
\end{equation*}
\begin{align*}
\frac{1}{12}(4) + \frac{1}{6}z &= \frac{5}{3} \\
\frac{1}{3} + \frac{1}{6}z &= \frac{5}{3} \\
\frac{1}{6}z = \frac{4}{3} \\
\boxed{z = 8}
\end{align*}
Plug $x$ and $z$ terms back into first equation:
\begin{align*}
\frac{1}{8}(4) + \frac{1}{2}y - \frac{1}{4}(8) &= - \frac{1}{2} \\
\frac{1}{2}y &= 1 \\
\boxed{y = 2}
\end{align*}

\item
Eliminate $y$ from first and second euqations:
\[
\left\{
\begin{aligned}
2(4x - 2y + 6z) &= 2(10 + 18a) \\
6x + 4y - z &= 22
\end{aligned}
\right.
\quad \Rightarrow \quad
14x + 11z = 42 + 36a
\]
Eliminate $y$ from first and third equations:
\[
\left\{
\begin{aligned}
4(4x - 2y + 6z) &= 4(10 + 18a) \\
2x + 8y - 5z &= 14 - 16a
\end{aligned}
\right.
\quad \Rightarrow \quad
18x + 19z = 54 + 56a
\]
Solve for $x$ and $z$:
\[
\left\{
\begin{aligned}
-\frac{9}{7}\left(14x + 11z\right) &= -\frac{9}{7}\left(42 + 36a\right) \\
18x + 19z &= 54 + 56a
\end{aligned}
\right.
\quad \Rightarrow \quad
\frac{34}{7}z = \frac{68}{7}a
\]
\begin{equation*}
\boxed{z = 2a} 
\end{equation*}
\begin{align*}
14x + 11(2a) &= 42 + 36a \\
14x &= 42 + 14a 
\end{align*}
\begin{equation*}
\boxed{x = a + 3}
\end{equation*}
Solve for $y$ using second equation:
\begin{align*}
6(a + 3) + 4y - (2a) &= 22 \\
4y &= -4a + 4
\end{align*}
\begin{equation*}
\boxed{y = -a + 1}
\end{equation*}

\end{enumerate}

\pagebreak

\begin{problem}
Charge Carriers and Current
\end{problem}

While current into and out of batteries is mediated by electrons, internally the flow of electricity is
mediated by ions. Traditional batteries rely on a single species of ion (for example, Lithium ions Li+)
traveling between the cathode (+) and anode (-). Novel “dual-ion” batteries have two species of ion
flowing simultaneously, such as batteries based on sodium hexafluorophosphate (NaPF6). Imagine a
battery oriented in the z direction with the cathode at the top (positive z) and the anode at the bottom
(negative z). If $3 \times 10^{20}$ hexafluorophosphate ions (PF6-) go through the interior of the 
battery in the positive z direction in 4 seconds, and simultaneously $1.5 \times 10^{20}$ sodium ions (Na+)
go through the same cross section in the negative z direction, what is the magnitude and direction of the 
current flowing through the cross section? Is the battery being charged or discharged?

\begin{soln}
\end{soln}

The magnitude of electrical current can be determined by calculating the
net current through the battery. This first involves finding the current
caused by each individual ion.

\begin{align*}
I &= \frac{C}{s} \\
C &= 6.24 \times 10^{18}e
\end{align*}

\begin{remark*}
Because the problem statement mentions the parallels between electrons
and ions when talking about current in batteries, I'm making the assumption
that the numerical number of ions can be used in place of electrons, and
that the difference in electrons between hexafluorophosphate and sodium
can be disregarded.
\end{remark*}

\begin{align*}
C_{\text{PF6-}} &= \frac{3 \times 10^{20}}{6.24 \times 10^{18}} (-e) = -48.08 \\
C_{\text{Na+}} &= \frac{1.5 \times 10^{20}}{6.24 \times 10^{18}} (+e) = 24.04 \\
i_{\text{PF6-}} &= \frac{-48.08}{4} = \SI{-12.02}{\ampere}_{+z} = \SI{12.02}{\ampere}_{-z} \\
i_{\text{Na+}} &= \frac{24.04}{4} = \SI{6.01}{\ampere}_{-z}
\end{align*}

\begin{equation*}
\sum I = 12.02 + 6.01 = \boxed{\SI{18.03}{\ampere}_{-z}}
\end{equation*}

$\boxed{\text{Because the net current flows from cathode to anode, the battery is
being charged.}}$

\pagebreak

\begin{problem}
Power and Energy
\end{problem}

I would like to bake some cookies in my electric oven which operates at 120 V. During pre-heating, the
oven operates at its rated power of 0.6 kW. However, after pre-heating, the oven only needs to consume
0.25 kW to maintain temperature. How much current does the oven draw in each case? How much energy
does it consume during the 20-minute pre-heat and the 12-minute baking time? How many batches of
cookies would I need to bake to use the same amount of energy in baking as I did in pre-heating?

\begin{soln}
\end{soln}

\begin{align*}
P &= IV \\
I &= \frac{P}{V} \\
I_{\text{pre-heat}} &= \frac{600}{120} = \boxed{\SI{5}{\ampere}} \\
I_{\text{heated}} &= \frac{250}{120} = \boxed{\SI{2.08}{\ampere}}
\end{align*}

\begin{remark*}
The approach I'm taking for this next step comes from knowing that
potential is a joule per coulomb. If current is defined as a coulomb per
second, we can find the amount of charge (coulombs) in both scenarios
and add them because we have already found the amperage for each case.
\end{remark*}

\begin{align*}
I &= \frac{C}{s} \\
C &= Is \\
C_{\text{pre-heat}} &= (5)(20 \times 60) = \SI{6000}{\coulomb} \\
C_{\text{heated}} &= (2.08)(12 \times 60) = \SI{1500}{\coulomb}
\end{align*}

\begin{align*}
V &= \frac{J}{C} \\
J &= VC \\
E_{\text{pre-heat}} &= (120)(6000) = \boxed{\SI{720}{\kilo\joule}} \\
E_{\text{heated}} &= (120)(1500) = \boxed{\SI{180}{\kilo\joule}}
\end{align*}

To find the number of batches of cookies to bake to equal the amount of
energy it took in pre-heating, assuming each batch of cookies takes
12-minutes to bake, we would just divide the pre-heating energy by the
eneregy it took one batch to bake.

\begin{equation*}
  \text{Batches} = \frac{720}{180} = \boxed{4} 
\end{equation*}

\pagebreak

\begin{problem}
Passive Sign Convention
\end{problem}

Before solving a problem, we usually label the voltage across and current through each component with
variable names and reference directions. Four students labeled a component’s voltage and current before
solving a problem, each in a different way, as shown in the Figure (the Figure shows a resistor, but it
could be any component). That actual polarity and magnitude of the voltage and current are also in the
Figure. The students all solved the problem correctly. For each student, please answer the following:

\begin{enumerate}
\item How would the student write down their numerical answer for the voltage across and current
through the resistor? In other words, what are ia, va, ib, and vb?
\item Does the student’s labeling conform to the passive sign convention?
\item If the student calculated power as v × i, what numerical answer would they get?
\item Would the student interpret their power number as power entering or leaving the resistor?
\end{enumerate}

Finally, write a few sentences answering whether it is necessary to follow the passive sign convention
when assigning reference directions, and what advantages or disadvantages it may provide to follow the
convention.

\begin{soln}
\end{soln}

\begin{enumerate}
\item
\begin{enumerate}[label = \alph*)]
\item
\[
\boxed{i_a = \SI{1}{\ampere} \quad v_a = \SI{5}{\volt}}
\]
\item
\boxed{\textbf{Yes}}
\item
\[
P = v \times i = \boxed{\SI{5}{\watt}}
\]
\item
The student would interpret power as entering the resistor
\end{enumerate}

\item
\begin{enumerate}[label = \alph*)]
\item
\[
\boxed{i_a = \SI{1}{\ampere} \quad v_a = \SI{-5}{\volt}}
\]
\item
\boxed{\textbf{No}}
\item
\[
P = v \times i = \boxed{\SI{-5}{\watt}}
\]
\item
The student would \textbf{still} interpret power as entering the 
resistor because in the end they all solved the problem correctly.
\end{enumerate}

\item
\begin{enumerate}
\item
\[
\boxed{i_a = \SI{-1}{\ampere} \quad v_a = \SI{5}{\volt}}
\]
\item
\boxed{\textbf{No}}
\item
\[
P = v \times i = \boxed{\SI{-5}{\watt}}
\]
\item
The student would \textbf{still} interpret power as entering the 
resistor because in the end they all solved the problem correctly.
\end{enumerate}

\item
\begin{enumerate}
\item
\[
\boxed{i_a = \SI{-1}{\ampere} \quad v_a = \SI{-5}{\volt}}
\]
\item
\boxed{\textbf{Yes}}
\item
\[
P = v \times i = \boxed{\SI{5}{\watt}}
\]
\item
The student would interpret power as entering the resistor
\end{enumerate}

\end{enumerate}

The passive sign convention is by no means necessary when assigning
reference directions, however, utilizing it allows the numerical value
of power to be directly interpreted. And, it also helps reduce the amount
of negative-positive sign flipping.

\pagebreak

\begin{problem}
Solving a Circuit
\end{problem}

\begin{remark*}
Not copying in the entire problem statement... refer to the HW sheet
\end{remark*}

\begin{soln}
\end{soln}

\begin{enumerate}[label=\alph*)]

\item
If different voltage DERs are connected together in the same grid, the
path of least resistance would pass through both DERs and result in a
short circuit, which could potentially set fire to the DERs.

\item
Refer to the following figure for correct labeling of the circuit:
\begin{figure}[H]
\centering
\includegraphics[width=0.8\textwidth]{DER_circuit2.png}
\end{figure}

Find KVL equations:
\begin{align}
\sum V &= V_1 + V_a + V_c = 0 \label{eq:1} \\
\sum V &= V_2 + V_b + V_c = 0 \label{eq:2} \\
\sum V &= V_1 + V_a - V_b - V_2 = 0 \label{eq:3}
\end{align}

Find KCL equations:
\begin{align}
\sum I &= I_a - I_{s1} = 0 \Rightarrow I_a = I_{s1} \label{eq:4} \\
\sum I &= I_b - I_{s2} = 0 \Rightarrow I_b = I_{s2} \label{eq:5} \\
\sum I &= I_c - I_a - I_b = 0 \label{eq:6} \\
\sum I &= I_{s1} + I_{s2} - I_c = 0 \Rightarrow I_c = I_{s1} + I_{s2} \label{eq:7}
\end{align}

Find I-V relationships:
\begin{align}
V_1 &= V_1 \quad \text{(constant)} \\
V_2 &= V_2 \quad \text{(constant)} \\
V_a &= I_aR_d = I_{s1}R_d \quad \text{\eqref{eq:4}} \\
V_b &= I_bR_d = I_{s2}R_d \quad \text{\eqref{eq:5}} \\
V_c &= I_cR = (I_{s1} + I_{s2})R \label{eq:12} \quad \text{\eqref{eq:7}} \\
\end{align}

Substitute I-V relationships into KVL equations:
\begin{align*}
V_1 + V_a + V_c &= 0 \quad \text{\eqref{eq:1}} \\
V_1 + I_{s1}R_d + (I_{s1} + I_{s2})R &= 0 \quad \text{\eqref{eq:4}} \text{\eqref{eq:7}} \\ \\
V_2 + V_b + V_c &= 0 \quad \text{\eqref{eq:2}} \\
V_2 + I_{s2}R_d + (I_{s1} + I_{s2})R &= 0 \quad \text{\eqref{eq:5}} \text{\eqref{eq:7}} \\
\end{align*}

\begin{remark*}
Notice now we have a 2-unknown 2-equation system of equations which we can
solve for $I_{s1}$ and $I_{s2}$.
\end{remark*}

\[
\left\{
\begin{aligned}
V_1 + I_{s1}R_d + (I_{s1} + I_{s2})R &= 0 \\
- \quad V_2 + I_{s2}R_d + (I_{s1} + I_{s2})R &= 0 \\
\end{aligned}
\right.
\quad \Rightarrow \quad
V_1 + I_{s1}R_d - V_2 - I_{s2}R_d = 0
\]

\begin{align*}
I_{s2} = \frac{V_1 + I_{s1}R_d - V_2}{R_d}
\end{align*}

Substitute $I_{s2}$ back in and solve for $I_{s1}$:
\begin{align*}
V_1 + I_{s1}R_d + (I_{s1} + \frac{V_1 + I_{s1}R_d - V_2}{R_d})R &= 0 \\
V_1 + I_{s1}R_d + (\frac{V_1 + 2I_{s1}R_d - V_2}{R_d})R &= 0 \\
V_1 + I_{s1}R_d + \frac{RV_1 + 2RI_{s1}R_d - RV_2}{R_d} &= 0 \\
V_1 + I_{s1}R_d + 2RI_{s1} + \frac{RV_1 - RV_2}{R_d} &= 0 \\
V_1 + \frac{RV_1 - RV_2}{R_d} &= -I_{s1}R_d - 2RI_{s1} \\
\frac{V_1R_d + RV_1 - RV_2}{R_d} &= I_{s1}(-R_d - 2R) \\
\frac{V_1R_d + RV_1 - RV_2}{R_d(-R_d - 2R)} &= I_{s1} \\
\boxed{\frac{V_1R_d + RV_1 - RV_2}{-R_d^2 -2RR_d}} &= I_{s1} = I_a \quad \text{\eqref{eq:4}} \\
\end{align*}

Substitute $I_{s1}$ back in and solve for $I_{s2}$:
\begin{align*}
I_{s2} &= \frac{V_1 + I_{s1}R_d - V_2}{R_d} \\
&= \frac{V_1 + \left(\frac{V_1R_d + RV_1 - RV_2}{-R_d^2 -2RR_d} \right)R_d - V_2}{R_d} \\
&= \frac{V_1}{R_d} + \frac{V_1R_d + RV_1 - RV_2}{-R_d^2 -2RR_d} - \frac{V_2}{R_d} \\
&= \frac{V_1(-R_d - 2R)}{R_d(-R_d - 2R)} + \frac{V_1R_d + RV_1 - RV_2}{-R_d^2 -2RR_d} - \frac{V_2(-R_d - 2R)}{R_d(-R_d - 2R)} \\
&= \frac{-V_1R_d - 2RV_1}{-R_d^2 -2RR_d} + \frac{V_1R_d + RV_1 - RV_2}{-R_d^2 -2RR_d} - \frac{-V_2R_d - 2RV_2}{-R_d^2 -2RR_d} \\
&= \boxed{\frac{V_2R_d + RV_2 - RV_1}{-R_d^2 -2RR_d}} = I_b \quad \text{\eqref{eq:5}}
\end{align*}

Calculate grid voltage $V_c$:
\begin{align*}
V_c &= (I_{s1} + I_{s2})R \quad \text{\eqref{eq:12}} \\
&= \left(\frac{V_1R_d + RV_1 - RV_2}{-R_d^2 -2RR_d} + \frac{V_2R_d + RV_2 - RV_1}{-R_d^2 -2RR_d} \right)R \\
&= \left(\frac{V_1R_d + V_2R_d}{-R_d^2 -2RR_d} \right)R \\
&= \boxed{\left(\frac{V_1 + V_2}{-R_d -2R} \right)R} \\
\end{align*}

Calculate ratio $\frac{I_a}{I_b}$:
\begin{align*}
\frac{I_a}{I_b} &= \frac{\frac{V_1R_d + RV_1 - RV_2}{-R_d^2 -2RR_d}}{\frac{V_2R_d + RV_2 - RV_1}{-R_d^2 -2RR_d}} \\
\frac{I_a}{I_b} &= \boxed{\frac{V_1R_d + RV_1 - RV_2}{V_2R_d + RV_2 - RV_1}} \\
\end{align*}

\item
Rewrite as functions of $R_d$:
\begin{align*}
\frac{I_a}{I_b}(R_d) &= \frac{V_1R_d + RV_1 - RV_2}{V_2R_d + RV_2 - RV_1} \\
V_c(R_d) &= \left(\frac{V_1 + V_2}{-R_d -2R} \right)R
\end{align*}

Substitute in new given numerical constants:
\begin{align*}
\frac{I_a}{I_b}(R_d) &= \frac{V_1R_d + RV_1 - RV_2}{V_2R_d + RV_2 - RV_1} = \frac{(105)R_d + (10)(105) - (10)(115)}{(115)R_d + (10)(115) - (10)(105)} = \frac{105R_d - 100}{115R_d + 100} \\
V_c(R_d) &= \left(\frac{V_1 + V_2}{-R_d -2R} \right)R = \left(\frac{105 + 115}{-R_d -2(10)} \right)(10) = \frac{2200}{-R_d - 20} \\
\frac{V_c(R_d)}{110} &= \frac{20}{-R_d - 20}
\end{align*}

\begin{remark*}
When graphing these functions next, we only have to graph the
magnitude of each function because the polarity of the current
and voltage are known when given numerical constants.
\end{remark*}

\begin{figure}[H]
\centering
\includegraphics[width=\textwidth]{DER_graph1}
\end{figure}

\item
From the graph, we can see that when $R_d$ gets too small there will be
\textbf{high voltage} but a \textbf{greater difference} between the two currents, and when $R_d$ gets too
big there will be \textbf{low voltage} but a \textbf{smaller difference} between currents, which
represents the trade off when choosing $R_d$.

\begin{remark*}
It's important to note here that the red graph represents the ratio of
of currents $\frac{I_a}{I_b}$ and not a numerical value for current,
unlike the voltage graph. This means that as the ratio gets closer to $1$,
the numerical current values get closer to each other, and vice versa.
\end{remark*}

The way I like to think about this question is in terms of maximizing a
function. It makes logical sense to set that function as some sort of a
sum of $\left|\frac{I_a}{I_b}(R_d)\right|$ and $\left|\frac{V_c(R_d)}{110}\right|$. 

\begin{equation*}
f(R_d) = n \times \left|\frac{I_a}{I_b}(R_d)\right| \quad + \quad (1 - n) \times \left|\frac{V_c(R_d)}{110}\right|
\end{equation*}

where $n$ is a coefficient from $0$ to $1$.

If we set $n=0.5$ that could be interpreted as an equal ``weight''
between voltage and current, and the resulting graph would look like:

\begin{figure}[H]
\centering
\includegraphics[width=\textwidth]{DER_graph2.png}
\end{figure}

The value of $R_d$ that maximizes $f(R_d)$ when $n = 0.5$ is $R_d = 6.884$.

\begin{remark*}
Depending on how you want to optimize between the ratio of voltage to 
current, a ``good'' value of $R_d$ would change.
\end{remark*}

\end{enumerate}

\end{document}
