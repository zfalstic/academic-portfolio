\documentclass[11pt]{article}

\usepackage[utf8]{inputenc}
\usepackage[T1]{fontenc}
\usepackage{lmodern}
\usepackage{microtype}

\usepackage[sexy]{evan}

\usepackage{fancyhdr}
\pagestyle{fancy}
\fancyhf{}

\usepackage{amsmath, amssymb, amsthm, mathtools}
\usepackage[siunitx]{circuitikz}
\usepackage{graphicx}
\usepackage{float}
\usepackage{chemformula}
\usepackage{booktabs}

\usepackage{pgfplots}
\pgfplotsset{compat=1.18}
\usepgfplotslibrary{statistics}

\usepackage{listings}
\usepackage{xcolor}

\lstset{
  basicstyle=\ttfamily\small,
  keywordstyle=\color{blue},
  commentstyle=\color{gray},
  stringstyle=\color{teal},
  frame=single,
  breaklines=true
}

\newcommand{\coursename}{\textbf{Introduction to Electrical Engineering}}
\newcommand{\coursecode}{ECE 302H}
\newcommand{\term}{Fall 2025}
\newcommand{\instructor}{Dr.\ Hanson}
\newcommand{\notetaker}{Dawson Zhang}
\newcommand{\lecturetitle}{Homework 9}

\fancyhead[L]{\coursecode}
\fancyhead[C]{\lecturetitle}
\fancyhead[R]{\term}
\fancyfoot[R]{\thepage}

\title{\coursename~(\coursecode) -- \lecturetitle}
\author{\notetaker~~|~~Instructor: \instructor}
\date{\term}

\begin{document}
  
\maketitle
\pagebreak

\setcounter{section}{9}

\begin{problem}

\textbf{Python Tutorial}

\begin{remark*}
Check HW document for full problem
\end{remark*}

\end{problem}

\pagebreak

\begin{soln}
\end{soln}

\begin{enumerate}[\alph*)]

\setcounter{enumi}{8}

\item Code:

\begin{lstlisting}[language=Python]
np.arange(-20, 24, 4)
\end{lstlisting}

\setcounter{enumi}{16}

\item

Picture for code and final graph:

\begin{figure}[H]
\centering
\includegraphics[width=0.8\textwidth]{1q.png}
\end{figure}

\end{enumerate}

\pagebreak

\begin{problem}

\textbf{Fourier Series Plotting}

Consider the sawtooth wave. One cycle of this wave looks like a straight line $f(\omega t) = \omega t$ from $- \pi$ to $\pi$.
This cycle repeats in the positive and negative $\omega$ directions forever and is therefore periodic. The Fourier
series for this wave is given by

\begin{align*}
f (\omega t) = 2 \sum \frac{(-1) ^{n + 1}}{n} \sin (n \omega t)
\end{align*}

Do the following in a single Python script and submit both your script and the final plots. Be sure to include
a title, legend, x label, and y label. Note that you may not have been explicitly shown how to do each and
every step here – a big part of learning to program is learning to research the commands for the tasks you
want to perform.

\begin{enumerate}[\alph*)]

\item

Plot the sawtooth wave as a function of $\omega t$ from $-3 \pi$ to $3 \pi$.

\item

Create a vector of $\omega t$ and use it to calculate a vector that represents the fundamental of the
Fourier series. Plot it on top of the original sawtooth wave in a different color and style.

\item

Calculate the sum of the first 3 harmonics of the sawtooth wave and plot it on top of the previous
two plots in a different color and style.

\item

Do the same again for the sum of the first 10 harmonics.

\item

On a separate plot with two subplots, use the stem command (if you imported matplotlib as plt,
then the stem plot command is plt.stem) to plot the magnitude and phase of the first 10
harmonics of the Fourier series of the sawtooth wave. Plot the magnitude on a log-log plot and
the phase on a semilog-x plot.

\end{enumerate}

\end{problem}

\pagebreak

\begin{soln}
\end{soln}

\begin{lstlisting}[language=Python]
#!/usr/bin/env python3
# -*- coding: utf-8 -*-
"""
Created on Wed Nov  5 19:47:47 2025

@author: dawson
"""

import numpy as np
import matplotlib.pyplot as plt

INFINITY = 1000
STEP = 0.001

x = np.arange(-3 * np.pi, 3 * np.pi, STEP)

y = ((x + np.pi) % (2 * np.pi)) - np.pi
    
y2 = 2 * (-1) ** (1 + 1) * (1 / 1) * np.sin(1 * x)

y3 = 0
for n in range(1, 4):
    y3 += 2 * (-1) ** (n + 1) * (1 / n) * np.sin(n * x)
    
y4 = 0
for n in range(1, 11):
    y4 += 2 * (-1) ** (n + 1) * (1 / n) * np.sin(n * x)
    
fig, axes = plt.subplots(1, constrained_layout=True)

fig.suptitle('ECE302 HW9 Problem 2 - Fourier Series Plotting')

axes.plot(x, y, label='Sawtooth')
axes.plot(x, y2, label='Fundamental')
axes.plot(x, y3, label='Sum of first 3 harmonics')
axes.plot(x, y4, label='Sum of first 10 harmonics')
axes.set_xlabel('Angular Displacement (rad)')
axes.set_ylabel('Amplitude')
axes.legend()


x = np.arange(1, 11)
y = np.abs(2 * (-1 ** (x + 1)) * (1 / x))
y2 = []

for n in range(1, 11):
    if n % 2 == 0:
        y2.append(np.pi)
    else:
        y2.append(0)

fig, axes = plt.subplots(2, constrained_layout=True)

fig.suptitle('ECE302 HW9 Problem 2 - Fourier Series 10th Harmonic Frequency Domain')

axes[0].stem(x, y)
axes[0].set_xscale('log')
axes[0].set_yscale('log')
axes[0].set_title('Magnitude vs. Angular Frequency')
axes[0].set_xlabel('Angular Frequency (log)')
axes[0].set_ylabel('Magnitude (log)')

axes[1].stem(x, y2)
axes[1].set_xscale('log')
axes[1].set_title('Phase vs. Angular Frequency')
axes[1].set_xlabel('Angular Frequency (log)')
axes[1].set_ylabel('Phase')
\end{lstlisting}

\begin{figure}[H]
\centering
\includegraphics[width=0.8\textwidth]{21.png}
\end{figure}

\begin{figure}[H]
\centering
\includegraphics[width=0.8\textwidth]{22.png}
\end{figure}

\pagebreak

\begin{problem}

\textbf{Quadrature Amplitude Modulation for Wireless Communication}

Wireless communication systems send signals using cosine waves at high frequencies (anywhere from
$\SI{}{\mega \hertz}$ to several $\SI{}{\giga \hertz}$). These cosine waves are modulated, or changed slightly over time, to represent
different digital values. Older modulation schemes would modulate frequency (FM) or amplitude (AM) to
convey information. Modern systems modulate both the amplitude and the phase to convey more
information in a single wave. As we've learned in class, a cosine wave with an amplitude and a phase can
also be represented as a cosine wave plus a sine wave at the same frequency, each with their own
amplitude but no phase. Wireless communications people usually think about their wave in the cosine
sine form. The cosine part is called the ``in-phase'' component, and the sine part is called the ``quadrature''
component. Modulating both parts is called ``Quadrature Amplitude Modulation'' (QAM).

\begin{figure}[H]
\centering
\includegraphics[width=0.6\textwidth]{3.jpg}
\end{figure}

\begin{enumerate}[\alph*)]

\item

There are two variables that can be adjusted, the cosine amplitude and the sine amplitude
(equivalent to the amplitude and phase of a single wave). These two variables can be plotted on
a graph such as the Figure. Redraw the graph and label the x-axis ``I'' to represent the in-phase
(cosine) amplitude. Label the y-axis ``Q'' to represent the quadrature (sin) amplitude.

\item

The Figure represents a 16-QAM system. Each dot represents a digital value. How many bits are
needed to represent all of the possible x coordinates? How many bits are needed to represent
all of the possible y coordinates? Therefore, how many bits of information can be contained in a
single incoming wave?

\item

One way to assign digital values to the dots is to use ``Gray Coding.'' Gray coding is not standard
binary counting – it's an alternative in which increasing the number by one only flips a single bit
from 1 to 0 or 0 to 1. Look up Gray Code in Wikipedia and create a table that shows the
numbers 0 through 15, their Gray code representation, and also their ordinary binary code
representation.

\item

Use Gray coding to assign digital values to the dots in your figure. Use the two MSBs to
represent the in-phase coordinate from most-negative to most-positive. Use the two LSBs to
represent the quadrature component from most-negative to most-positive. A system that
receives a cosine wave with a certain amplitude and phase will decompose that wave into its
cosine (in-phase) and sine (quadrature) parts, look at their amplitudes, and find the closest dot
on the graph – the system will then interpret the incoming cosine wave as the digital value
assigned to that dot.

\item

Each dot has an I and Q amplitude of either 1 or 3, so they are evenly spaced. Calculate the
amplitude-phase representation of the [1011], [1111], [1010], and [1110] dots. Do you notice
any correlation between the amplitude and phase of your answers and the locations of the dots
on the graph?

\item

If you wanted to have 64 dots instead of 16 without making the graph any larger (meaning,
without having to design a circuit that can deal with any larger voltages), how many times more
accurate would the transmitter need to be in creating I and Q components? How many times
faster could your bitrate be?

\end{enumerate}

\end{problem}

\pagebreak

\begin{soln}
\end{soln}

\begin{enumerate}[\alph*)]

\item

Labeled Axis:

\begin{figure}[H]
\centering
\includegraphics[width=0.6\textwidth]{3a.jpg}
\end{figure}

\item

$\boxed{\text{2 bits}}$ for 4 unique x-coordinate

$\boxed{\text{2 bits}}$ for 4 unique y-coordinate

$\boxed{\text{4 bits}}$ of information $2 + 2 = 4$

\item

Table:

\begin{center}
\begin{tabular}{c c c}
\toprule
\textbf{Decimal} & \textbf{Binary} & \textbf{Gray} \\
\midrule
0  & 0000 & 0000 \\
1  & 0001 & 0001 \\
2  & 0010 & 0011 \\
3  & 0011 & 0010 \\
4  & 0100 & 0110 \\
5  & 0101 & 0111 \\
6  & 0110 & 0101 \\
7  & 0111 & 0100 \\
8  & 1000 & 1100 \\
9  & 1001 & 1101 \\
10 & 1010 & 1111 \\
11 & 1011 & 1110 \\
12 & 1100 & 1010 \\
13 & 1101 & 1011 \\
14 & 1110 & 1001 \\
15 & 1111 & 1000 \\
\bottomrule
\end{tabular}
\end{center}

\item

Labeled Ticks and Points:

\begin{figure}[H]
\centering
\includegraphics[width=0.6\textwidth]{3d.jpg}
\end{figure}

\item

[1011]: (3, 1). $\boxed{A = \sqrt{10}}$, $\boxed{\phi = \arctan(\frac{1}{3}) = \SI{18.435}{\degree}}$

[1111]: (1, 1). $\boxed{A = \sqrt{2}}$, $\boxed{\phi = \arctan(\frac{1}{1}) = \SI{45}{\degree}}$

[1010]: (3, 3). $\boxed{A = 3\sqrt{2}}$, $\boxed{\phi = \arctan(\frac{3}{3}) = \SI{45}{\degree}}$

[1110]: (1, 3). $\boxed{A = \sqrt{10}}$, $\boxed{\phi = \arctan(\frac{3}{1}) = \SI{71.565}{\degree}}$

All of these four points reside in the first quadrant.

\item

The transmitter would need to be $\boxed{\text{2 times more}}$ accurate in producing
in each component (I and Q).

In total, the bitrate would be $\boxed{\text{4 times faster}}$ since $2 \times 2 = 4$.

\end{enumerate}

\pagebreak

\begin{problem}

\textbf{Complex Arithmetic Boot Camp}

Complex numbers can be expressed as a real part plus an imaginary part, $z = r + qj$, where $j = \sqrt{-1}$.
This is called rectangular form. They can also be expressed as a complex exponential, $z = Me ^{j \phi}$. This is
called polar form. In polar form, $\abs{M}$ is called the magnitude, i.e. the distance from (0,0) to $z$ on the
complex plane, and $\phi$ is called the angle or the argument, i.e. the angle from the real (+x) axis to $z$ on
the complex plane. It is easier to add/subtract complex numbers in rectangular form, while it is easier to
multiply/divide complex numbers in polar form.

\begin{enumerate}[\alph*)]

\item

Express $z = 3 + 4j$ as a complex exponential (polar form) and plot it on the complex plane

\item

Express $z = -6 + 8j$ as a complex exponential (polar form) and plot it on the complex plane

\item


Express $z = -j - 4j$ as a complex exponential (polar form) and plot it on the complex plane

\item

Express $z = 2j$ as a complex exponential (polar form) and plot it on the complex plane

\item

Express $z = \sqrt{-1 + j}$ as a complex exponential (polar form) and plot it on the complex plane

\item

Express $z = 2e ^{j \pi / 6}$ in rectangular form and plot in on the complex plane

\item

Express $z = -3e ^{-j \pi / 4}$ in rectangular form and plot in on the complex plane

\item

Express $z = \sqrt{3} e ^{j 3 \pi / 4}$ in rectangular form and plot in on the complex plane

\item

Express $z = -j^3$ in rectangular form

\item

Express $z = -j ^{-4}$ in rectangular form

\item

Express $z = (2 + j)^{2}$ in rectangular form and polar form

\item

Express $z = (3 - 2j)^{3}$ in rectangular form and polar form

\item

Calculate the magnitude and phase of $T = \frac{1 + 3j}{2 - 7j}$

\item

Calculate the magnitude and phase of $T = \frac{1}{3 + 4j}$

\end{enumerate}

\end{problem}

\pagebreak

\begin{soln}
\end{soln}

\begin{enumerate}[\alph*)]

\item

\begin{align*}
z &= 3 + 4j \\
z &= \sqrt{3^2 + 4^2} e ^{j \arctan(4/3)} \\
\multicolumn{2}{c}{
\boxed{z = 5 e ^{j 0.927}}
}
\end{align*}

\begin{figure}[H]
\centering
\includegraphics[width=0.4\textwidth]{4a.jpg}
\end{figure}

\item

\begin{align*}
z &= -6 + 8j \\
z &= \sqrt{6^2 + 8^2} e ^{j \arctan(8/-6)} \\
\multicolumn{2}{c}{
\boxed{z = 10 e ^{j (\pi - 0.927)}}
}
\end{align*}

\begin{figure}[H]
\centering
\includegraphics[width=0.4\textwidth]{4b.jpg}
\end{figure}

\item

\begin{align*}
z &= -j -4j \\
z &= \sqrt{5^2} e ^{j \arctan(-5/0)} \\
\multicolumn{2}{c}{
\boxed{z = 5 e ^{j 3 \pi / 2}}
}
\end{align*}

\begin{figure}[H]
\centering
\includegraphics[width=0.4\textwidth]{4c.jpg}
\end{figure}

\item

\begin{align*}
z &= 2j \\
z &= \sqrt{2^2} e ^{j \arctan(2/0)} \\
\multicolumn{2}{c}{
\boxed{z = 2 e ^{j \pi / 2}}
}
\end{align*}

\begin{figure}[H]
\centering
\includegraphics[width=0.4\textwidth]{4d.jpg}
\end{figure}

\item

\begin{align*}
z &= \sqrt{-1 + j} \\
z &= \sqrt{\sqrt{1^2 + 1^2} e ^{j \arctan(1/-1)}} \\
z &= \sqrt{\sqrt{2} e ^{j 3 \pi / 4}} \\
\multicolumn{2}{c}{
\boxed{z = 2 ^{1/4} e ^{j 3 \pi / 8}}
}
\end{align*}

\begin{figure}[H]
\centering
\includegraphics[width=0.4\textwidth]{4e.jpg}
\end{figure}

\item

\begin{align*}
z &= 2 e ^{j \pi / 6} \\
z &= 2 \cos (\pi / 6) + 2 \sin (\pi / 6) j \\
\multicolumn{2}{c}{
\boxed{z = \sqrt{3} + j}
}
\end{align*}

\begin{figure}[H]
\centering
\includegraphics[width=0.4\textwidth]{4f.jpg}
\end{figure}

\item

\begin{align*}
z &= -3 e ^{-j \pi / 4} \\
z &= -3 \cos (-\pi / 4) - 3 \sin (-\pi / 4) j \\
\multicolumn{2}{c}{
\boxed{z = -\frac{3 \sqrt{2}}{2} + \frac{3 \sqrt{2}}{2} j}
}
\end{align*}

\begin{figure}[H]
\centering
\includegraphics[width=0.4\textwidth]{4g.jpg}
\end{figure}

\item

\begin{align*}
z &= \sqrt{3} e ^{j 3 \pi / 4} \\
z &= \sqrt{3} \cos (3 \pi / 4) + \sqrt{3} \sin (3 \pi / 4) j \\
\multicolumn{2}{c}{
\boxed{z = - \frac{\sqrt{6}}{2} + \frac{\sqrt{6}}{2} j}
}
\end{align*}

\begin{figure}[H]
\centering
\includegraphics[width=0.4\textwidth]{4h.jpg}
\end{figure}

\item

\begin{align*}
z &= - j^3 \\
z &= - (-j) \\
\multicolumn{2}{c}{
\boxed{z = j}
}
\end{align*}

\item

\begin{align*}
z &= - j ^{-4} \\
z &= - \frac{1}{j^4} \\
\multicolumn{2}{c}{
\boxed{z = -1}
}
\end{align*}

\item

\begin{align*}
z &= (2 + j)^{2} \\
z &= j^2 + 4j + 4 \\
\multicolumn{2}{c}{
\boxed{z = 3 + 4j}
} \\
z &= \sqrt{3^2 + 4^2} e ^{j \arctan(4/3)} \\
\multicolumn{2}{c}{
\boxed{z = 5 e ^{j 0.927}}
}
\end{align*}

\item

\begin{align*}
z &= (3 - 2j)^{3} \\
z &= (\sqrt{3^2 + 2^2} e ^{j \arctan (-2/3)})^{3} \\
z &= (\sqrt{13} e ^{j 5.695})^{3} \\
\multicolumn{2}{c}{
\boxed{z = 13 ^{3/2} e ^{j 4.519}}
} \\
z &= 13 ^{3/2} \cos (4.519) + 13 ^{3/2} \sin (4.519) j \\
\multicolumn{2}{c}{
\boxed{z = -9 - 46j}
}
\end{align*}

\item

\begin{align*}
T &= \frac{1 + 3j}{2 - 7j} \\
T &= \frac{\sqrt{1^2 + 3^2} e ^{j \arctan(3/1)}}{\sqrt{2^2 + 7^2} e ^{j \arctan(-7/2)}} \\
T &= \frac{\sqrt{10} e ^{j 1.249}}{\sqrt{53} e ^{j 4.991}} \\
T &= \sqrt{\frac{10}{53}} e ^{j 2.541} \\
\multicolumn{2}{c}{
\boxed{\abs{M} = \sqrt{\frac{10}{53}}}
} \\
\multicolumn{2}{c}{
\boxed{\phi = \SI{2.541}{\radian}}
}
\end{align*}

\item

\begin{align*}
T &= \frac{1}{3 + 4j} \\
T &= (\sqrt{3^2 + 4^2} e ^{j \arctan(4/3)})^{-1} \\
T &= (5 e ^{j 0.927}) ^{-1} \\
T &= \frac{1}{5} e ^{j 5.356} \\
\multicolumn{2}{c}{
\boxed{\abs{M} = \frac{1}{5}}
} \\
\multicolumn{2}{c}{
\boxed{\phi = \SI{5.356}{\radian}}
}
\end{align*}

\end{enumerate}

\end{document}
