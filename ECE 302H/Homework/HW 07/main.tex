\documentclass[11pt]{article}

\usepackage[utf8]{inputenc}
\usepackage[T1]{fontenc}
\usepackage{lmodern}
\usepackage{microtype}

\usepackage[sexy]{evan}

\usepackage{fancyhdr}
\pagestyle{fancy}
\fancyhf{}

\usepackage{amsmath, amssymb, amsthm, mathtools}
\usepackage{siunitx}
\usepackage{graphicx}
\usepackage{float}
\usepackage{chemformula}

\usepackage{pgfplots}
\pgfplotsset{compat=1.18}
\usepgfplotslibrary{statistics}

\newcommand{\coursename}{\textbf{Introduction to Electrical Engineering}}
\newcommand{\coursecode}{ECE 302H}
\newcommand{\term}{Fall 2025}
\newcommand{\instructor}{Dr.\ Hanson}
\newcommand{\notetaker}{Dawson Zhang}
\newcommand{\lecturetitle}{Homework 7}

\fancyhead[L]{\coursecode}
\fancyhead[C]{\lecturetitle}
\fancyhead[R]{\term}
\fancyfoot[R]{\thepage}

\title{\coursename~(\coursecode) -- \lecturetitle}
\author{\notetaker~~|~~Instructor: \instructor}
\date{\term}

\begin{document}
  
\maketitle
\pagebreak

\setcounter{section}{7}

\begin{problem}

\textbf{Capacitors in Circuits}

In class, we mainly learned about capacitors in terms of the charge-voltage relationship, $Q = CV$. This
becomes a new circuit component for us. In the circuits context, as usual, we focus on it’s current-
voltage relationship, $i = C \frac{dV}{dt}$.

\begin{figure}[H]
\centering
\includegraphics[width=0.75\textwidth]{1.png}
\end{figure}

\begin{enumerate}[\alph*)]

\item

Write the KVL, KCL, and component law equations that define how this circuit works (as usual)

\item

Combine the equations into a single equation in which $v_c$ is the only unknown (as usual)

\item

Your equation will be a differential equation, which is different from the algebraic equations
we’ve seen so far. Let $V = 0$ and solve the differential equation assuming that the capacitor’s
voltage starts at $V _{ic}$. Write a phrase/sentence stating how the capacitor’s \textbf{initial condition}
enters into the calculation.

\item

Your solution will rely on the product $R \times C$. What are the units of $RC$? What is the significance
of this number, known as the time constant of the system?

\item

Create a table with the percent of the overall transition that the capacitor voltage has
completed after $N$ time constants, where $N = 1,2,3,4,5$.

\item

Use Thevenin’s Theorem to demonstrate that a system with a single capacitor (plus any number
of resistors and sources) will result in the same solution as above and therefore will be
characterized by a single time constant. Do the sources affect the time constant?

\item

Use LTSpice to simulate the RC circuit with $V = \SI{0}{\volt}$, $R = \SI{10}{\ohm}$, $C = \SI{3}{\micro \farad}$. Label the node below the
capacitor as ground and the node above the capacitor as ``vc.'' Place a ``spice directive''
anywhere on the schematic that says ``.ic V(vc)=5'' to set the initial condition on the capacitor.
Use a transient simulation and make sure to simulate for a long enough time to see the whole
transition but not so long of a time that you can’t see the transition. Include a screenshot of the
schematic and the capacitor voltage as your solution to this part.

\end{enumerate}

\end{problem}

\pagebreak

\begin{soln}
\end{soln}

\begin{enumerate}[\alph*)]

\item

KVL:

\begin{align*}
\sum V = V - v_R - v_c = 0
\end{align*}

KCL:

\begin{align*}
\sum i = i_R - i_c = 0
\end{align*}

Component Laws:

\begin{align*}
v_R &= i_R R \\
i_c &= C \frac{dv_c}{dt}
\end{align*}

\item

\begin{align*}
\frac{V - v_c}{R} &= C \frac{dv_c}{dt} \\
\frac{v_c}{R} + C \frac{dv_c}{dt} &= \frac{V}{R}
\end{align*}

\item

\begin{align*}
\frac{v_c}{R} + C \frac{dv_c}{dt} &= \frac{V}{R} \\
\frac{v_c}{R} + C \frac{dv_c}{dt} &= 0 \\
C \frac{dv_c}{dt} &= -\frac{v_c}{R} \\
\frac{RC}{v_c} dv_c &= -dt \\
\int \frac{RC}{v_c} dv_c &= \int -dt \\
RC \ln \abs{v_c} &= -t + C \\
\end{align*}

Here the initial condition enters the equation to solve for $C$,

\begin{align*}
RC \ln V _{ic} &= 0 + C \\
RC \ln V _{ic} &= C
\end{align*}

\begin{align*}
RC \ln \abs{v_c} &= -t + C \\
RC \ln \abs{v_c} &= -t + RC \ln V _{ic} \\
\ln \abs{v_c} &= -t/RC + \ln V _{ic} \\
e ^{\ln \abs{v_c}} &= e ^{-t/RC + \ln V _{ic}} \\
v_c &= V _{ic} e ^{-t/RC}
\end{align*}

\item

Resistance has units of $\SI{}{\volt / \ampere}$ while capacitance has units of $\SI{}{\ampere \cdot \second / \volt}$. 
Multiplying the two yeilds \SI{}{\second}. This number is significant because it controls the rate at which $v_c$ changes.

\item

\begin{align*}
v_c &= V _{ic} e ^{-N} \\
v _{\text{initial}} &= V _{ic} \\
v _{\text{final}} &= 0 \\
\text{\% transition} &= \frac{V _{ic} - V _{ic} e ^{-N}}{V _{ic} - 0} \\
\text{\% transition} &= 1 - e ^{-N}
\end{align*}

\begin{table}[h]
\centering
\begin{tabular}{c c c}
$N$ & $1-e^{-N}$ & \% Complete \\
\hline
1 & 0.6321 & 63.2\% \\
2 & 0.8647 & 86.5\% \\
3 & 0.9502 & 95.0\% \\
4 & 0.9817 & 98.2\% \\
5 & 0.9933 & 99.3\% \\
\hline
\end{tabular}
\end{table}

\end{enumerate}

\pagebreak

\begin{problem}

\textbf{Diode-Connected Device}

Sometimes NMOS transistors have their gates shorted to their drains as shown in the Figure. This is
known as a diode connected device.

\begin{enumerate}[\alph*)]

\item

Will the transistor operate in the cutoff, linear, or saturation region?

\item

Assume the MOSFET parameters and geometry $(\mu_n, C _{ox}, V_t, W, L)$ are known and ignore
channel length modulation $(\lambda = 0)$. Calculate the I-V characteristic of this two-terminal device
when $v$ is positive and when $v$ is negative. In what way is this characteristic similar to the
exponential behavior of a diode and in what way is it not similar?

\item

The NMOS transistor in the second Figure have $V_t = \SI{0.5}{\volt}$, $\mu_n C _{ox} = \SI{400}{\micro \ampere / \volt \squared}$, $\lambda = 0$,
$W = \SI{0.72}{\micro \meter}$, and $L = \SI{0.18}{\micro \meter}$. Find the value of $R$ that results in $V_D = \SI{0.7}{\volt}$.

\item

The third Figure is obtained by augmenting the second Figure with a transistor $Q_2$ identical to
$Q_1$ but with $W = \SI{2.16}{\micro \meter}$. As long as $Q_2$ is in the saturation region, what will $I _{D2}$ be? Does it
matter what value $R_2$ is?

\item

What value of $R_2$ results in $Q_2$ being at the edge of saturation? Should $R_2$ be bigger or smaller
than this value to keep $Q_2$ in saturation?

\end{enumerate}

\begin{figure}[H]
\centering
\includegraphics[width=0.75\textwidth]{2.png}
\end{figure}

\end{problem}

\pagebreak

\begin{soln}
\end{soln}

\begin{enumerate}[\alph*)]

\item

The transistor can be in one of the three states of cutuff, linear, or saturation. Because the gate is shorted to the drain on the NMOS 
transistor, $V _{DS} = V _{GS} = v$. 

Pinch-off occurs at,

\begin{align*}
V _{DS} &= V _{GS} - V_T \\
v &= v - V_T \\
V_T &= 0
\end{align*}

That means that the saturation condition becomes if $V_T \geq 0$.

And the linear condition becomes if $V_T < 0$.

Cutoff happens when $v < V_T$.

Assuming this is a normal MOSFET with $V_T \geq 0$, it will be in saturation unless it cutoffs at $v < V_T$.

\item

\[
i(v) =
\left\{
\begin{aligned}
&0, &v < V_T \text{(cutoff)} \\
&\frac{1}{2} \mu_n C _{ox} \frac{W}{L} (v - V_T)^{2}, &v \geq V_T \text{(saturation)}
\end{aligned}
\right.
\]

This is similar to a diode because it's a two terminal element with a threshold and strongly nonlinear increase in current.

This is different from a diode because it's quadratic and not exponential.

\item

\begin{align*}
i &= \frac{1}{2} \mu_n C _{ox} \frac{W}{L} (V _{GS} - V_T)^{2} \\
i &= \frac{1}{2} (\SI{400E-6}{}) \frac{\SI{0.72E-6}{}}{\SI{0.18E-6}{}} (0.7 - 0.5)^{2} \\
i &= \SI{32}{\micro \ampere}
\end{align*}

\begin{align*}
R &= \frac{V _{DD} - V _{D}}{i} \\
R &= \frac{1.8 - 0.7}{\SI{32E-6}{}} \\
R &= \boxed{\SI{34.375}{\kilo \ohm}}
\end{align*}

\item

\begin{align*}
i _{D2} &= \frac{1}{2} \mu_n C _{ox} \frac{W}{L} (V _{GS} - V_T)^{2} \\
i _{D2} &= \frac{1}{2} (\SI{400E-6}{}) \frac{\SI{2.16E-6}{}}{\SI{0.18E-6}{}} (0.7 - 0.5)^{2} \\
i _{D2} &= \boxed{\SI{96}{\micro \ampere}}
\end{align*}

It doesn't matter what the value of $R_2$ is because the $R_2$ only sets $V _{D2} = V _{DD} - i _{D2} R_2$.
The current is still determined by $V_D$ and $V_T$.

\item

Edge of saturation happens at pinchoff,

\begin{align*}
V _{D2} &= V _{GS} - V_T \\
V _{D2} &= 0.7 - 0.5 \\
V _{D2} &= \SI{0.2}{\volt}
\end{align*}

\begin{align*}
R_2 &= \frac{V _{DD} - V _{D2}}{i _{D2}} \\
R_2 &= \frac{1.8 - 0.2}{\SI{96E-6}{}} \\
R_2 &= \boxed{\SI{16.667}{\kilo \ohm}}
\end{align*}

To be more in saturation, we want $V _{D2} > V _{GS} - V_T$. This will increase the \SI{0.2}{\volt} term, decreasing
$R_2$. That means we want a smaller value of $R_2$.

\end{enumerate}

\pagebreak

\begin{problem}

\textbf{MIS Capacitors}

An MIS capacitor is composed of an \ch{SiO2} $(\text{relative permittivity} = 3.9)$ dielectric of thickness \SI{40}{\nano \meter} on top
of p-type silicon with doping level $N_A = \SI{10E17}{/ \centi \meter \cubed}$. Assume the threshold voltage is $\SI{0.5}{\volt}$ and a voltage
of $\SI{1.8}{\volt}$ is applied to the capacitor.

\begin{enumerate}[\alph*)]

\item

Calculate the induced surface charge density (charge per unit area) in the metal electrode. Are
these charges positive or negative, mobile or stationary?

\item

What must the induced charge density be at the semiconductor surface?

\item

How much of the induced charge density at the semiconductor surface is due to stationary
charges vs mobile charges?

\end{enumerate}

\end{problem}

\pagebreak

\begin{soln}
\end{soln}

\begin{enumerate}[\alph*)]

\item

\begin{align*}
\frac{Q}{A} &= \frac{\epsilon}{d} V \\
\frac{Q}{A} &= \frac{(3.9)(\SI{8.824E-12}{})}{\SI{40E-9}{}} (1.8) \times \frac{\SI{1}{\meter \squared}}{\SI{10000}{\centi \meter \squared}} \\
\frac{Q}{A} &= \boxed{\SI{1.549E-7}{\coulomb / \centi \meter \squared}}
\end{align*}

Positive mobile charges.

\item

\begin{align*}
\frac{Q}{A} &= \boxed{\SI{-1.549E-7}{\coulomb / \centi \meter \squared}}
\end{align*}

\item

Almost all of the induced charge density at the surface is due to stationary charges because
of inversion.

\end{enumerate}

\pagebreak

\begin{problem}

\textbf{An MOSFET}

A MOSFET is made with a dielectric composed of \ch{SiO2} $(\text{relative permittivity} = 3.9)$ and thickness $\SI{40}{\nano \meter}$. The
channel length is $\SI{180}{\nano \meter}$ and the channel width is $\SI{3}{\micro \meter}$. The applied gate voltage is $\SI{3}{\volt}$, the source voltage
is $\SI{0}{\volt}$, and the drain voltage is $\SI{0.5}{\volt}$. The threshold voltage is $\SI{0.5}{\volt}$. Assume the mobility of electrons to
be $\SI{1000}{\centi \meter \squared / \volt \second}$.

\begin{enumerate}[\alph*)]

\item

Calculate the drain current

\item

Calculate the equivalent resistance of the MOSFET assuming that its drain current rises linearly
with its voltage up to the operating point in the problem.

\item

All else equal, if the drain voltage is gradually increased, at what value of drain voltage will the
drain current saturate to a constant value? What will that constant value be?

\end{enumerate}

\end{problem}

\pagebreak

\begin{soln}
\end{soln}

\begin{enumerate}[\alph*)]

\item

\begin{align*}
i_D &= \mu_n C _{ox} \frac{W}{L}[(V _{GS} - V_T)V _{DS} - V _{DS}^{2}/2] \\
i_D &= (\SI{1000E-4}{}) \frac{(3.9) (\SI{8.854E-12}{})}{\SI{40E-9}{}} \frac{\SI{3E-6}{}}{\SI{180E-9}{}} [(3 - 0.5)(0.5) - (0.5)^{2}/2] \\
i_D &= \boxed{\SI{1.619}{\milli \ampere}} 
\end{align*}

\item

\begin{align*}
R _{eq} &= \frac{V_D}{i_D} \\
R _{eq} &= \frac{0.5}{\SI{1.619E-3}{}} \\
R _{eq} &= \boxed{\SI{308.833}{\ohm}}
\end{align*}

\item

\begin{align*}
V _{DS} &= V _{GS} - V_T \\
V _{DS} &= 3 - 0.5 \\
V _{DS} &= \boxed{\SI{2.5}{\volt}}
\end{align*}

\begin{align*}
I _{D, \text{sat}} &= \frac{1}{2} \mu_n C _{ox} \frac{W}{L} V _{DS}^{2} \\
I _{D, \text{sat}} &= \frac{1}{2} (\SI{1000E-4}{}) \frac{(3.9) (\SI{8.854E-12}{})}{\SI{40E-9}{}} \frac{\SI{3E-6}{}}{\SI{180E-9}{}} (2.5)^{2} \\
I _{D, \text{sat}} &= \boxed{\SI{4.496}{\milli \ampere}}
\end{align*}

\end{enumerate}

\pagebreak

\begin{problem}

\textbf{Small Signal Modeling}

Consider the so-called ``common-gate'' amplifier below. For all parts, assume the following:
\begin{align*}
&\mu_n C _{ox} = \SI{200}{\micro \ampere / \volt \squared} &\mu_p C _{ox} = \SI{100}{\micro \ampere / \volt \squared} \\
&V_T = \SI{0.4}{\volt} \quad \text{for NMOS devices} &V_T = \SI{-0.4}{\volt} \quad \text{for PMOS devices} \\
&W/L = 10
\end{align*}

\begin{figure}[H]
\centering
\includegraphics[width=0.3\textwidth]{5.png}
\end{figure}

\begin{enumerate}[\alph*)]

\item

An amplifier is made with an NMOS transistor as shown in the figure. $R = \SI{2}{\kilo \ohm}$ and $V _{DD} = \SI{2}{\volt}$.
The DC value of the gate voltage ($V_G$) is \SI{0}{\volt}. Calculate the value of the input, $V_S$, that will result
in the FET being at the edge of saturation.

\item

Calculate the small signal transconductance $g_m$ for the MOSFET in (a).

\item

$v_S(t)$ has a dc value $V_S$ calculated in (a) as well as small perturbation such that $v_S(t) = V_S + \tilde{v_S}$.
Draw the small signal model of the amplifier and calculate the small signal voltage gain $\tilde{v_O}/\tilde{v_S}$.
Let your answer be symbolic, i.e., a function of $g_m$ and $R$ rather than a number.

\item

At the output port, calculate the Thevenin equivalent of the small-signal model of the circuit as a
function of $g_m$ and $R$.

\item

At the input port (i.e., across $v_S$), calculate the Thevenin input impedance of the small signal
model of the circuit as a function of $g_m$ and $R$, assuming that the output is open circuited.

\end{enumerate}

\end{problem}

\pagebreak

\begin{soln}
\end{soln}

\begin{enumerate}[\alph*)]

\item

\begin{align*}
i_D &= \frac{1}{2} \mu_n C _{ox} \frac{W}{L} (V _{G} - V _{S} - V_T)^{2} \\
i_D &= \frac{1}{2} (\SI{200E-6}{}) (10) (0 - V _{S} - 0.4)^{2} \\
i_D &= \SI{E-3}{} (- V _{S} - 0.4)^{2} \\
\end{align*}

\begin{align*}
V _{DS} &= V _{GS} - V_T \\
V_D - V_S &= V_G - V_S - V_T \\
V_D &= V_G - V_T \\
V _{DD} - i_D R &= V _{G} - V_T \\
2 - \SI{E-3}{} (- V_S - 0.4)^{2} (2000) &= 0 - 0.4 \\
2 - 2(- V_S - 0.4)^{2} &= - 0.4 \\
2 - 2 V_S^2 - 1.6 V_S - 0.32 &= - 0.4 \\
- 2 V_S^2 - 1.6 V_S + 2.08 &= 0 \\
2 V_S^2 + 1.6 V_S - 2.08 &= 0
\end{align*}

\begin{align*}
V_S &= \frac{-1.6 \pm \sqrt{1.6^2 - 4(2)(-2.08)}}{2(2)} \\
V_S &= \SI{0.695}{\volt} \\
V_S &= \SI{-1.495}{\volt}
\end{align*}

Test both solutions for inversion,

\begin{align*}
V _{GS} &\geq V_T \\
V_G - V_S &\geq V_T \\
0 - V_S &\geq V_T \\
V_S &\leq - V_T
\end{align*}

\begin{align*}
0.695 &\nleq -0.4 \\
-1.495 &\leq -0.4
\end{align*}

\begin{align*}
\boxed{V_S = \SI{-1.495}{\volt}}
\end{align*}

\item

\begin{align*}
i_D &= \frac{1}{2} \mu_n C _{ox} \frac{W}{L} (V _{GS} - V_T)^{2} \\
g_m &= \frac{d i_D}{d V _{GS}} = \frac{d}{d V _{GS}} \frac{1}{2} \mu_n C _{ox} \frac{W}{L} (V _{GS} - V_T)^{2} \\
g_m &= \mu_n C _{ox} \frac{W}{L} (V _{GS} - V_T) \\
g_m &= \mu_n C _{ox} \frac{W}{L} (V_G - V_S - V_T) \\
g_m &= (\SI{200E-6}{}) (10) (0 - (-1.495) - 0.4) \\
g_m &= \boxed{\SI{2.190}{\milli \siemens}}
\end{align*}

\item

Small signal circuit,

\begin{figure}[H]
\centering
\includegraphics[width=0.4\textwidth]{5c1.jpg}
\end{figure}

\begin{align*}
i &= g_m \Delta v _{gs} \\
i &= g_m (0 - \Delta v_s) \\
\Delta v_s &= -\frac{i}{g_m}
\end{align*}

\begin{align*}
\Delta v_o &= - i R \\
\end{align*}

\begin{align*}
\boxed{\frac{\Delta v_o}{\Delta v_s} = g_m R}
\end{align*}

\item

\begin{align*}
\frac{\Delta v_o}{\Delta v_s} &= g_m R \\
\Delta v_o &= \boxed{V _{th} = g_m R \Delta v_s}
\end{align*}

\begin{align*}
\boxed{R _{th} = R}
\end{align*}

\item

\begin{align*}
\Delta i_s &= g_m \Delta v_s \\
\frac{\Delta v_s}{\Delta i_s} &= \frac{\Delta v_s}{g_m \Delta v_s} \\
\frac{\Delta v_s}{\Delta i_s} &= \boxed{\frac{1}{g_m}}
\end{align*}

\end{enumerate}

\end{document}
