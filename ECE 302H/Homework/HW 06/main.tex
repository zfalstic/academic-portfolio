\documentclass[11pt]{article}

\usepackage[utf8]{inputenc}
\usepackage[T1]{fontenc}
\usepackage{lmodern}
\usepackage{microtype}

\usepackage[sexy]{evan}

\usepackage{fancyhdr}
\pagestyle{fancy}
\fancyhf{}

\usepackage{amsmath, amssymb, amsthm, mathtools}
\usepackage{siunitx}
\usepackage{graphicx}
\usepackage{float}

\usepackage{pgfplots}
\pgfplotsset{compat=1.18}
\usepgfplotslibrary{statistics}

\newcommand{\coursename}{\textbf{Introduction to Electrical Engineering}}
\newcommand{\coursecode}{ECE 302H}
\newcommand{\term}{Fall 2025}
\newcommand{\instructor}{Dr.\ Hanson}
\newcommand{\notetaker}{Dawson Zhang}
\newcommand{\lecturetitle}{Homework 6}

\fancyhead[L]{\coursecode}
\fancyhead[C]{\lecturetitle}
\fancyhead[R]{\term}
\fancyfoot[R]{\thepage}

\title{\coursename~(\coursecode) -- \lecturetitle}
\author{\notetaker~~|~~Instructor: \instructor}
\date{\term}

\begin{document}
  
\maketitle
\pagebreak

\setcounter{section}{6}

\begin{problem}

The Mass Action Law and Massively Large Numbers

There is a very important relationship in semiconductors that relates the number of free electrons and
the number of holes known as the Mass Action Law, which you will derive below. In addition, we deal
with numbers between $10 ^{10}$ and $10 ^{22}$ when talking about electron and atom concentrations in
semiconductors. These numbers are pretty wild to think about and can become fairly abstract. This
problem will help you think about these numbers.

\begin{enumerate}[\alph*)]

\item

In semiconductors, mobile electrons and holes are constantly being generated. They are also
constantly undergoing ``recombination,'' meaning that an electron fills a hole and both
``particles'' become immobilized. Let the rate of generation be $G$ (measured in $\text{\#}/cm^3$ per
second) and let the rate of recombination be $R$ (with the same units). In equilibrium, what must
the relationship between $R$ and $G$ be? Explain why.

\item

The rate of generation $G$ is a function of temperature, $G(T)$. Explain why this makes sense.

\item

The rate of recombination is proportional to the product of the mobile electron concentration $n$
and the mobile hole concentration $p$. Explain why this makes sense.

\item

By the above, the product $np$ must be equal to a quantity that only depends on temperature.
This quantity is called $n^2_i$ where $n_i$ is called the ``intrinsic concentration'' and, at $\SI{300}{\kelvin}$, is
approximately equal to $10 ^{10}/cm^3$. This yields the \textbf{mass-action law} $np = n^2_i(T)$ What is $n^2_i(T)$ 
at $\SI{300}{\kelvin}$, and what will $n$ and $p$ be for a piece of pure silicon?

\item

The silicon lattice cell is a cube with $8$ silicon atoms at the corners, $6$ silicon atoms on the ``faces''
of the cube, and $4$ silicon atoms wholly enclosed in the cube. The side length is $5.43$ angstroms.
What is the concentration of silicon atoms $N_s$ in crystalline silicon? (Hint – atoms on the edges
and faces are shared between multiple lattice cells) To appreciate the relative magnitudes of $N_s$
and $n_i$, write them out long-form (not in scientific notation).

\item

If you had a crystal with one silicon atom for every person on earth, how many mobile electrons
would there be, rounded to the nearest whole number? (Current population of earth is about $8$
billion people)

\item

Suppose that a piece of silicon is doped with a concentration of donors $N_d$. There are now three
kinds of uncompensated charges in the silicon – free electrons, free holes, and stationary donor
atoms. At equilibrium, the material must be electrically neutral. Write an equation representing
this statement using the three mentioned concentrations.

\item

Using the mass action law, calculate the concentration of electrons $n$ as a function of $n_i$ and $N_d$.
Show that, for $N_d \gg n_i$, $n \approx N_d$. What would $p$ be in this approximation?

\item

The maximum dopant concentration in silicon without substantially changing its crystal
structure and electronic properties is about $10 ^{19}/cm^3$. If you had a crystal with one silicon
atom for every person on earth, doped at the maximum rate given, how many free electrons
would there be? (Current population of earth is about $8$ billion people) This is approximately the
size of what US metropolitan area? (visit the Wikipedia page for Metropolitan Statistical Area for
a table)

\end{enumerate}

\end{problem}

\pagebreak

\begin{soln}
\end{soln}

\begin{enumerate}[\alph*)]

\item

In equilibrium, $R = G$ because the rate of generation and rate of recombination would
have to be equal.

\item

The rate of generation $G$ is a function of temperature $G(T)$ bceause the temperature
is the property that vibrates the electrons and allow them to move into the lattice.

\item

The rate of recombination being proportional to the product of the mobile electron concentration
$n$ and the mobile hole concentration $p$ makes sense because an increase in either
will lead to an increase in the rate of recombination.

\item

\begin{align*}
n_i(300) &= 10 ^{10} / \SI{}{\centi \meter}^{3} \\
n_i^2(300) &= 10 ^{10 ^{2}} / \SI{}{\centi \meter}^{3^2} \\
n_i^2(300) &= \boxed{10 ^{20} / \SI{}{\centi \meter}^{6}}
\end{align*}

In pure silicon $n = p$.

\begin{align*}
n &= p \\
np &= n_i^2(T) \\
np _{300} &= n_i^2(300) \\
np _{300} &= 10 ^{20} / \SI{}{\centi \meter}^{6} \\
n &= \boxed{10 ^{10} / \SI{}{\centi \meter}^{3}} \\
p &= \boxed{10 ^{10} / \SI{}{\centi \meter}^{3}}
\end{align*}

\item

The main factor that we need to consider for this problem are the shared edges and faces between
distinct silicon lattice cells within the overall structure. We know from the strcuture of
stacking cubes that each corner is shared by 8 cells while each face is shared by 2 cells. That
means that on average, each cell only constitutes $8/8$ silicon atoms at the corner, and $6/2$
atoms for each face.

\begin{align*}
Si _{\text{atoms per cell}} &= \frac{8}{8} + \frac{6}{2} + 4 \\
Si _{\text{atoms per cell}} &= 8
\end{align*}

An angstrom is $10 ^{-8} \SI{}{\centi \meter}$. That means that the concentration of silicon 
atoms $N_s$ in $\text{\#} / \SI{}{\centi \meter}^{2}$ is,

\begin{align*}
N_s &= \frac{8 Si}{5.43^3 \text{angstrom}^{3}} \times \frac{1 \text{angstrom}^{3}}{10 ^{-8^3} \SI{}{\centi \meter}^{3}} \\
N_s &= 5.0 \times 10 ^{22} Si / \SI{}{\centi \meter}^{3} \\
N_s &= \boxed{50,000,000,000,000,000,000,000 \quad Si / \SI{}{\centi \meter}^{3}}
\end{align*}

\item

\begin{align*}
n _{\text{mobile electrons}} &= \frac{8 \times 10^9 Si}{1} \times \frac{1 \SI{}{\centi \meter}^{3}}{5.0 \times 10 ^{22} Si} \times \frac{10 ^{10}}{\SI{}{\centi \meter}^{3}} \\
n _{\text{mobile electrons}} &= 4.828 \times 10 ^{-5} \\
n _{\text{mobile electrons}} &= \boxed{1.6 \times 10 ^{-3}}
\end{align*}

\begin{remark*}
With how small this number is and the fact that you can't have $1.6 \times 10 ^{-3}$ of an electron,
the effective number of mobile electrons is zero.
\end{remark*}

\item

\begin{align*}
\boxed{\text{free holes} + \text{donor atoms} - \text{free electrons} = 0}
\end{align*}

\begin{align*}
\boxed{p + N_d - n = 0}
\end{align*}

\item

\begin{align*}
np &= n_i^2 \\
p &= n - N_d \\
n(n - N_d) &= n_i^2 \\
n^2 - nN_d - n_i^2 &= 0 \\
\end{align*}

Quadratic formula and solve for positive $n$ solution,

\begin{align*}
n &= \frac{N_d + \sqrt{N_d^2 + 4n_i^2}}{2} \\
\end{align*}

Notice that because $N_d \gg n_i$, $N_d^2 + 4n_i^2 = N_d^2$,

\begin{align*}
n &\approx \frac{N_d + \sqrt{N_d^2}}{2} \\
n &\approx \frac{2N_d}{2} \\
n &\approx N_d
\end{align*}

Considering what $p$ would be,

\begin{align*}
np &= n_i^2 \\
p &= \frac{n_i^2}{n} \\
p &\approx \frac{n_i^2}{N_d}
\end{align*}

\item

\begin{align*}
n _{\text{mobile electrons}} &= \frac{8 \times 10^9 Si}{1} \times \frac{1 \SI{}{\centi \meter}^{3}}{5.0 \times 10 ^{22} Si} \times \frac{10 ^{19}}{\SI{}{\centi \meter}^{3}} \\
n _{\text{mobile electrons}} &= 1.6 \times 10^6
\end{align*}

By the 2024 census, Milwaukee–Waukesha, WI comes in the closes with $1.574 \times 10^6$ population.

\end{enumerate}

\pagebreak

\begin{problem}

An Op Amp Current Source

\begin{enumerate}[\alph*)]

\item

Let the load be a voltage source of value $v_L$. Use node analysis to calculate the voltage at the
negative op amp input $v_-$ , the voltage at the positive op amp input $v_+$, the voltage at the
output of the op amp $v_o$, and the load current $i_o$. Assume an ideal op amp with infinite gain.

\item

In order for the op-amp circuit to act like a current source with respect to the load, what must
its output resistance be? What relationship between the resistors makes this circuit behave as a
current source?

\item

When (b) is satisfied, what is the value of the current source?

\end{enumerate}

\begin{figure}[H]
\centering
\includegraphics[width=0.5\textwidth]{2.png}
\end{figure}

\end{problem}

\pagebreak

\begin{soln}
\end{soln}

\begin{enumerate}[\alph*)]

\item

Circuit equivalents,

\begin{align*}
v_L = v_+ = v_-
\end{align*}

Node analysis at $v_-$,

\begin{align*}
\frac{v_o - v_L}{R_4} - \frac{v_L}{R_3} &= 0 \\
R_3 v_o - R_3 v_L &= R_4 v_L \\
R_3 v_o &= (R_3 + R_4) v_L \\
v_o &= \frac{R_3 + R_4}{R_3} v_L \\
\end{align*}

Node analysis at $v_+$,

\begin{align*}
\frac{v_I - v_L}{R_1} - i_o - \frac{v_L - v_o}{R_2} &= 0 \\
R_2 v_I - R_2 v_L - R_1 R_2 i_o - R_1 v_L + R_1 v_o &= 0 \\
R_2 v_I - R_2 v_L - R_1 R_2 i_o - R_1 v_L + \frac{R_1 R_3 + R_1 R_4}{R_3} v_L &= 0 \\
\frac{R_1 R_3 + R_1 R_4 - R_1 R_3 - R_2 R_3}{R_2} v_L + R_2 v_I &= R_1 R_2 i_o \\
\frac{R_1 R_4 - R_2 R_3}{R_1 R_2^2} v_L + \frac{1}{R_1} v_I &= i_o
\end{align*}

\begin{align*}
v_- &= v_L \\
v_+ &= v_L \\
v_o &= \frac{R_3 + R_4}{R_3} v_L \\
i_o &= \boxed{\frac{R_1 R_4 - R_2 R_3}{R_1 R_2^2} v_L + \frac{1}{R_1} v_I}
\end{align*}

\item

For the circuit to act like a current source with respsect to the load, we want to set the $v_L$ term
to $0$.

\begin{align*}
\frac{R_1 R_4 - R_2 R_3}{R_1 R_2^2} &= 0 \\
R_1 R_4 - R_2 R_3 &= 0 \\
\boxed{R_1 R_4 = R_2 R_3}
\end{align*}

\item

\begin{align*}
\boxed{i_o = \frac{1}{R_1} v_I}
\end{align*}

\end{enumerate}

\pagebreak

\begin{problem}

Sheet resistance

In integrated circuits, it is common to use thin sheets of resistive material to build resistors. Current
flows laterally through the sheet. In this case, a common metric for the resistor deposition process is
called the ``sheet resistance'' $R_S$.

\begin{enumerate}[\alph*)]

\item

Sheet resistance is expressed in units of ``ohms per square,'' i.e., the amount of resistance a
square sheet would have. The size of the square is not specified. Explain why it is intellectually
defensible to do this based on how resistance relates to geometry.

\item

Calculate the sheet resistance for a film of thickness $t$ and resistivity $\rho$.

\item

In terms of $R_S$, what is the resistance of a film that is twice as long as it is wide? Twice as wide
as it is long?

\end{enumerate}

\end{problem}

\pagebreak

\begin{soln}
\end{soln}

\begin{enumerate}[\alph*)]

\item

We know that $R = \rho \frac{l}{A}$ for a material of length $l$ and perpendicular cross sectional area $A$.
Because current flows laterally through the sheet, the length of the sheet $l$ stays the same. The remaining
side of the sheet, lets say $W$ makes up part of the area $A$. Let's call the other part of $A$ a the thickness $t$.

\begin{align*}
R_S &= \rho \frac{l}{Wt}
\end{align*}

In a square $l = W$,

\begin{align*}
R_S &= \rho \frac{l}{Wt} \\
R_S &= \frac{\rho}{t}
\end{align*}

As we can see, $R_S$ is independent of the size of the square. So the measurement ``ohms per square'' makes sense.

\item

From the previous part,

\begin{align*}
R_S &= \boxed{\frac{\rho}{t}}
\end{align*}

\item

\begin{align*}
R &= \rho \frac{l}{Wt} \\
R &= R_S \frac{l}{W} \\
\end{align*}

Twice as long as it is wide,

\begin{align*}
R &= R_S \frac{l}{W} \\
R &= R_S \frac{2W}{W} \\
R &= \boxed{2 R_S}
\end{align*}

Twice as wide as it is long,

\begin{align*}
R &= R_S \frac{l}{W} \\
R &= R_S \frac{l}{2l} \\
R &= \boxed{\frac{1}{2} R_S}
\end{align*}

\end{enumerate}

\pagebreak

\begin{problem}

Eletrical speed

We envision electricity as a fast process, but how fast do electrons actually move?

\begin{enumerate}[\alph*)]


\item

The conductivity of copper is about $5.9 \times 10^7 S/m$, due almost entirely to mobile electrons.
There is about one free electron per atom, of which there are about $8.5 \times 10 ^{28}$ per cubic meter.
The maximum current density that one would normally allow in a copper wire is about $500 A/cm^2$. 
What is the maximum electron velocity in copper under these conditions?

\item

Given your answer, explain how it is possible that a light bulb turns on almost instantly even
when it is several meters away from the switch.

\end{enumerate}

\end{problem}

\pagebreak

\begin{soln}
\end{soln}

\begin{enumerate}[\alph*)]

\item

\begin{align*}
V &= \frac{\SI{500}{\coulomb}}{\SI{}{\second} \times \SI{}{\centi \meter}^{2}} \times \frac{\SI{}{\meter}^{3}}{8.5 \times 10 ^{28} e} \times \frac{6.24 \times 10 ^{18} e}{\SI{1}{\coulomb}} \times \frac{100^2 \SI{}{\centi \meter}^{2}}{1 \SI{}{\meter}^2} \\
V &= \boxed{3.671 \times 10 ^{-4} \SI{}{\meter} / \SI{}{\second}}
\end{align*}

\item

Although the velocity is very small, the light bulb turns on near instantly because there already electrons
throughout the entire volume of wire. That means that there are already electrons right next to the light bulb,
and when the switch is flicked on, those electrons will flow into the bulb.

\end{enumerate}

\pagebreak

\begin{problem}

Skin effect

At high frequencies, current does not flow over the entire cross-sectional area of a conductor, but rather
only in a surface layer of thickness $\delta$, where $\delta$ is known as the skin depth.

\begin{enumerate}[\alph*)]

\item

Calculate the ordinary resistance of a cylindrical conductor of radius $b$, length $l$, and conductivity
$\sigma$.

\item

Calculate the effective resistance of the cylindrical conductor when operated at high frequency
such that the skin depth is $\delta$.

\item

Imagine an AWG 22 copper wire ($\sigma = 5.96 \times 10^7 S/m$), such as the ones we use in the lab. The
skin depth is given by $\delta = \sqrt{2 \rho / \omega \mu_0}$ where $\omega = 2 \pi f$ is the angular frequency of the current
and $\mu_0 = 4 \pi \times 10 ^{-7} H/m$ is a constant of the universe known as the permeability of free
space. At what frequency $f$ is the skin depth equal to the wire radius? Below this frequency,
current is well modeled as flowing evenly over the cross-sectional area of the conductor, as
we’re used to doing and as we calculated in part (a). Above this frequency, the conductor is
''skin depth limited'' and one must use the result from part (b).

\item

For the wire in part (c), what is the ratio of effective resistance to ordinary resistance at $\SI{100}{\kilo \hertz}$,
$\SI{1}{\mega \hertz}$, and $\SI{10}{\mega \hertz}$?

\end{enumerate}

\end{problem}

\pagebreak

\begin{soln}
\end{soln}

\begin{enumerate}[\alph*)]

\item

\begin{align*}
R &= \rho \frac{l}{A _{\text{cross section}}} \\
\rho &= \frac{1}{\sigma} \\
R &= \frac{l}{\sigma \pi b^2}
\end{align*}

\item

\begin{align*}
A &= \pi b^2 - \pi (b - \delta)^{2} \\
A &= \pi b^2 - \pi b^2 + 2 \pi b \delta - \pi \delta^2 \\
A &= 2 \pi b \delta - \pi \delta^2
\end{align*}

\begin{align*}
R &= \frac{l}{\sigma A} \\
R &= \frac{l}{\sigma 2 \pi b \delta - \sigma \pi \delta^2}
\end{align*}

\item

\begin{align*}
\delta &= \sqrt{\frac{2 \rho}{\omega \mu_0}} \\
\delta &= \sqrt{\frac{2 \rho}{2 \pi f \mu_0}} \\
\delta &= \sqrt{\frac{\rho}{\pi f \mu_0}} \\
\delta &= \sqrt{\frac{1}{\sigma \pi f \mu_0}} \\
b &= \sqrt{\frac{1}{\sigma \pi f \mu_0}} \\
b^2 &= \frac{1}{\sigma \pi f \mu_0} \\
f &= \frac{1}{\sigma \pi b^2 \mu_0} \\
f &= \frac{1}{(5.96 \times 10^7)(\pi)(0.322 \times 10 ^{-3})^{2}(4 \pi \times 10 ^{-7})} \\
f &= \SI{40.990}{\kilo \hertz}
\end{align*}

\item

First calculate the ordinary resistance of an AWG 22 wire ($\SI{0.322}{\milli \meter}$ radius),

\begin{remark*}
Because we are taking a ratio of resistances, I will treat each length $l$ as $\SI{1}{\meter}$,
because it a constant variable.
\end{remark*}

\begin{align*}
R_0 &= \frac{l}{\sigma \pi b^2} \\
R_0 &= \frac{1}{(5.96 \times 10^7)(\pi)(0.322 \times 10 ^{-3})^{2}} \\
R_0 &= \SI{51.510}{\milli \ohm}
\end{align*}

At $\SI{100}{\kilo \hertz}$,

\begin{align*}
\delta &= \sqrt{\frac{1}{\sigma \pi f \mu_0}} \\
\delta &= \sqrt{\frac{1}{(5.96 \times 10^7)(\pi)(100 \times 10^3)(4\pi \times 10 ^{-7})}} \\
\delta &= 2.062 \times 10 ^{-4}
\end{align*}

\begin{align*}
R &= \frac{1}{\sigma 2 \pi b \delta - \sigma \pi \delta^2} \\
R &= \frac{1}{\sigma \pi \delta (2 b - \delta)} \\
R &= \frac{1}{(5.96 \times 10^7)(\pi)(2.062 \times 10 ^{-4})(2(0.322 \times 10 ^{-3}) - 2.062 \times 10 ^{-4})} \\
R &= \SI{59.162}{\milli \ohm}
\end{align*}

\begin{align*}
\frac{R}{R_0} &= 1.149
\end{align*}

At $\SI{1}{\mega \hertz}$,

\begin{align*}
\delta &= \sqrt{\frac{1}{\sigma \pi f \mu_0}} \\
\delta &= \sqrt{\frac{1}{(5.96 \times 10^7)(\pi)(1 \times 10^6)(4\pi \times 10 ^{-7})}} \\
\delta &= 6.519 \times 10 ^{-5}
\end{align*}

\begin{align*}
R &= \frac{1}{\sigma \pi \delta (2 b - \delta)} \\
R &= \frac{1}{(5.96 \times 10^7)(\pi)(6.519 \times 10 ^{-5})(2(0.322 \times 10 ^{-3}) - 6.519 \times 10 ^{-5})} \\
R &= \SI{141.543}{\milli \ohm}
\end{align*}

\begin{align*}
\frac{R}{R_0} &= 2.748
\end{align*}

At $\SI{10}{\mega \hertz}$,

\begin{align*}
\delta &= \sqrt{\frac{1}{\sigma \pi f \mu_0}} \\
\delta &= \sqrt{\frac{1}{(5.96 \times 10^7)(\pi)(10 \times 10^6)(4\pi \times 10 ^{-7})}} \\
\delta &= 2.062 \times 10 ^{-5}
\end{align*}

\begin{align*}
R &= \frac{1}{\sigma \pi \delta (2 b - \delta)} \\
R &= \frac{1}{(5.96 \times 10^7)(\pi)(2.062 \times 10 ^{-5})(2(0.322 \times 10 ^{-3}) - 2.062 \times 10 ^{-5})} \\
R &= \SI{415.492}{\milli \ohm}
\end{align*}

\begin{align*}
\frac{R}{R_0} &= 8.066
\end{align*}

\end{enumerate}

\pagebreak

\begin{problem}

Semiconductor Exercises

\begin{enumerate}[\alph*)]

\item

For silicon doped with acceptors (like Boron) at a rate of $N_A = 2 \times 10 ^{18} / \SI{}{\centi \meter}^3$, find the hole and
electron concentrations.

\item

For silicon doped with donors (like Phosphorus) at a rate of $N_d$ ($\text{\#}/ \SI{}{\centi \meter}^3$), what must $N_d$ be if
the hole concentration is below the intrinsic level by a factor of $10^8$?

\item

Find the end-to-end resistance of a bar that is $\SI{15}{\micro \meter}$ long, $\SI{3}{\micro \meter}$ wide, and $\SI{1}{\micro \meter}$ thick, made of
the following materials:

\begin{enumerate}[\roman*.]

\item

Intrinsic silicon

\item

Silicon doped with donors (like Phosphorus) with $N_d = 5 \times 10 ^{16}/ \SI{}{\centi \meter}^3$

\item

Silicon doped with donors (like Phosphorus) with $N_d = 5 \times 10 ^{18}/ \SI{}{\centi \meter}^3$

\item

Silicon doped with acceptors (like Boron) with $N_a = 5 \times 10 ^{16}/ \SI{}{\centi \meter}^3$

\item

Aluminum with resistivity $\SI{2.8}{\micro \ohm} \cdot \SI{}{\centi \meter}$

\end{enumerate}

Assume that the mobility of electrons in silicon is $\mu_n = 1200 \SI{}{\centi \meter}^2 / V s$ and the mobility of holes
is $\mu_p = \mu_n / 3 = 400 \SI{}{\centi \meter}^2 / V s$. Be sure to take into account the contributions of both electrons
and holes.

\end{enumerate}

\end{problem}

\pagebreak

\begin{soln}
\end{soln}

\begin{enumerate}[\alph*)]

\item

Because the silicon is doped, we can ignore the concentration of electrons and holes as a result of thermal generation,

\begin{align*}
P &\approx \boxed{2 \times 10 ^{18} / \SI{}{\centi \meter}^{3}} \\
n \times P &= 10 ^{20} / \SI{}{\centi \meter}^{6} \\
n &\approx \frac{10 ^{20} / \SI{}{\centi \meter}^{6}}{10 ^{18} / \SI{}{\centi \meter}^{3}} \\
n &\approx \boxed{50 / \SI{}{\centi \meter}^{3}}
\end{align*}

\item

\begin{align*}
P &= \frac{10 ^{10} / \SI{}{\centi \meter}^{3}}{10^8} \\
P &= \SI{100}{/ \centi \meter \cubed}\\
n \times P &= 10 ^{20} / \SI{}{\centi \meter}^{6} \\
n &= \frac{10 ^{20} / \SI{}{\centi \meter}^{6}}{100 / \SI{}{\centi \meter}^{3}} \\
n &= \boxed{\SI{E18}{/ \centi \meter \cubed}}
\end{align*}

\item

\begin{enumerate}[\roman*.]

\item

We want $\sigma$ in terms of $\SI{}{\siemens} / \SI{}{\meter}$. First convert mobility and concentration
to be in terms of $\SI{}{\meter}$,

\begin{align*}
n_i &= \frac{10 ^{10}}{\SI{}{\centi \meter}^{3}} \times \frac{100^3 \SI{}{\centi \meter}^{3}}{\SI{}{\meter}^{3}} \\
n_i &= \SI{E16}{\per \meter \cubed}
\end{align*}

\begin{align*}
\mu_n &= \frac{1200 \SI{}{\centi \meter}^{2}}{\SI{}{\volt \second}} \times \frac{\SI{}{\meter}^{2}}{100^2 \SI{}{\centi \meter}^{2}} \\
\mu_n &= \SI{0.12}{\meter \squared \per \volt \per \second}
\end{align*}

\begin{align*}
\mu_h &= \SI{0.04}{\meter \squared \per \volt \per \second}
\end{align*}

\begin{align*}
\sigma &= q (n \mu_n + p \mu_p) \\
\sigma &= 1.602 \times 10 ^{-19} ( 10 ^{16} (0.12) + 10 ^{16} (0.04) ) \\
\sigma &= \SI{2.563E-4}{\siemens \per \meter}
\end{align*}

\begin{align*}
R &= \frac{1}{\sigma} \times \frac{l}{A} \\
R &= \frac{1}{2.563 \times 10 ^{-4}} \times \frac{15 \times 10 ^{-6}}{(3 \times 10 ^{-6})(10 ^{-6})} \\
R &= \boxed{\SI{19.508}{\giga \ohm}}
\end{align*}

\item

\begin{align*}
N_d = n &= \frac{5 \times 10 ^{16}}{\SI{}{\centi \meter \cubed}} \times \frac{100^3 \SI{}{\centi \meter \cubed}}{\SI{1}{\meter \cubed}} \\
n &= \SI{5E22}{\per \meter \cubed}
\end{align*}

\begin{align*}
np &= \SI{E20}{\per \centi \meter \tothe{6}} \\
p &= \frac{\SI{E20}{\per \centi \meter \tothe{6}}}{\SI{5E16}{\per \centi \meter \cubed}} \times \frac{100^3 \SI{}{\centi \meter \cubed}}{\SI{1}{\meter \cubed}} \\
p &= \SI{2E9}{\per \meter \cubed}
\end{align*}

\begin{align*}
\sigma &= q (n \mu_n + p \mu_p) \\
\sigma &= 1.602 \times 10 ^{-19} ( (5 \times 10 ^{22}) (0.12) + (2 \times 10^9) (0.04) ) \\
\sigma &= \SI{961.2}{\siemens \per \meter}
\end{align*}

\begin{align*}
R &= \frac{1}{\sigma} \times \frac{l}{A} \\
R &= \frac{1}{961.2} \times \frac{15 \times 10 ^{-6}}{(3 \times 10 ^{-6})(10 ^{-6})} \\
R &= \boxed{\SI{5.201}{\kilo \ohm}}
\end{align*}

\item

\begin{align*}
N_d = n &= \frac{5 \times 10 ^{18}}{\SI{}{\centi \meter \cubed}} \times \frac{100^3 \SI{}{\centi \meter \cubed}}{\SI{1}{\meter \cubed}} \\
n &= \SI{5E24}{\per \meter \cubed}
\end{align*}

\begin{align*}
np &= \SI{E20}{\per \centi \meter \tothe{6}} \\
p &= \frac{\SI{E20}{\per \centi \meter \tothe{6}}}{\SI{5E18}{\per \centi \meter \cubed}} \times \frac{100^3 \SI{}{\centi \meter \cubed}}{\SI{1}{\meter \cubed}} \\
p &= \SI{2E7}{\per \meter \cubed}
\end{align*}

\begin{align*}
\sigma &= q (n \mu_n + p \mu_p) \\
\sigma &= 1.602 \times 10 ^{-19} ( (5 \times 10 ^{24}) (0.12) + (2 \times 10^7) (0.04) ) \\
\sigma &= \SI{9.612E4}{\siemens \per \meter}
\end{align*}

\begin{align*}
R &= \frac{1}{\sigma} \times \frac{l}{A} \\
R &= \frac{1}{9.612 \times 10^4} \times \frac{15 \times 10 ^{-6}}{(3 \times 10 ^{-6})(10 ^{-6})} \\
R &= \boxed{\SI{52.018}{\ohm}}
\end{align*}

\item

\begin{align*}
N_a = p &= \SI{5E16}{\per \centi \meter \cubed} \\
p &= \frac{\SI{5E16}{}}{\SI{}{\centi \meter \cubed}} \times \frac{100^3 \SI{}{\centi \meter \cubed}}{\SI{1}{\meter \cubed}} \\
p &= \SI{5E22}{\per \meter \cubed}
\end{align*}

\begin{align*}
np &= \SI{E20}{\per \centi \meter \tothe{6}} \\
n &= \frac{\SI{E20}{\per \centi \meter \tothe{6}}}{\SI{5E16}{\per \centi \meter \cubed}} \times \frac{100^3 \SI{}{\centi \meter \cubed}}{\SI{1}{\meter \cubed}} \\
n &= \SI{2E9}{\per \meter \cubed}
\end{align*}

\begin{align*}
\sigma &= q (n \mu_n + p \mu_p) \\
\sigma &= 1.602 \times 10 ^{-19} ( (2 \times 10^9) (0.12) + (5 \times 10 ^{22}) (0.04) ) \\
\sigma &= \SI{320.4}{\siemens \per \meter}
\end{align*}

\begin{align*}
R &= \frac{1}{\sigma} \times \frac{l}{A} \\
R &= \frac{1}{320.4} \times \frac{15 \times 10 ^{-6}}{(3 \times 10 ^{-6})(10 ^{-6})} \\
R &= \boxed{\SI{15.605}{\kilo \ohm}}
\end{align*}

\item

\begin{align*}
\rho &= \SI{2.8}{\micro \ohm \centi \meter} \\
\rho &= \SI{2.8}{\micro \ohm \centi \meter} \times \frac{\SI{1}{\ohm}}{\SI{E6}{\micro \ohm}} \times \frac{\SI{1}{\meter}}{\SI{100}{\centi \meter}} \\
\rho &= \SI{2.8E-8}{\ohm \meter}
\end{align*}

\begin{align*}
R &= \rho \times \frac{l}{A} \\
R &= 2.8 \times 10 ^{-8} \times \frac{15 \times 10 ^{-6}}{(3 \times 10 ^{-6})(10 ^{-6})} \\
R &= \boxed{\SI{0.14}{\ohm}}
\end{align*}

\end{enumerate}

\end{enumerate}

\pagebreak

\begin{problem}

Parallel plate capacitor

A parallel plate capacitor has a dielectric of relative permittivity $3.9$ and a thickness of $\SI{30}{\nano \meter}$.

\begin{enumerate}[\alph*)]

\item

Find the capacitance per unit area

\item

If a voltage of $\SI{2}{\volt}$ is applied and the dimensions of the rectangular plates are $\SI{180}{\nano \meter} \times \SI{2}{\micro \meter}$,
calculate the capacitance.

\end{enumerate}

\end{problem}

\pagebreak

\begin{soln}
\end{soln}

\begin{enumerate}[\alph*)]

\item

\begin{align*}
C &= \frac{\epsilon A}{d} \\
\frac{C}{A} &= \frac{\epsilon_r \epsilon_0}{d} \\
\frac{C}{A} &= \frac{3.9 \times \SI{8.854E-12}{}}{\SI{30E-9}{}} \\
\frac{C}{A} &= \boxed{\SI{1.151E-3}{\farad / \meter \squared}}
\end{align*}

\item

\begin{align*}
C &= \SI{180E-9}{\meter} \times \SI{2E-6}{\meter} \times \SI{1.151E-3}{\farad / \meter \squared} \\
C &= \boxed{\SI{4.144E-16}{\farad}}
\end{align*}

\end{enumerate}

\pagebreak

\begin{problem}

Capacitor with two dielectrics

A parallel plate capacitor has a dielectric that is composed of two layers. The first layer is $\SI{30}{\nano \meter}$ thick and
has a relative permittivity of $3.9$, while the second layer has a thickness of $\SI{50}{\nano \meter}$ and a relative permittivity
of $18$.

\begin{enumerate}[\alph*)]

\item

Draw a box on which to apply Gauss’ law. The box’s top should be embedded in the metal of the
top electrode; the box’s bottom should be embedded in the first dielectric layer. Use Gauss’ law
to calculate the electric field $E_1$ in the first dielectric as a function of the charge per unit area on
the top plate.

\item

Draw another box on which to apply Gauss’ law, but this time let the bottom of the box be
embedded in the second dielectric layer. Use Gauss’ law to calculate the electric field $E_2$ in the
second dielectric as a function of the charge per unit area on the top plate.

\item

Which dielectric will have the larger electric field?

\item

Integrate $\int E \cdot dl$ from the top electrode to the bottom electrode to calculate the voltage across
the capacitor as a function of $E_1$ and $E_2$.

\item

Find the capacitance per unit area.

\end{enumerate}

\end{problem}

\pagebreak

\begin{soln}
\end{soln}

\begin{enumerate}[\alph*)]

\item

\begin{figure}[H]
\centering
\includegraphics[width=0.5\textwidth]{6a.jpg}
\end{figure}

\begin{align*}
\oint E \epsilon \cdot dA &= Q _{\text{enclosed}} \\
E_1 \epsilon_1 \epsilon_0 A &= Q _{\text{enclosed}} \\
E_1 &= \frac{1}{\epsilon_1 \epsilon_0} \times \frac{Q _{\text{enclosed}}}{A} \\
E_1 &= \boxed{\frac{1}{3.9 \epsilon_0} \times \frac{Q _{\text{enclosed}}}{A}}
\end{align*}

\item

\begin{align*}
\oint E \epsilon \cdot dA &= Q _{\text{enclosed}} \\
E_2 \epsilon_2 \epsilon_0 A &= Q _{\text{enclosed}} \\
E_2 &= \frac{1}{\epsilon_2 \epsilon_0} \times \frac{Q _{\text{enclosed}}}{A} \\
E_2 &= \boxed{\frac{1}{18 \epsilon_0} \times \frac{Q _{\text{enclosed}}}{A}}
\end{align*}

\item

First dielectric will have a larger field because the relative permittivity is lower,

\begin{align*}
E \propto \frac{1}{\epsilon}
\end{align*}

\item

\begin{align*}
V &= \boxed{E_1 \times \SI{30}{\nano \meter} + E_2 \times \SI{50}{\nano \meter}}
\end{align*}

\item

\begin{align*}
Q &= CV \\
C &= \frac{Q}{V} \\
C &= \frac{Q}{E_1 \times \SI{30}{\nano \meter} + E_2 \times \SI{50}{\nano \meter}} \\
C &= \frac{Q}{\frac{1}{3.9 \epsilon_0} \times \frac{Q}{A} \times \SI{30E-9}{} + \frac{1}{18 \epsilon_0} \times \frac{Q}{A} \times \SI{50E-9}{}} \\
C &= \frac{Q}{\frac{Q}{A} \left( \frac{\SI{30E-9}{}}{3.9 \times \SI{8.854E-12}{}} + \frac{\SI{50E-9}{}}{18 \times \SI{8.854E-12}{}} \right)} \\
\frac{C}{A} &= \frac{1}{\frac{\SI{30E-9}{}}{3.9 \times \SI{8.854E-12}{}} + \frac{\SI{50E-9}{}}{18 \times \SI{8.854E-12}{}}} \\
\frac{C}{A} &= \boxed{\SI{8.456E-4}{\farad / \meter \squared}}
\end{align*}

\end{enumerate}

\end{document}
