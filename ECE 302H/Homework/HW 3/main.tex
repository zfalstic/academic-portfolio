\documentclass[11pt]{article}

\usepackage[utf8]{inputenc}
\usepackage[T1]{fontenc}
\usepackage{lmodern}
\usepackage{microtype}

\usepackage[sexy]{evan}

\usepackage{fancyhdr}
\pagestyle{fancy}
\fancyhf{}

\usepackage{amsmath, amssymb, amsthm, mathtools}
\usepackage{siunitx}
\usepackage{graphicx}
\usepackage{float}

\usepackage{pgfplots}
\pgfplotsset{compat=1.18}
\usepgfplotslibrary{statistics}

\newcommand{\coursename}{\textbf{Introduction to Electrical Engineering}}
\newcommand{\coursecode}{ECE 302H}
\newcommand{\term}{Fall 2025}
\newcommand{\instructor}{Dr.\ Hanson}
\newcommand{\notetaker}{Dawson Zhang}
\newcommand{\lecturetitle}{Homework 3}

\fancyhead[L]{\coursecode}
\fancyhead[C]{\lecturetitle}
\fancyhead[R]{\term}
\fancyfoot[R]{\thepage}

\title{\coursename~(\coursecode) -- \lecturetitle}
\author{\notetaker~~|~~Instructor: \instructor}
\date{\term}

\begin{document}
  
\maketitle
\pagebreak

\setcounter{section}{3}

\begin{problem}
Consider the circuit below which is composed of resistors (which you know) and a new component
called a comparator. Comparators are quite easy to understand – they simply compare the voltages at
their inputs. If the positive input is greater than the negative input, then the comparator outputs a
digital “1”. If the negative input is greater, then the comparator outputs a digital “0”. Comparators do
not draw any current from their inputs, i.e. they do not disturb the circuit they are measuring. If the
outputs of the comparators are (from top down) [0001] then we interpret the digital output D as “D=1”;
if the output of the comparators is [0011] then we interpret the digital output D as “2”, and so forth.
This is “thermometer code” because the 1’s at the output rise like mercury in an old thermometer.

\begin{enumerate}[\alph*)]

\item
Calculate the voltage at the positive input to each comparator assuming vin is set by a voltage
source. Call it vin,low (note that this node is not connected to the other resistor string; the wire
lines “hop over” the other resistor string). You expect the maximum value of vin to be 19.8 V
and each pin of the comparator is rated to handle up to 3.3 V; what must the ratio R1 / R2 be so
that you can measure the full input voltage range while using the full pin rating? Please use that
value for the rest of the problem.

\item
Calculate the voltage at the negative input to each comparator assuming VDD is set by a voltage
source. Call them Vref1, Vref2, and Vref3. Let VDD = 3.3V.

\item
Sketch a graph of the digital output D (“1”, “2”, etc.) as a function of vin. How many different
levels can D represent? How many bits would you need to represent this number of levels in
binary code (i.e., how many bits of resolution are there)?

\item
How many more resistors and comparators would you need to add one more bit of precision?

\item
Discrete resistors are typically rated to dissipate about 1/4 W of power each. Using this rating
and the ratio you found in (a), calculate the minimum allowable values for R1,R2, and R3.

\end{enumerate}

\begin{figure}[H]
\centering
\includegraphics[width=0.293\textwidth]{problem1.png}
\end{figure}

\end{problem}

\begin{soln}
\end{soln}

\begin{enumerate}[\alph*)]

\item
Redraw circuit and use node analysis to solve for $V_a$:

\begin{figure}[H]
\centering
\includegraphics[width=0.65\textwidth]{11.png}
\end{figure}

\begin{align*}
\sum I = I _{R _{1}} - I _{R _{2}} &= 0 \\
\frac{V _{in} - V _{a}}{R _{1}} - \frac{V _{a} - 0}{R _{2}} &= 0 \\
\frac{V _{in} - V _{a}}{R _{1}} &= \frac{V _{a}}{R _{2}} \\
\frac{V _{in}}{R _{1}} &= \frac{V _{a}}{R _{2}} + \frac{V _{a}}{R _{1}} \\
&= V _{a} \left( \frac{1}{R _{1}} + \frac{1}{R _{2}} \right) \\
&= V _{a} \left( \frac{R _{1} + R _{2}}{R _{1} R _{2}} \right) \\
V _{a} &= \boxed{\frac{V _{in} R _{2}}{R _{1} + R _{2}}}
\end{align*}

Plug in $\SI{3.3}{\volt}$ for $V _{a}$ and $\SI{19.8}{\volt}$ for $V _{in}$:

\begin{align*}
V _{a} &= \frac{V _{in} R _{2}}{R _{1} + R _{2}} \\
3.3 &= \frac{19.8 R _{2}}{R _{1} + R _{2}} \\
\frac{3.3}{19.8} &= \frac{R _{2}}{R _{1} + R _{2}} \\
\frac{19.8}{3.3} &= \frac{R _{1} + R _{2}}{R _{2}} \\
&= \frac{R _{1}}{R _{2}} + 1 \\
\frac{R _{1}}{R _{2}} &= \frac{19.8}{3.3} - 1 \\
&= \boxed{5}
\end{align*}

\item
Use node analysis on node B to set up KCL equation:

\begin{align*}
\sum I = \frac{V _{DD} - V _{b}}{R _{3} / 2} - \frac{V _{b} - V _{c}}{R _{3}} = 0
\end{align*}

Use node analysis on node C to set up KCL equation:

\begin{align*}
\sum I = \frac{V _{b} - V _{c}}{R _{3}} - \frac{V _{c} - V _{d}}{R _{3}} = 0
\end{align*}

Use node analysis on node D to set up KCL equation:

\begin{align*}
\sum I = \frac{V _{c} - V _{d}}{R _{3}} - \frac{V _{d} - 0}{R _{3} / 2} = 0
\end{align*}

\begin{remark*}
Notice now that we have a scenario with 3 unknown variables: $V_b$, $V_c$, and $V_d$
and 3 equations. That means that we can solve for each unkown variable
\end{remark*}

\begin{align*}
\frac{2 V _{DD}}{R _{3}} - \frac{2 V_b}{R_3} - \frac{V_b}{R_3} + \frac{V_c}{R_3} = 0 \quad &\Rightarrow \quad V_c = 3 V_b - 2 V _{DD} \\
\frac{V_b}{R_3} - \frac{V_c}{R_3} - \frac{V_c}{R_3} + \frac{V_d}{R_3} = 0 \quad &\Rightarrow \quad V_b = 2V_c - V_d \\
\frac{V_c}{R_3} - \frac{V_d}{R_3} - \frac{2V_d}{R_3} = 0 \quad &\Rightarrow \quad V_c = 3 V_d
\end{align*}

Solve system of equations:

\begin{align*}
V_c &= 3 V_b - 2 V _{DD} \\
&= 3 (2 V_c - V_d) - 2 V _{DD} \\
&= 6 V_c - 3 V_d - 2 V _{DD} \\
-5 V_c &= -3 V_d - 2 V _{DD}
\end{align*}

\begin{align*}
V_c &= 3 V_d \\
5 V_c &= 15 V_d
\end{align*}

\[
\left\{
\begin{aligned}
-5 V_c &= -3 V_d - 2 V _{DD} \\
5 V_c &= 15 V_d
\end{aligned}
\right.
\quad \Rightarrow \quad
0 = 12 V_d - 2 V _{DD}
\quad \Rightarrow \quad
V_d = \frac{1}{6} V _{DD}
\]

\begin{align*}
V_c &= 3 V_d \\
&= 3 (\frac{1}{6} V _{DD}) \\
&= \frac{1}{2} V _{DD}
\end{align*}

\begin{align*}
V_b &= 2 V_c - V_d \\
&= 2 (\frac{1}{2} V _{DD}) - \frac{1}{6} V _{DD} \\
&= \frac{5}{6} V _{DD}
\end{align*}

Plug in $V _{DD} = \SI{3.3}{\volt}$ and solve for reference voltages:

\begin{align*}
V_d = V _{ref1} = \frac{1}{6} (3.3) &= \boxed{\SI{0.55}{\volt}} \\
V_c = V _{ref2} = \frac{1}{2} (3.3) &= \boxed{\SI{1.65}{\volt}} \\
V_b = V _{ref3} = \frac{5}{6} (3.3) &= \boxed{\SI{2.75}{\volt}}
\end{align*}

\item
This problem asks us to find digital output $D$ as a function of
$V _{in}$. The issue is that $D$ is a non-continuous piece-wise
of $V_a$. But this is okay because $V_a$ is a function of $V _{in}$, so
in the end, we are really graphing $D(V_a(V _{in}))$.

\begin{remark*}
I just wanted to make that clear.
\end{remark*}

Consider what $D$ would be given a particular $V_a$:

\begin{align*}
D &= 0 \text{:} \quad 0 < V_a < 0.55 \\
D &= 1 \text{:} \quad 0.55 < V_a < 1.65 \\
D &= 2 \text{:} \quad 1.65 < V_a < 2.75 \\
D &= 3 \text{:} \quad 2.75 < V_a < 3.3 \\
\end{align*}

Rewrite $V_a$ in the form of $V_a(V _{in})$:

\begin{align*}
V_a &= \frac{V _{in} R_2}{R_1 + R_2} \\
\frac{1}{V_a} &= \frac{1}{V _{in}} \frac{R_1 + R_2}{R_2} \\
V _{in} &= V_a \frac{R_1}{R_2} + 1 \\
&= 5 V_a + 1 \\
V_a &= \frac{1}{5}V _{in} - \frac{1}{5}
\end{align*}

Graph $D(V_a)$:

\begin{figure}[H]
\centering
\includegraphics[width=\textwidth]{31c1.png}
\end{figure}

There are two ways to interpret how many bits in binary you need to add.

The first representation is similar to the way the problem statements
represents levels where 4 levels is represented with: 000, 001, 011, 111.
And each next consecutive level is represented with one extra bit. 5 levels
would be 0000, 0001, 0011, 0111, 1111. 6 levels: 00000, 00001, 00011, 00111,
01111, 11111. The formula for how many bits are needed to represent $n$ levels
would be $n _{\text{bits}} = n _{\text{levels}} - 1$. So in this case, 
you would need 3 bits to represent 4 levels.

The second representation is based off of how many distinct numbers each
bit of binary could represent. 4 levels can be represented by 2 bits of
binary: 00, 01, 10, 11. 3 bits of binary can represent $2^3$ levels. $n$ bits
of binary can represent $2^n$ levels.

\item
You would only need 1 more resistor and 1 more comparator to add another
level of precision.

\item
We know that $P _{\text{resistor}} = \frac{V^2}{R}$. Write equation for
$P_1$:

\begin{align*}
P_1 = \frac{1}{4} &= \frac{(19.8 - 3.3)^{2}}{R_1} \\
R_1 &= 4(19.8 - 3.3)^{2} \\
R_1 &= \boxed{\SI{1089}{\ohm}}
\end{align*}

Apply the ratio $R_1/R_2$:

\begin{align*}
\frac{R_1}{R_2} &= 5 \\
R_2 &= \frac{R_1}{5} \\
R_2 &= \frac{1089}{5} \\
R_2 &= \boxed{\SI{217.8}{\ohm}}
\end{align*}

Write equation for $P_3$:

\begin{align*}
P_3 = \frac{1}{4} &= \frac{(1.65 - 0.55)^{2}}{R_3} \\
R_3 &= 4(1.65 - 0.55)^{2} \\
R_3 &= \boxed{\SI{4.84}{\ohm}}
\end{align*}

\end{enumerate}

\pagebreak

\begin{problem}
In radio-frequency (rf) systems, it is common to have a high-power source of energy and
a sensitive detector or load that can’t absorb very much power. In these cases, you can use an
attenuator to reduce the amount of power absorbed by the load by a known amount. Let the energy
source be modeled by a voltage vin with a resistor Rs in series; let the load be modeled by a resistor RL.
Refer to the Figure.

\begin{enumerate}[\alph*)]

\item
Calculate the power delivered to the load when no attenuator is used.

\item
Consider the attenuator in the Figure, which consists of a T-network of two resistors of values R1
and one resistor of value R2. Calculate what the new network “looks like” from the perspective
of the generator. In other words, calculate the equivalent resistance Rnet of everything to the
right of the generator.

\item
Generators of rf energy often behave best when they are loaded with the same resistance as the
generator’s internal resistance (i.e., when RL
=RS), but when we add the attenuator we risk
changing the resistance that the source “sees” (i.e., Rnet). Let RL = Rs = 50 ohm. Normally our
design task would be to find values for R1 and R2 that give a target attenuation while making the
load+attenuator resistance look like 50 ohm. This is pretty messy to calculate in the general case so
let’s do a specific case instead. Suppose that R2 = 25.975 ohm. Calculate the required value of R1 to
make the equivalent resistance of the load+attenuator combination equal to 50 ohm.

\item
Calculate the attenuation of the resulting attenuator circuit, i.e., the ratio of power delivered to
the actual load when the attenuator is inserted in the circuit divided by the power delivered to
the load with no attenuator.

\item
Simulate the circuit you designed in LTspice. What measurements would you take, and what
calculations would you make based on those measurements, to evaluate if your circuit does
what you want it to do? Submit a screenshot of your LTSpice graphs and circuit and, separately,
any calculations you need to do to prove that the circuit works. Be sure to label any nodes
whose voltage you will be measuring so that the graph is self-explanatory. (Currents and powers
don’t need to be labeled since they’re named after their component)

\end{enumerate}

\begin{figure}[H]
\centering
\includegraphics[width=0.46\textwidth]{problem2.png}
\end{figure}

\end{problem}

\begin{soln}
\end{soln}

\begin{enumerate}[\alph*)]

\item
Recall that $P = V \times I$ and that $V$ in a resistor is $IR$. Thus,
power through a resistor can be represented as $P _{\text{resistor}} = I^2R$.

\begin{align*}
\sum V &= V _{in} - V_S - V_L = 0 \\
&= V _{in} - I_S R_S - I_L R_L = 0 \\
&= V _{in} - I_L R_S - I_L R_L = 0 \\
I_L R_L + I_L R_S &= V _{in} \\
I_L &= \frac{V _{in}}{R_L + R_S}
\end{align*}

\begin{align*}
P_L &= I_L^2 R_L \\
&= \boxed{\frac{V _{in}^{2} R_L}{(R_L + R_S)^{2}}}
\end{align*}

\item
For this problem we can just use equivalent resistances to solve for
$R _{eq}$.

\begin{align*}
R _{eq1} &= R_1 + R_L \\
\frac{1}{R _{eq2}} &= \frac{1}{R_2} + \frac{1}{R _{eq1}} \\
&= \frac{1}{R_2} + \frac{1}{R_1 + R_L} \\
&= \frac{R_1 + R_2 + R_L}{R_2(R_1 + R_L)} \\
R _{eq2} &= \frac{R_2(R_1 + R_L)}{R_1 + R_2 + R_L} \\
R _{eq} &= R_1 + R _{eq2} \\
&= \boxed{R_1 + \frac{R_2(R_1 + R_L)}{R_1 + R_2 + R_L}}
\end{align*}

\item
Plug in values of $R_2$, $R_L$, and $R _{eq}$:

\begin{align*}
R _{eq} &= R_1 + \frac{R_2(R_1 + R_L)}{R_1 + R_2 + R_L} \\
50 &= R_1 + \frac{25.975(R_1 + 50)}{R_1 + 25.975 + 50} \\
R_1 &= \boxed{\SI{30.37}{\ohm}}
\end{align*}

\item
Label circuit and perform node analysis.

\begin{figure}[H]
\centering
\includegraphics[width=0.4\textwidth]{23.png}
\end{figure}

Node A:

\begin{align*}
\sum I = \frac{V _{in} - V_a}{R_1 + R_L} - \frac{V_a - V_b}{R_1} - \frac{V_a - 0}{R_2} &= 0 \\
\frac{V _{in} - V_a}{30.37 + 50} - \frac{V_a - V_b}{30.37} - \frac{V_a}{25.975} &= 0 \\
\frac{V _{in} - V_a}{80.37} - \frac{V_a - V_b}{30.37} - \frac{V_a}{25.975} &= 0 \\
\end{align*}

Node B:

\begin{align*}
\sum I = \frac{V_a - V_b}{R_1} - \frac{V_b - 0}{R_L} &= 0 \\
\frac{V_a - V_b}{30.37} - \frac{V_b}{50} &= 0
\end{align*}

\begin{remark*}
Notice now that we have a 2 equation 2 unknown: $V_a$ and $V_b$ system
of equations that we can solve using algebra.
\end{remark*}

\begin{align*}
\frac{V_a - V_b}{30.37} - \frac{V_b}{50} &= 0 \\
\frac{V_a}{30.37} - \frac{V_b}{30.37} - \frac{V_b}{50} &= 0 \\
V_a &= V_b + \frac{30.37 V_b}{50} \\
&= V_b + 0.61 V_b \\
&= 1.61 V_b
\end{align*}

\begin{align*}
\frac{V _{in} - V_a}{80.37} - \frac{V_a - V_b}{30.37} - \frac{V_a}{25.975} &= 0 \\
\frac{V _{in} - 1.61 V_b}{80.37} - \frac{1.61 V_b - V_b}{30.37} - \frac{1.61 V_b}{25.975} &= 0 \\
\frac{V _{in}}{80.37} - 0.02 V_b - 0.02 V_b - 0.06 V_b &= 0 \\
\frac{V _{in}}{80.37} &= 0.1 V_b \\
0.124 V _{in} &= V_b \\
V_b &\approx \frac{1}{8} V _{in}
\end{align*}

\begin{align*}
V_a &= 1.61 V_b \\
&= 1.61 (0.125 V _{in}) \\
&= 0.201 V _{in} \\
&\approx \frac{1}{5} V _{in}
\end{align*}

\begin{align*}
I_L &= \frac{V_b}{R_L} \\
&= \frac{V_b}{50} \\
&= \frac{V _{in}}{8(50)} \\
&= \frac{V _{in}}{400}
\end{align*}

\begin{align*}
P _{\text{attenuated}} &= I_L^2 R_L \\
&= \left( \frac{V _{in}}{400} \right)^{2} 50 \\
&= \frac{V _{in}^{2}}{3200}
\end{align*}

\begin{align*}
P _{\text{without}} &= \frac{V _{in}^{2} 50}{(50 + 50)^{2}} \\
&= \frac{V _{in}^{2}}{200}
\end{align*}

\begin{align*}
\frac{P _{\text{attenuated}}}{P _{\text{without}}} &= \frac{V _{in}^{2}}{3200} \frac{200}{V _{in}^{2}} \\
&\approx \boxed{0.063}
\end{align*}

\item
Create attenuator circuit on LTSpice and plug in values for resistors.

\begin{figure}[H]
\centering
\includegraphics[width=0.8\textwidth]{21.png}
\end{figure}

\begin{figure}[H]
\centering
\includegraphics[width=0.8\textwidth]{22.png}
\end{figure}

From the data in the graph:

\begin{align*}
V _{in} &= \SI{5}{\volt} \\
V _{out} &= \SI{-610}{\milli \volt} \\
\end{align*}

From the expected equation:

\begin{align*}
V_b = V _{out} &\approx \frac{1}{8} V _{in} \\
V _{out} &\approx \frac{1}{8} (5) \\
V _{out} &\approx \SI{625}{\milli \volt} \\
\end{align*}

We can see that the expected and experimental data are reflective in
their magnitudes. The difference is caused by the approximation of 
$V _{out} \approx \frac{1}{8} V _{in}$.

\end{enumerate}

\pagebreak

\begin{problem}
Use the definition of linearity to show whether the following SISO functions are linear:

\begin{enumerate}[\alph*)]

\item
$f(x) = mx$, where $m$ is a constant

\item
$f(x) = mx$, where $m = 1/x^2$

\item
$f(x) = mx + b$, where $m$ and $b$ are constants

\item
$f(x) = e^x$

\end{enumerate}

\end{problem}

\begin{soln}
\end{soln}

\begin{enumerate}[\alph*)]

\item
\begin{align*}
f(x) &= mx \\
f(u_1) &= m u_1 \\
f(u_2) &= m u_2
\end{align*}

\begin{align*}
f(x) &= mx \\
f(a u_1 + b u_2) &= m(a u_1 + b u_2) \\
f(a u_1 + b u_2) &= a(m u_1) + b(m u_2) \\
f(a u_1 + b u_2) &= a f(u_1) + b f(u_2) \\
\boxed{\text{Linear :D}}
\end{align*}

\item
\begin{align*}
f(x) &= \frac{1}{x^2} x \\
f(x) &= \frac{1}{x} \\
f(u_1) &= \frac{1}{u_1} \\
f(u_2) &= \frac{1}{u_2}
\end{align*}

\begin{align*}
f(x) &= \frac{1}{x} \\
f(a u_1 + b u_2) &= \frac{1}{a u_1 + b u_2} \\
a f(u_1) + b f(u_2) &= \frac{a}{u_1} + \frac{b}{u_2} \\
\frac{1}{a u_1 + b u_2} &\neq \frac{a}{u_1} + \frac{b}{u_2} \\
\boxed{\text{Not Linear :(}}
\end{align*}

\item
\begin{align*}
f(x) &= mx + b \\
f(u_1) &= m u_1 + b \\
f(u_2) &= m u_2 + b \\
\end{align*}

\begin{align*}
f(x) &= mx + b \\
f(au_1 + cu_2) &= m(au_1 + cu_2) + b \\
f(au_1 + cu_2) &= amu_1 + cmu_2 + b \\
af(u_1) + cf(u_2) &= a m u_1 + ab + c m u_2 + cb \\
amu_1 + cmu_2 + b &\neq amu_1 + ab + cmu_2 + cb \\
\boxed{\text{Not Linear :(}}
\end{align*}

\item
\begin{align*}
f(x) &= e^x \\
f(u_1) &= e ^{u_1} \\
f(u_2) &= e ^{u_2}
\end{align*}

\begin{align*}
f(x) &= e^x \\
f(au_1 + bu_2) &= e ^{au_1 + bu_2} \\
af(u_1) + bf(u_2) &= ae ^{u_1} + be ^{u_2} \\
e ^{au_1 + bu_2} &\neq ae ^{u_1} + be ^{u_2} \\
\boxed{\text{Not Linear :(}}
\end{align*}

\end{enumerate}

\pagebreak

\begin{problem}
In a previous homework, we analyzed a MOSFET-based circuit. We said that the
MOSFET could be modeled by a current source of value gmvgs, where gm was called the
transconductance and vgs was the voltage between the MOSFET gate and source. This model is actually
a simplification from a more complete model for the current that flows from drain to source, given by

\begin{figure}[H]
\centering
\includegraphics[width=0.3\textwidth]{problem41.png}
\end{figure}

where micro n is the mobility of electrons, Cox is the gate capacitance per unit area, Vth is called the
“threshold voltage” of the MOSFET, and W and L are the MOSFET width and length. The parameter lambda
represents channel-length modulation (if lambda = 0, then the term 1 + lambda vds falls out of the expression). We’ll
learn more about all of these parameters later in the course.

\begin{enumerate}[\alph*)]

\item
Please identify two features of the above equation that make it nonlinear. (Hint – identify which
terms are circuit variables and which ones are constants)

\item
We have surprisingly few tools for dealing with nonlinear systems. The most common thing we
can do is “linearize” the system, which means that we need to find a linear equation that is
approximately equal to the real equation for small deviations away from an operating point. This
is also called “small-signal modeling.” Replace the circuit quantities i, vgs, and vds with an
operating point value plus a small deviation away from that operating point. Thus i → I + delta i, vgs
→ Vgs + delta vgs, and vds → Vds + delta vds. Show that the resulting equation reduces to the following:

\begin{figure}[H]
\centering
\includegraphics[width=0.7\textwidth]{problem42.png}
\end{figure}

Be sure to not confuse the operating point values (indicated by capital letter) from the small
perturbation values (indicated by delta). You will need to cancel an object on the left hand side with
an equal object on the right hand side. Further, you will need to make an approximation that
any product of small quantities can be approximated as zero (a small number times a small
number is a really small number). So, any terms in your answer with two or more delta quantities
multiplied together can be approximated as zero. These are the nonlinear terms –
approximating them as zero is the “linearization” part of this process.

\item
Using sentences and a circuit diagram, show how this equation can be converted to a dependent
current source in parallel with a resistor. What is the current source voltage-to-current constant
gm? What is the resistance value ro? You can see now that the previous homework lied to you
just a little bit. The model was correct, but it only models the behavior of the deviations of the 
variables from their operating points (the previous homework also assumed lambda=0). Still, that’s
often all we care about, so the result is useful.

\end{enumerate}

\end{problem}

\begin{soln}
\end{soln}

\begin{enumerate}[\alph*)]

\item
Given the list of constants that are defined to us. We can simplify down the
equation to represent constants together.

\begin{align*}
i = C_1(v _{gs} - C_2)^{2}(1 + \lambda v _{ds})
\end{align*}

\begin{enumerate}

\item
The first giveaway is the $(v _{gs} + C_2)^{2}$ term. In a linear equation
the dependent variables have to be in the first order, but when
$(v _{gs} + C_2)^{2}$ is expanded, there will be a dependent variable in the
second order, violating the rule of linearity.

\item
The next giveaway is that when this equation is fully expanded, there will
be a $+ C$ term when the equation is fully expanded. In the previous problem,
we saw that $f(x) = mx + b$ is non linear because of the added constant
term, so we can conclude that the $+ C$ term here will also violate
linearity.

\end{enumerate}

\item
\begin{align*}
i &= C_1(v _{gs} - C_2)^{2}(1 + \lambda v _{ds}) \\
I + \Delta i &= C_1(V _{gs} + \Delta v _{gs} - C_2)^{2}(1 + \lambda (V _{ds} + \Delta v _{ds})) \\
I + \Delta i &= C_1(V _{gs} + \Delta v _{gs} - C_2)^{2}(1 + \lambda V _{ds} + \lambda \Delta v _{ds}) \\
I + \Delta i &= C_1((V _{gs} - C_2)^{2} + 2 \Delta v _{gs} (V _{gs} - C^2) + (\Delta v _{gs})^{2})(1 + \lambda V _{ds} + \lambda \Delta v _{ds}) \\
I + \Delta i &= C_1((V _{gs} - C_2)^{2} + 2 \Delta v _{gs} (V _{gs} - C^2) + 0)(1 + \lambda V _{ds} + \lambda \Delta v _{ds}) \\
I + \Delta i &= C_1((V _{gs} - C_2)^{2} + 2 \Delta v _{gs} (V _{gs} - C^2))(1 + \lambda V _{ds} + \lambda \Delta v _{ds}) \\
I + \Delta i &= i + C_1 2 \Delta v _{gs} (V _{gs} - C^2)(1 + \lambda V _{ds}) + C_1((V _{gs} - C_2)^{2} \lambda \Delta v _{ds} + 0) \\
I + \Delta i &= i + C_1 2 \Delta v _{gs} (V _{gs} - C^2)(1 + \lambda V _{ds}) + C_1(V _{gs} - C_2)^{2} \lambda \Delta v _{ds} \\
\Delta i &= C_1 2 \Delta v _{gs} (V _{gs} - C^2)(1 + \lambda V _{ds}) + C_1(V _{gs} - C_2)^{2} \lambda \Delta v _{ds} \\
\Delta i &= C_1 2 \Delta v _{gs} (V _{gs} - V _{TH})(1 + \lambda V _{ds}) + C_1(V _{gs} - V _{TH})^{2} \lambda \Delta v _{ds} \\
\Delta i &= \mu _{n} C _{ox} \frac{W}{L} \Delta v _{gs} (V _{gs} - V _{TH})(1 + \lambda V _{ds}) + \frac{1}{2} \mu _{n} C _{ox} \frac{W}{L}(V _{gs} - V _{TH})^{2} \lambda \Delta v _{ds} \\
\end{align*}

\item
The equation given to use here can be represented in the form of
\begin{align*}
\Delta i &= A \Delta v _{gs} + B \Delta v _{ds}
\end{align*}
where $A$ and $B$ are constants.

In this equation we see the summation of two currents which happens when
a circuit is in parallel, thus agreeing with the problem statement.

Consider this circuit:

\begin{figure}[H]
\centering
\includegraphics[width=0.4\textwidth]{4c.png}
\end{figure}

In this scenario, $g_m = A$ and $R_0 = \frac{1}{B}$.

\begin{align*}
A &= \mu_n C _{ox} \frac{W}{L} (1 + \lambda V _{ds})(V _{gs} - V _{TH}) \\
B &= \frac{1}{2} \mu_n C _{ox} \frac{W}{L}(V _{gs} - V _{TH})^{2} \lambda
\end{align*}

\begin{align*}
g_m &= \boxed{\mu_n C _{ox} \frac{W}{L} (1 + \lambda V _{ds})(V _{gs} - V _{TH})} \\
R_0 &= \boxed{\left(\frac{1}{2} \mu_n C _{ox} \frac{W}{L}(V _{gs} - V _{TH})^{2} \lambda \right)^{-1}}
\end{align*}

\end{enumerate}

\pagebreak

\begin{problem}
We most commonly encounter dependent voltage sources when we use circuit elements called
operational amplifiers or “op-amps.” An ideal op-amp is nothing more than a voltage-dependent
voltage source; nonideal op-amps also have some input resistance Ri and output resistance Ro.
Therefore, even though we have not yet covered op-amps explicitly in class, you are nevertheless
prepared to analyze op-amp circuits. A circuit involving an op amp is shown in the figure, with its
equivalent circuit model. For all parts, let the parameter A be a variable and then take A→infinity in your
final answers.

\begin{enumerate}[\alph*)]

\item
Using the equivalent circuit model, analyze the circuit for an ideal op-amp, i.e. let Ri=infinity (open
circuit) and let Ro
=0 (short circuit). What is the output voltage for a given input voltage? What
is the voltage at the negative input of the op-amp, v1?

\item
Using the equivalent circuit model, analyze the circuit again for a non-ideal op-amp (Ri=2 M ohm,
Ro
=50 ohm as shown), calculating vo and v1.

\end{enumerate}

\begin{figure}[H]
\centering
\includegraphics[width=0.8\textwidth]{problem5.png}
\end{figure}

\end{problem}

\begin{soln}
\end{soln}

\begin{enumerate}[\alph*)]

\item
The first thing we need to realize is that if $R_i = \infty$, that means
that no current will flow from node 1 through $R_i$ into GND.

The second thing we need to realize is that if $R_0 = 0$, that means
that node 0 is directly connected to the dependent voltage source $Av_d$.
Thus, $v_0 = Av_d$.

The third thing we need to recognize from the circuit is that $v_1 = - v_d$.
This relationship can be seen where $v_d$ is labeled by $R_i$.

Node 1 KCL:
\begin{align*}
\sum I = \frac{v_s - v_1}{10000} - \frac{v_1 - v_0}{20000} &= 0 \\
2v_s - 2v_1 - v_1 + v_0 &= 0 \\
2v_s - 3v_1 + v_0 &= 0 \\
v_0 &= 3v_1 - 2v_s
\end{align*}

\begin{align*}
v_0 &= Av_d \\
v_0 &= -Av_1 \\
v_1 &= - \frac{v_0}{A}
\end{align*}

\begin{align*}
v_0 &= 3v_1 - 2v_s \\
v_0 &= - \frac{3 v_0}{A} - 2v_s \\
v_0 &= - \frac{3 v_0}{\infty} - 2v_s \\
v_0 &= \boxed{-2 v_s}
\end{align*}

\begin{align*}
v_1 &= - \frac{v_0}{A} \\
v_1 &= - \frac{v_0}{\infty} \\
v_1 &= \boxed{0} 
\end{align*}

\item
Node 1 KCL:
\begin{align*}
\sum I = \frac{v_s - v_1}{10000} - \frac{v_1 - 0}{2000000} - \frac{v_1 - v_0}{20000} &= 0 \\
200v_s - 200v_1 - v_1 - 100v_1 + 100v_0 &= 0 \\
200v_s - 301v_1 + 100v_0 &= 0
\end{align*}

Node 0 KCL:
\begin{align*}
\sum I = \frac{v_1 - v_0}{20000} - \frac{v_0 - Av_d}{50} &= 0 \\
\frac{v_1 - v_0}{20000} - \frac{v_0 + Av_1}{50} &= 0 \\
v_1 - v_0 - 400v_0 - 400Av_1 &= 0 \\
v_1 - 401v_0 - 400Av_1 &= 0 \\
\end{align*}

\begin{remark*}
Notice that now it is a 2 equation 2 unknown system of equations.
\end{remark*}

\begin{align*}
200v_s - 301v_1 + 100v_0 &= 0 \\
v_1 &= \frac{200v_s + 100v_0}{301}
\end{align*}

\begin{align*}
v_1 - 401v_0 - 400Av_1 &= 0 \\
v_1(1 - 400A) &= 401v_0 \\
\frac{200v_s + 100v_0}{301}(1 - 400A) = 401v_0 \\
200v_s + 100v_0 - 80000Av_s - 40000Av_0 &= 120701v_0 \\
200v_s + 100v_0 - 80000Av_s - 40000Av_0 - 120701v_0 &= 0 \\
200v_s - 80000Av_s - v_0(40000A + 120701) &= 0 \\
200v_s - 80000Av_s &= v_0(40000A + 120701) \\
v_0 &= \frac{200v_s - 80000Av_s}{40000A + 120701} \\
v_0 &= \frac{200v_s}{40000A + 120701} - \frac{80000Av_s}{40000A + 120701} \\
v_0 &= \frac{200v_s}{40000A + 120701} - \frac{80000v_s}{40000 + \frac{120701}{A}} \\
v_0 &= \frac{200v_s}{40000\infty + 120701} - \frac{80000v_s}{40000 + \frac{120701}{\infty}} \\
v_0 &= 0 - \frac{80000v_s}{40000} \\
v_0 &= \boxed{- 2v_s}
\end{align*}

\begin{align*}
200v_s - 301v_1 + 100v_0 &= 0 \\
v_0 &= \frac{301v_1 - 200v_s}{100}
\end{align*}

\begin{align*}
v_1 - 401v_0 - 400Av_1 &= 0 \\
v_1 - \frac{120701v_1 - 80200v_s}{100} - 400Av_1 &= 0 \\
100v_1 - 120701v_1 - 80200v_s - 40000Av_1 &= 0 \\
-120601 v_1 - 40000Av_1 - 80200v_s &= 0 \\
v_1(-120601 - 40000A) &= 80200v_s \\
v_1 &= \frac{-80200v_s}{120601 + 40000A} \\
v_1 &= \frac{-80200v_s}{120601 + 40000\infty} \\
v_1 &= \boxed{0}
\end{align*}

\end{enumerate}

\pagebreak

\begin{problem}
The first transistor was not a MOSFET, but rather a Bipolar Junction Transistor (BJT, or bipolar
transistor). The BJT has three terminals like the MOSFET, but they are called the Collector (instead of
Drain), the Emitter (instead of Source), and Base (instead of Gate). BJTs have largely been supplanted by
MOSFETs in modern times, but still find use in applications that demand very high speed or high current.

A BJT is shown in the figure below along with its equivalent circuit when in its active mode, which
consists of a fixed voltage source VBE and a current-dependent current source. The parameters VBE
and beta are properties of the transistor. Use Node Analysis to solve for IB and vo in the circuit in the
figure for a BJT with beta=200 and VBE
=0.7 V.

\begin{figure}[H]
\centering
\includegraphics[width=0.85\textwidth]{problem6.png}
\end{figure}

\end{problem}

\begin{soln}
\end{soln}

Redraw circuit:

\begin{figure}[H]
\centering
\includegraphics[width=0.6\textwidth]{61.png}
\end{figure}

Node A KCL:
\begin{align*}
\sum I = \frac{3 - V_a}{6000} - \frac{V_a - 0}{2000} - I_B &= 0 \\
\end{align*}

Node B KCL:
\begin{align*}
\sum I = I_B + 200 I_B - \frac{V_b - 0}{400} &= 0 \\
\end{align*}

Node C KCL:
\begin{align*}
\sum I = - 200 I_B - \frac{V_c - 9}{5000} &= 0 \\
\end{align*}

Floating voltage source:
\begin{align*}
V_a - V_0 &= 0.7 \\
V_a &= V_0 + 0.7
\end{align*}

Voltage difference equivalents:
\begin{align*}
V_b = V_0
\end{align*}

\begin{remark*}
5 equation 5 unkown :D
\end{remark*}

\begin{align*}
\sum I = \frac{3 - V_a}{6000} - \frac{V_a - 0}{2000} - I_B &= 0 \\
\frac{3 - V_0 - 0.7}{6000} - \frac{V_0 + 0.7 - 0}{2000} - I_B &= 0 \\
\frac{2.3 - V_0}{6000} - \frac{V_0 + 0.7}{2000} - I_B &= 0 \\
\frac{2.3 - V_0}{6000} - \frac{3V_0 + 2.1}{6000} - I_B &= 0 \\
\frac{0.2 - 4 V_0}{6000} - I_B &= 0 \\
\frac{0.2}{6000} - \frac{V_0}{1500} - I_B &= 0 \\
\end{align*}

\begin{align*}
I_B + 200 I_B - \frac{V_b - 0}{400} &= 0 \\
201 I_B - \frac{V_0}{400} &= 0 \\
V_0 &= 80400 I_B
\end{align*}

\begin{align*}
\frac{0.2}{6000} - \frac{V_0}{1500} - I_B &= 0 \\
\frac{0.2}{6000} - \frac{80400I_B}{1500} - I_B &= 0 \\
\frac{0.2}{6000} - 54.6I_B &= 0 \\
I_B &= 0.000000611 \\
I_B &= \boxed{\SI{0.611}{\micro \ampere}}
\end{align*}

\begin{align*}
V_0 &= 80400 I_B \\
V_0 &= 80400 (0.000000611) \\
V_0 &= 0.0491 \\
V_0 &= \boxed{\SI{0.0491}{\volt}} \\
\end{align*}

\end{document}
