\documentclass[11pt]{article}

\usepackage[utf8]{inputenc}
\usepackage[T1]{fontenc}
\usepackage{lmodern}
\usepackage{microtype}

\usepackage[sexy]{evan}

\usepackage{fancyhdr}
\pagestyle{fancy}
\fancyhf{}

\usepackage{amsmath, amssymb, amsthm, mathtools}
\usepackage{siunitx}
\usepackage{graphicx}
\usepackage{float}

\usepackage{pgfplots}
\pgfplotsset{compat=1.18}
\usepgfplotslibrary{statistics}

\newcommand{\coursename}{\textbf{Introduction to Electrical Engineering}}
\newcommand{\coursecode}{ECE 302H}
\newcommand{\term}{Fall 2025}
\newcommand{\instructor}{Dr.\ Hanson}
\newcommand{\notetaker}{Dawson Zhang}
\newcommand{\lecturetitle}{Homework 2}

\fancyhead[L]{\coursecode}
\fancyhead[C]{\lecturetitle}
\fancyhead[R]{\term}
\fancyfoot[R]{\thepage}

\title{\coursename~(\coursecode) -- \lecturetitle}
\author{\notetaker~~|~~Instructor: \instructor}
\date{\term}

\begin{document}
  
\maketitle
\pagebreak

\setcounter{section}{2}

\begin{problem}
Equivalent Circuits
\end{problem}

\begin{soln}
\end{soln}
\begin{enumerate}[\alph*)]
\item Sketch: 
\begin{figure}[H]
\centering
\includegraphics[width=0.35\textwidth]{12a.png}
\end{figure}
KVL:
\begin{align*}
  \sum V = -V _{\text{test}} - V _{\text{s}} &= 0 \\
  V _{\text{test}} &= \boxed{- V _{\text{s}}} \\
\end{align*}
i-v Characteristic:
\begin{figure}[H]
\centering
\includegraphics[width=0.35\textwidth]{12a1.png}
\end{figure}
\item Sketch:
\begin{figure}[H]
\centering
\includegraphics[width=0.35\textwidth]{12b.png}
\end{figure}
KCL:
\begin{align*}
  \sum I = I _{\text{test}} - I _{\text{s}} &= 0 \\
I _{\text{test}} &= \boxed{I _{\text{s}}}
\end{align*}
i-v Characteristic:
\begin{figure}[H]
\centering
\includegraphics[width=0.35\textwidth]{12b1.png}
\end{figure}
\item Sketch:
\begin{figure}[H]
\centering
\includegraphics[width=0.35\textwidth]{12c.png}
\end{figure}
i-v Characteristic:
\begin{align*}
  \sum V = V _{\text{test}} - V _{R_1} - V _{R_2} &= 0 \\
  V _{\text{test}} &= V _{R_1} + V _{R_2} \\
  \sum I = I _{\text{test}} - I _{R_1} &= 0 \\
  \sum I = I _{R_1} - I _{R_2} &= 0 \\
  I _{\text{test}} &= I _{R_1} = I _{R_2} \\
  V _{\text{test}} &= I _{\text{test}} R_1 + I _{\text{test}} R_2 \\
                   &= I _{\text{test}}(R_1 + R_2) \\
                   &= I _{\text{test}} R _{\text{eq}} \\
  R _{\text{eq}} &= R_1 + R_2 \\
  I _{\text{test}} &= \frac{1}{R_1 + R_2} V _{\text{test}}
\end{align*}
\begin{figure}[H]
\centering
\includegraphics[width=0.35\textwidth]{12c1.png}
\end{figure}
$R_{\text{eq}} = R_1 + R_2$. If $R_1 \gg R_2$, then $R_{\text{eq}} = R_1$. 
\item Sketch:
\begin{figure}[H]
\centering
\includegraphics[width=0.35\textwidth]{12d.png}
\end{figure}
i-v Characteristic:
\begin{align*}
  \sum V = V _{\text{test}} - V _{R_1} &= 0 \\
  \sum V = V _{\text{test}} - V _{R_2} &= 0 \\
  V _{\text{test}} &= V _{R_1} = V _{R_2} \\
  \sum I = I _{\text{test}} - I _{R_1} - I _{R_2} &= 0 \\
  I _{\text{test}} &= I _{R_1} + I _{R_2} \\
  V _{\text{test}} &= I _{R_1} R_1 = I _{R_2} R_2 \\
  I _{\text{test}} &= \frac{V _{\text{test}}}{R_1} + \frac{V _{\text{test}}}{R_2} \\
                   &= \left( \frac{1}{R_1} + \frac{1}{R_2} \right) V _{\text{test}} \\
  I _{\text{test}} &= \frac{V _{\text{test}}}{R _{\text{eq}}} \\
  \frac{1}{R _{\text{eq}}} &= \frac{1}{R_1} + \frac{1}{R_2}
\end{align*}
\begin{figure}[H]
\centering
\includegraphics[width=0.35\textwidth]{12d1.png}
\end{figure}
$\frac{1}{R_{\text{eq}}} = \frac{1}{R_1} + \frac{1}{R_2}$, If $R_1 \gg R_2$, then $\frac{1}{R_{\text{eq}}} = \frac{1}{R_2}$ so $R_{\text{eq}} = R_2$
\begin{remark*}
Logically this makes a lot of sense because if you have a path of 
higher and lower resistance in parallel, the current will flow through
the one with lower resistance, $R_2$.
\end{remark*}

\item Sketch:
\begin{figure}[H]
\centering
\includegraphics[width=0.35\textwidth]{12e.png}
\end{figure}
i-v Characteristic:
\begin{align*}
  \sum V = - V _{\text{test}} - V _{\text{s}} - V_R &= 0 \\
  V _{\text{test}} &= - V _{\text{s}} - V_R \\
  \sum I = I _{\text{test}} - I _{\text{s}} &= 0 \\
  \sum I = I _{\text{s}} - I _{R} &= 0 \\
  I _{\text{test}} &= I _{\text{s}} = I_R \\
  V _{R} &= I _{R}R \\
         &= I _{\text{test}} R \\
  V _{\text{test}} &= - V _{\text{s}} - I _{\text{test}} R \\
  I _{\text{test}} &= \frac{- V _{\text{s}} - V _{\text{test}}}{R} \\
                   &= - \frac{V _{\text{s}}}{R} - \frac{V _{\text{test}}}{R}
\end{align*}
\begin{figure}[H]
\centering
\includegraphics[width=0.35\textwidth]{12e1.png}
\end{figure}
Open circuit when $I _{\text{test}} = 0$:
\begin{align*}
V _{\text{test}} &= - V _{\text{s}} - I _{\text{test}} R \\
                 &= \boxed{- V _{\text{s}}}
\end{align*}
Short circuit when $V _{\text{test}} = 0$:
\begin{align*}
  I _{\text{test}} &= \frac{- V _{\text{s}} - V _{\text{test}}}{R} \\
                   &= \boxed{\frac{- V _{\text{s}}}{R}}
\end{align*}

\item
Sketch:
\begin{figure}[H]
\centering
\includegraphics[width=0.35\textwidth]{12f.png}
\end{figure}
i-v Characteristic:
\begin{align*}
  \sum V = - V _{\text{test}} - V _{\text{s}} &= 0 \\
  \sum V = - V _{\text{test}} - V _{R} &= 0 \\
  - V _{\text{test}} &= V _{\text{s}} = V _{R} \\
  \sum I = I _{\text{test}} - I _{\text{s}} - I _{R} &= 0 \\
  I _{\text{test}} &= I _{\text{s}} + I_R \\
  V_R &= I_R R \\
  I _{\text{test}} &= I _{\text{s}} + \frac{V_R}{R} \\
                   &= I _{\text{s}} - \frac{V _{\text{test}}}{R}
\end{align*}
\begin{figure}[H]
\centering
\includegraphics[width=0.35\textwidth]{12f1.png}
\end{figure}
Open circuit when $I _{\text{test}} = 0$:
\begin{align*}
  I _{\text{test}} &= I _{\text{s}} - \frac{V _{\text{test}}}{R} \\
  0 &= I _{\text{s}} - \frac{V _{\text{test}}}{R} \\
  V _{\text{test}} &= \boxed{I _{\text{s}} R}
\end{align*}
Short circuit when $V _{\text{test}} = 0$:
\begin{align*}
I _{\text{test}} &= I _{\text{s}} - \frac{V _{\text{test}}}{R} \\
I _{\text{test}} &= \boxed{I _{\text{s}}}
\end{align*}

Now that we have both i-v equations for e) and f), lets see what happens
when we combine them.

\begin{align*}
  I _{\text{test}} &= \frac{- V _{\text{s}} - V _{\text{test}}}{R} \\
  I _{\text{test}} &= I _{\text{s}} - \frac{V _{\text{test}}}{R} \\
  \frac{- V _{\text{s}} - V _{\text{test}}}{R} &= I _{\text{s}} - \frac{V _{\text{test}}}{R} \\
  - V _{\text{s}} - V _{\text{test}} &= I _{\text{s}} R - V _{\text{test}} \\
  - V _{\text{s}} &= I _{\text{s}} R \\
  V _{\text{s}} &= \boxed{- I _{\text{s}} R} \\
  I _{\text{s}} &= \boxed{- V _{\text{s}} R} \\
\end{align*}

\end{enumerate}

\pagebreak

\begin{problem}
Series and parallel; voltage, current, and power
\end{problem}

\begin{soln}
\end{soln}

We're trying to create an object that quadruples the power rating
$P _{\text{total}} = 4P$ while keeping the resistance $R$ the same.

\begin{enumerate}
\item Parallel-Series
\begin{figure}[H]
\centering
\includegraphics[width=0.4\textwidth]{22a.png}
\end{figure}
Check equivalency
\begin{align*}
  R _{\text{eq}} &= \frac{1}{\frac{1}{R + R} + \frac{1}{R + R}} \\
                 &= \frac{1}{\frac{1}{2R} + \frac{1}{2R}} \\
                 &= \frac{1}{\frac{1}{R}} \\
                 &= R
\end{align*}
Solve for $P _{\text{total}}$:
\begin{align*}
  I _{\text{branch}} &= \frac{I}{2} \\
  P &= I _{\text{branch}} ^{2} R \\
  P &= \left( \frac{I}{2} \right)^{2}R \\
         &= \frac{1}{4} I^2 R \\
         &= \frac{1}{4} P _{\text{total}} \\
  P _{\text{total}} &= 4P
\end{align*}
\item Series-Parallel
\begin{figure}[H]
\centering
\includegraphics[width=0.4\textwidth]{22b.png}
\end{figure}
Check equivalency
\begin{align*}
  R _{\text{eq}} &= \frac{1}{\frac{1}{R} + \frac{1}{R}} + \frac{1}{\frac{1}{R} + \frac{1}{R}} \\
                 &= \frac{1}{\frac{2}{R}} + \frac{1}{\frac{2}{R}} \\
                 &= \frac{R}{2} + \frac{R}{2} \\
                 &= R
\end{align*}
Solve for $P _{\text{total}}$:
\begin{align*}
  I _{\text{branch}} &= \frac{I}{2} \\
  P &= I _{\text{branch}} ^{2} R \\
  P &= \left( \frac{I}{2} \right)^{2}R \\
         &= \frac{1}{4} I^2 R \\
         &= \frac{1}{4} P _{\text{total}} \\
  P _{\text{total}} &= 4P
\end{align*}
\end{enumerate}

\pagebreak

\begin{problem}
Resistor equivalent circuit exercise
\end{problem}

\begin{soln}
\end{soln}

\begin{figure}[H]
\centering
\includegraphics[width=0.8\textwidth]{13a.png}
\end{figure}

Use series and parallel resistor equivalent equations to solve for $R_{\text{eq}}$

\begin{align*}
  \frac{1}{R_{(1,2)}} &= \frac{1}{12} + \frac{1}{6} = \frac{1}{6.5} \\
  R_{(1,3)} &= 6.5 + 8 = 14.5 \\
  \frac{1}{R_{(1,5)}} &= \frac{1}{14.5} + \frac{1}{12} + \frac{1}{6} = \frac{1}{3.14} \\
  R_{(1,6)} &= 3.14 + 5 = 8.14 \\
  \frac{1}{R_{(1,8)}} &= \frac{1}{8.14} + \frac{1}{8} + \frac{1}{4} = \frac{1}{2} \\
  R_{(1,9)} = R_{\text{eq}} &= 2 + 3 = \boxed{\SI{5}{\ohm}}
\end{align*}

Redraw equivalent circuit
\begin{figure}[H]
\centering
\includegraphics[width=0.3\textwidth]{13b.png}
\end{figure}

Use KVL to solve for $I_{R_{\text{eq}}}$:
\begin{align*}
  \sum V = 25 - I_{R_{\text{eq}}} R_{\text{eq}} &= 0 \\
  25 &= I_{R_{\text{eq}}}(5) \\
  I_{R_{\text{eq}}} = I _{R _{9}} &= 5
\end{align*}

\begin{remark*}
In this case, $I_{R_{\text{eq}}} = I _{R _{9}}$ because the current
going into $R _{\text{eq}}$ is the same as $R _{9}$.
\end{remark*}

\begin{align*}
  V _{ R _{9}} &= I _{R _{9}} R_9 = (5)(3) = \boxed{\SI{15}{\volt}}
\end{align*}

Use KVL to solve for $V _{R _{8}}$ and $V _{R _{7}}$:
\begin{align*}
  \sum V = 25 - V _{R _{9}} - V _{R _{8}} &= 0 \\
  \sum V = 25 - V _{R _{9}} - V _{R _{7}} &= 0  \\
  25 - 15 &= V _{R _{8}} = V _{R _{7}} \\
  V _{R _{8}} = V _{R _{7}} &= \boxed{\SI{10}{\volt}}
\end{align*}

Use KCL on node A to solve for $V _{R _{6}}$:
\begin{align*}
  \sum I  = I _{R _{9}} - I _{R _{8}} - I _{R _{7}} - I _{R _{6}} &= 0 \\
  I _{R _{6}} &= 5 - \frac{10}{4} - \frac{10}{8} = 1.25 \\
  V _{R _{6}} &= I _{R _{6}} R_6 \\
              &= (1.25)(5) = \boxed{\SI{6.25}{\volt}}
\end{align*}

Use KVL to solve for $V _{R _{5}}$ and $V _{R _{4}}$:
\begin{align*}
  \sum V = 25 - V _{R _{9}} - V _{R _{6}} - V _{R _{5}} &= 0 \\
  \sum V = 25 - V _{R _{9}} - V _{R _{6}} - V _{R _{4}} &= 0 \\
  25 - 15 - 6.25 &= V _{R _{5}} = V _{R _{4}} \\
  V _{R _{5}} = V _{R _{4}} &= \boxed{\SI{3.75}{\volt}}
\end{align*}

Use KCL on node B to solve for $V _{R _{3}}$:
\begin{align*}
  \sum I  = I _{R _{6}} - I _{R _{5}} - I _{R _{4}} - I _{R _{3}} &= 0 \\
  I _{R _{3}} &= 1.25 - \frac{3.75}{6} - \frac{3.75}{12} = 0.3125 \\
  V _{R _{3}} &= I _{R _{3}} R_3 \\
              &= (0.3125)(8) = \boxed{\SI{2.5}{\volt}}
\end{align*}

Use KVL to solve for $V _{R _{2}}$ and $V _{R _{1}}$:
\begin{align*}
  \sum V = 25 - V _{R _{9}} - V _{R _{6}} - V _{R _{3}} - V _{R _{2}} &= 0 \\
  \sum V = 25 - V _{R _{9}} - V _{R _{6}} - V _{R _{3}} - V _{R _{1}} &= 0 \\
  25 - 15 - 6.25 - 2.5 &= V _{R _{2}} = V _{R _{1}} \\
  V _{R _{2}} = V _{R _{1}} &= \boxed{\SI{1.25}{\volt}}
\end{align*}
\pagebreak

\begin{problem}
Solving Single-Loop Circuits
\end{problem}

\begin{soln}
\end{soln}

\begin{enumerate}[\alph*)]

\item
\begin{align*}
  \sum V = V _{\text{in}} - V _{\text{diode}} - V _{R} &= 0 \\
  V _{R} &= V _{\text{in}} - V _{\text{diode}} \\
  I _{R} = I &= \frac{V _{R}}{R} = \frac{V _{\text{in}} - V _{\text{diode}}}{R} \\
  I &\propto \frac{1}{R} 
\end{align*}

\begin{remark*}
Here $I = I_R$ beacuse of KCL. There is only one path current flows in so
it must be the same current that flows through all the components.
\end{remark*}

As $R$ approaches $0$, current approaches infinity. In real life, this
would cause a short circuit because current will continuously ramp up
without resistance.

\item Label voltage drops and current direction
\begin{figure}[H]
\centering
\includegraphics[width=0.3\textwidth]{24b.png}
\end{figure}
Use KVL and KCL to solve for $I$
\begin{align*}
  \sum V = V _{\text{in}} - V _{\text{fwd}} - V _{R} &= 0 \\
  V _{R} &= V _{\text{in}} - V _{\text{fwd}} \\
  I _{\text{in}} = I _{\text{fwd}} = I _{R} &= I \\
  IR &= V _{\text{in}} - V _{\text{fwd}} \\
  I &= \boxed{\frac{V _{\text{in}} - V _{\text{fwd}}}{R}}
\end{align*}

\item
  Typical intensity $= \SI{35}{\milli\candela}$ @ $\SI{20}{\milli \ampere}$

\item
From the graph, we need approximately $\SI{27}{\milli \ampere}$ of current
to supply $\SI{45}{\milli\candela}$ of luminosity.
\begin{align*}
  R &= \frac{V _{\text{in}} - V _{\text{fwd}}}{I} \\
    &= \frac{5 - 2.2}{0.027} \\
    &= \boxed{\SI{103.7}{\ohm}}
\end{align*}

\end{enumerate}

\pagebreak
\begin{problem}
Circuit Tradeoffs
\end{problem}

\begin{soln}
\end{soln}

\begin{enumerate}[\alph*)]

\item
KVLs:
\begin{align*}
  \sum V = V _{\text{in}} - V _{\text{gs}} &= 0 \\
  V _{\text{in}} &= V _{\text{gs}} \\
  \sum V = V_m - V _{\text{out}} &= 0 \\
  V_m &= V _{\text{out}} \\
  \sum V = V _{\text{out}} - V _{\text{D}} &= 0 \\
  V _{\text{out}} &= V _{\text{D}} \\
  V _{\text{out}} &= V _{\text{D}} = V_m 
\end{align*}
KCLs:
\begin{align*}
  \sum I = - I _{\text{in}} - I _{\text{gs}} &= 0 \\
  I _{\text{gs}} &= - I _{\text{in}} \\
  \sum I = - I_m - I _{\text{D}} - I _{\text{out}} &= 0 \\
  I _{\text{D}} &= - I_m - I _{\text{out}} \\
\end{align*}
i-v's:
\begin{align*}
  V _{\text{D}} &= I _{\text{D}} R _{\text{D}} \\
  I_m &= V _{\text{gs}} G_m
\end{align*}
Solve for $V _{\text{out}}$:
\begin{align*}
  V _{\text{out}} &= V _{\text{D}} \\
                  &= I _{\text{D}} R _{\text{D}} \\
                  &= (- I_m - I _{\text{out}}) R _{\text{D}} \\
                  &= (- V _{\text{gs}} G_m - 0) R _{\text{D}} \\
                  &= - V _{\text{gs}} G_m R _{\text{D}} \\
\end{align*}
\begin{remark*}
It says that in the problem statement that you can treat open circuit
elements like $I _{\text{out}}$ as $i = 0$.
\end{remark*}
Solve for $V _{\text{out}} / V _{\text{in}}$:
\begin{align*}
  \frac{V _{\text{out}}}{V _{\text{in}}} &= \frac{- V _{\text{gs}} G_m R _{\text{D}}}{V _{\text{gs}}} \\
                                         &= \boxed{- G_m R _{\text{D}}}
\end{align*}

\item
If $G_m$ changes by a factor of $2$, the gain would also change by a
factor of $2$.

\item
KVLs:
\begin{align*}
  \sum V = V _{\text{in}} - V _{\text{gs}} - V _{\text{S}} &= 0 \\
  V _{\text{in}} &= V _{\text{gs}} + V _{\text{S}} \\
  \sum V = V _{\text{S}} + V_m - V _{\text{out}} &= 0 \\
  \sum V = V _{\text{out}} - V _{\text{D}} &= 0 \\
  V _{\text{out}} &= V _{\text{D}} \\
\end{align*}
KCLs:
\begin{align*}
  \sum I = - I _{\text{in}} - I _{\text{gs}} &= 0 \\
  \sum I = I _{\text{gs}} + I_m - I _{\text{S}} &= 0 \\
  I _{\text{S}} &= I _{\text{gs}} + I_m \\
  \sum I = - I_m - I _{\text{D}} - I _{\text{out}} \\
  I _{\text{D}} &= - I_m - I _{\text{out}}
\end{align*}
i-v's:
\begin{align*}
  V _{\text{S}} &= I _{\text{S}} R _{\text{S}} \\
  V _{\text{D}} &= I _{\text{D}} R _{\text{D}} \\
  I_m &= V _{\text{gs}} G_m
\end{align*}
Solve for $V _{\text{out}}$:
\begin{align*}
  V _{\text{out}} &= V _{\text{D}} \\
                  &= I _{\text{D}} R _{\text{D}} \\
                  &= (- I_m - I _{\text{out}}) R _{\text{D}} \\
                  &= (- V _{\text{gs}} G_m - 0) R _{\text{D}} \\
                  &= - V _{\text{gs}} G_m R _{\text{D}} \\
\end{align*}
Solve for $V _{\text{in}}$:
\begin{align*}
  V _{\text{in}} &= V _{\text{gs}} + V _{\text{S}} \\
                 &= V _{\text{gs}} + I _{\text{S}} R _{\text{S}} \\
                 &= V _{\text{gs}} + (I _{\text{gs}} + I_m) R _{\text{S}} \\
                 &= V _{\text{gs}} + (0 + I_m) R _{\text{S}} \\
                 &= V _{\text{gs}} + I_m R _{\text{S}} \\
                 &= V _{\text{gs}} + V _{\text{gs}} G_m R _{\text{S}} \\
\end{align*}
Solve for $V _{\text{out}} / V _{\text{in}}$:
\begin{align*}
  \frac{V _{\text{out}}}{V _{\text{in}}} &= \frac{- V _{\text{gs}} G_m R _{\text{D}}}{V _{\text{gs}}(1 + G_m R _{\text{S}}} \\
                                         &= \boxed{- \frac{G_m R _{\text{D}}}{1 + G_m R _{\text{S}}}}
\end{align*}
The magnitude of the gain of this amplifier would be smaller because
there is $1 + G_m R _{\text{S}}$ on the denominator.

\item
We can approximate the magnitude of the gain to be,
\begin{align*}
  \frac{V _{\text{out}}}{V _{\text{in}}} &= \frac{G_m R _{\text{D}}}{1 + G_m R _{\text{S}}} \\
  &= \frac{G_m R _{\text{D}}}{G_m R _{\text{S}}} \\
                                         &= \frac{R _{\text{D}}}{R _{\text{S}}}
\end{align*}
This means that even for very large $G_m$, the gain would remain relatively
similar.

\item
Gain would not change with our approximation if it varied by a factor of $2$.

\item
In the first circuit, you can achieve more gain because it is a function
of $G_m$, but in the second circuit, you can control the exact value of
what gain can be.

\end{enumerate}

\end{document}
