\documentclass[11pt]{article}

\usepackage[utf8]{inputenc}
\usepackage[T1]{fontenc}
\usepackage{lmodern}
\usepackage{microtype}

\usepackage[sexy]{evan}

\usepackage{fancyhdr}
\pagestyle{fancy}
\fancyhf{}

\usepackage{amsmath, amssymb, amsthm, mathtools}
\usepackage{graphicx}

\usepackage{pgfplots}
\pgfplotsset{compat=1.18}

\newcommand{\coursename}{\textbf{Sequences, Series, and Multivariable Calculus}}
\newcommand{\coursecode}{MATH 408D}
\newcommand{\term}{Fall 2025}
\newcommand{\instructor}{Dr.\ Sarower}
\newcommand{\notetaker}{Dawson Zhang}
\newcommand{\lecturetitle}{Techniques of Integration}

\fancyhead[L]{\coursecode}
\fancyhead[C]{\lecturetitle}
\fancyhead[R]{\term}
\fancyfoot[C]{\leftmark}
\fancyfoot[R]{\thepage}

\title{\coursename~(\coursecode) -- \lecturetitle}
\author{\notetaker~~|~~Instructor: \instructor}
\date{\term}

\begin{document}
  
\maketitle
\pagebreak

\subsection{u-Substitution Review}

u-Substitution is one of the most useful and prioritized tools when it
comes to solving antiderivatives. The antiderivative power-rule doesn't
help us evaluate integrals such as

\begin{equation*}
\int 2x \sqrt{1 + x^2} dx
\end{equation*}

\begin{remark}
To find this integral, we can change the variable $x$ into another
variable to simplify the integration process.
\end{remark}

\begin{soln}
Consider setting $1+x^2$ to another variable $u$
\begin{equation*}
u = 1 + x^2  
\end{equation*}
Taking the derivative of $u$ with respect to $x$ yields
\begin{align*}
u' = \frac{du}{dx} &= 2x \\
     \frac{du}{2x} &= dx
\end{align*}
Substituting $u$ and $dx$ back into the original equation,
\begin{equation*}
\int \frac{2x \sqrt{u}}{2x} du = \int \sqrt{u} du = \frac{2}{3} u^{3/2} + C
\end{equation*}
Finally, plugging in the original $u$,
\begin{equation*}
= \boxed{\frac{2}{3} (1 + x^2)^{3/2} + C}
\end{equation*}
\end{soln}

\begin{remark}
For the most part, u-substitution usually involves setting the ``inside''
function to u.
\end{remark}

\begin{theorem}
If $u = g(x)$ is a differentiable function whose range is an interval 
$I$ and $f$ is continuous on $I$, then
\begin{equation*}
\int f(g(x))g'(x)dx = \int f(u)du  
\end{equation*}
\end{theorem}

\pagebreak

\begin{example}
\begin{equation*}
\int \cos(5x) dx 
\end{equation*}
\end{example}

\begin{soln}
Set up u-substitution terms
\begin{align*}
u &= 5x \\
\frac{du}{dx} &= 5 \\
 \frac{du}{5} &= dx
\end{align*}
Substitute into original equation
\begin{align*}
\int \cos(5x) dx &= \frac{1}{5} \int \cos(u) du \\
                 &= \frac{1}{5} \sin(u) + C \\
                 &= \boxed{\frac{1}{5} \sin(5x) + C}
\end{align*}
\end{soln}

\begin{example}
\begin{equation*}
\int^{2}_{0} 2x(x^2 + 1)^3 dx  
\end{equation*}
\end{example}

\begin{soln}
Set up u-substitution terms
\begin{align*}
            u &= x^2 + 1 \\
\frac{du}{dx} &= 2x \\
\frac{du}{2x} &= dx
\end{align*}
Substitute into original equation
\begin{align*}
\int^{2}_{0} 2x(x^2 + 1)^3 dx &= \int^{2}_{0} \frac{2x(u)^3}{2x} du \\
                              &= \int^{2}_{0} u^3 du \\
                              &= \frac{1}{4} u^4 \bigg|_0^2 \\
                              &= \frac{1}{4} (x^2 + 1)^4 \bigg|_0^2 \\
                              &= \boxed{156}
\end{align*}
\begin{remark}
In this solution, I chose to substitute the $u = x^2 +1$ equation back
into into my solved integral instead of finding new limits of integration
with resepect to $u$.
\end{remark}
\end{soln}

\begin{example}
\begin{equation*}
\int^{1}_{0} \frac{x}{1 + x^2} dx  
\end{equation*}
\end{example}

\begin{soln}
\begin{align*}
u &= 1 + x^2 \\
\frac{du}{dx} &= 2x \\
dx &= \frac{du}{2x} \\
\end{align*}
\end{soln}

\begin{align*}
\int^{1}_{0} \frac{x}{1 + x^2} dx &= \int^{1}_{0} \frac{x}{u(2x)} du \\
                                    &= \frac{1}{2} \int^{2}_{0} \frac{1}{u} du \\
                                    &= \frac{1}{2} \ln(|u|) \bigg|_1^2 \\
                                    &= \boxed{\frac{1}{2} \ln(2)}
\end{align*}

\pagebreak
\section{Integration by Parts}

Integration by parts comes from the product rule from derivatives. 
Consider the product rule where $f = f(x)$ and $g = g(x)$
\begin{equation*}
  (fg)' = f'g + fg'
\end{equation*}
Integrate both sides of the equation
\begin{align*}
\int (fg)'dx &= \int f'gdx + \int fg'dx \\
fg &= \int f'gdx + \int fg'dx \\
\int fg'dx &= fg - \int f'gdx
\end{align*}

\begin{theorem}
If $f$ and $g$ are differentiable functions then,
\begin{equation*}
\int f(x)g'(x)dx = f(x)g(x) - \int g(x)f'(x)dx
\end{equation*}
or, equivalently,
\begin{equation*}
\int u dv = uv - \int vdu  
\end{equation*}
where $u = f(x)$ and $v = g(x)$
\end{theorem}

\subsection{Steps to consider}

\begin{itemize}
\item Look for a product of two functions, e.g. $f(x)g(x)$
\item Choose $u$ and $dv$
\item Choose $u$ in order of \textbf{L.I.A.T.E} (logarithmic, inverse trig,
algebraic, trig, exponential), choose $u$ in that order
\begin{remark}
In practice, once integration by parts becomes more fluid, it's not
necessary to consider the \textbf{L.I.A.T.E} rule every time.
\end{remark}
\item Differentiate $u$ to find $du$
\item Integrate $dv$ to find $v$
\item Plug into formula $\int udv = uv - \int vdu$ and solve
\end{itemize}

\pagebreak

\begin{example}
\[
\int xe^x dx
\]
\end{example}

\begin{soln}
Find $u$ and $dv$
\begin{align*}
u &= x \\
dv &= e^x dx
\end{align*}
Find $du$
\begin{align*}
u &= x \\
\frac{du}{dx} &= 1 \\
du &= dx
\end{align*}
Find $v$
\begin{align*}
  dv &= e^x dx \\
  \int dv &= \int e^x dx \\
  v &= e^x
\end{align*}
Plug back into equation $\int udv = uv - \int vdu$
\begin{align*}
  \int xe^xdx &= xe^x - \int e^x dx \\
              &= \boxed{xe^x - e^x + C}
\end{align*}
\end{soln}

\begin{example}
\[
\int \ln(x) dx
\]
\end{example}

\begin{soln}
\begin{align*}
  u &= \ln(x) \\
  \frac{du}{dx} &= \frac{1}{x} \\
  du &= \frac{1}{x} dx
\end{align*}
\begin{align*}
  dv &= dx \\
  \int dv &= \int dx \\
  v &= x
\end{align*}
Plug back into equation $\int udv = uv - \int vdu$
\begin{align*}
  \int \ln(x)dx &= x\ln(x) - \int \frac{x}{x} dx \\
                &= \boxed{x\ln(x) - x + C}
\end{align*}
\end{soln}

\begin{example}
\[
\int x\sin(x) dx
\]
\end{example}

\begin{soln}
\begin{align*}
  u &= x \\
  \frac{du}{dx} &= 1 \\
  du &= dx
\end{align*}
\begin{align*}
  dv &= \sin(x) dx \\
  \int dv &= \int \sin(x) dx \\
  v &= -\cos(x)
\end{align*}
Plug back into equation $\int udv = uv - \int vdu$
\begin{align*}
  \int x\sin(x) &= -x\cos(x) + \int \cos(x)dx \\
                &= \boxed{-x\cos(x) + \sin(x)}
\end{align*}
\end{soln}

\begin{example}
\[
\int e^x\cos(x)dx
\]
\end{example}

\begin{soln}
\begin{align*}
  u &= \cos(x) \\
  du &= -\sin(x) dx
\end{align*}
\begin{align*}
  dv &= e^x dx \\
  v &= e^x
\end{align*}
\begin{align*}
  \int e^x \cos(x) dx &= e^x \cos(x) + \int e^x \sin(x) dx 
\end{align*}
\begin{align*}
  u &= \sin(x) \\
  du &= \cos(x) dx \\
  dv &= e^x dx \\
  v &= e^x
\end{align*}
\begin{align*}
  \int e^x \cos(x) dx &= e^x \cos(x) + \int e^x \sin(x) dx \\
                      &= e^x \cos(x) + e^x \sin(x) - \int e^x \cos(x) dx \\
2\int e^x \cos(x) dx &= e^x \cos(x) + e^x \sin(x) \\
\int e^x \cos(x) dx &= \boxed{\frac{e^x \cos(x) + e^x \sin(x)}{2} + C}
\end{align*}
\end{soln}

\begin{example}
\begin{equation*}
\int^{1}_{0} \arctan(x)dx
\end{equation*}
\end{example}

\section{Trigonometric Integration}

\subsection{Review of Trig Identities}
\begin{align}
  \sin^2(x) + \cos^2(x) &= 1 \\
  \sec^2(x) - \tan^2(x) &= 1 \\
  \csc^2(x) - \cot^2(x) &= 1 \\
\end{align}

\begin{align}
  \sin(2x) &= 2\sin(x)\cos(x) \\
  \cos(2x) &= \cos^2(x) - \sin^2(x) \\
  \sin^2x &= \frac{1 - \cos(2x)}{2} \\
  \cos^2x &= \frac{1  \cos(2x)}{2} \\
\end{align}

\begin{example}
\begin{equation*}
\int \cos^5(x)dx  
\end{equation*}
\end{example}

\begin{soln}
Plug in pythagorean identities:
\begin{align*}
  \int \cos^5(x)dx &= \int \cos^4(x)\cos(x)dx \\
                   &= \int (1 - \sin^2(x))^2\cos(x) dx \\
\end{align*}
Use u-substitution:
\begin{align*}
  u &= \sin(x) \\
  \frac{du}{dx} &= \cos(x) \\
  du &= \cos(x) dx
\end{align*}
Solve:
\begin{align*}
  \int \cos^5(x)dx &= \int (1-u^2)^2du
\end{align*}
\begin{remark}
Clever students have noticed that $u = 1-\sin^2(x)$ is probably easier
\end{remark}
\end{soln}

\subsection{Products of sine and cosine}

\begin{example}
\begin{equation*}
\int \sin^5(x)\cos^2(x)dx  
\end{equation*}
\end{example}

\begin{soln}
Plug in pythagorean identities:
\begin{align*}
  \int \sin^5(x)\cos^2(x)dx &= \int \sin^4(x)\cos^2(x)\sin(x) dx \\
   &= \int (1-\cos^2(x))^2\cos^2(x)\sin(x) dx
\end{align*}
Use u-substitution:
\begin{align*}
  u& = \cos(x) \\
  \frac{du}{dx} &= -\sin(x) \\
  du &= -\sin(x)dx
\end{align*}
Solve:
\begin{align*}
  \int \sin^5(x)\cos^2(x)dx &= -\int(1-u^2)^2u^2du
\end{align*}
\end{soln}

\begin{example}
\begin{equation*}
\int \sin^2(x)\cos^2(x)dx  
\end{equation*}
\end{example}

\begin{soln}
Plug in identities
\begin{align*}
  \int \sin^2(x)\cos^2(x)dx &= \int \frac{1-\cos(2x)}{2} \times \frac{1+\cos(2x)}{2} dx \\
                            &= \frac{1}{4} \int 1 - \cos^2(2x) dx \\
                            &= \frac{1}{4} \int \sin^2(2x) dx \\
                            &= \frac{1}{4} \int \frac{1 - \cos(4x)}{2} dx \\
                            &= \frac{1}{8} \int 1 - \cos(4x) dx
\end{align*}
\end{soln}

\subsection{Products of secant and tangent}

\begin{example}
\begin{equation*}
\int \sec^4 x dx  
\end{equation*}
\end{example}

\begin{soln}
\begin{align*}
  \int \sec^4 x dx &= \int \sec^2 x \sec^2 x dx \\
\begin{aligned}
u &= \tan x \\
du &= \sec^2 x dx
\end{aligned} \\
                   &= \int (1 + \tan^2 x) \sec^2 x dx \\
                   &= \int (1 + u^2) du
\end{align*}
\end{soln}

\begin{example}
\begin{equation*}
\int \tan^6 x \sec^4 x dx  
\end{equation*}
\end{example}

\begin{soln}
\begin{align*}
  \int \tan^6 x \sec^4 x dx &= \int \tan^6 x \sec^2 x \sec^2 x dx \\
                            &= \int \tan^6 x (1 + \tan^2 x) \sec^2 x dx \\
\begin{aligned}
u &= \tan x \\
du &= \sec^2 x dx
\end{aligned} \\
                            &= \int u^6(1 + u^2) du
\end{align*}
\end{soln}

\begin{example}
\begin{equation*}
\int \tan^5 x \sec^7 x dx  
\end{equation*}
\end{example}

\begin{soln}
\begin{align*}
  \int \tan^5 x \sec^7 x dx &= \int \tan^4 x \sec^6 x \tan x \sec x dx \\
                            &= \int (\sec^2x - 1)^{2} \sec^6 x \tan x \sec x dx \\
                            \begin{aligned}
                            u &= \sec x \\
                            du &= \tan x \sec x dx
                            \end{aligned} \\
                            &= \int (u^2 - 1)^{2} u^6 du
\end{align*}
\end{soln}

\begin{example}
\begin{equation*}
\int \sec^2 x \tan^3 x dx   
\end{equation*}
\end{example}

\begin{soln}
\begin{align*}
\begin{aligned}
u &= \tan x \\
du &= \sec^2 x dx
\end{aligned} \\
\int \sec^2 x \tan^3 x dx &= u^3 du \\
                          &= \frac{1}{4} u^4 + C \\
                          &= \boxed{\frac{1}{4} \tan^4 x + C}
\end{align*}
\end{soln}

\subsection{Proudct of sine and cosine with different angles}

\begin{align*}
  \sin A \cos B &= \frac{1}{2} (\sin (A + B) + \sin(A - B)) \\
  \sin A \sin B &= \frac{1}{2} (\cos (A - B) - \cos(A + B)) \\
  \cos A \cos B &= \frac{1}{2} (\cos (A - B) + \cos(A + B)) \\
\end{align*}

\begin{example}
\begin{equation*}
\int \sin (3x) \cos (2x) dx 
\end{equation*}
\end{example}

\begin{soln}
\begin{align*}
\int \sin (3x) \cos (2x) dx &= \frac{1}{2} \int (\sin (5x) + \sin x) dx 
\end{align*}
\end{soln}

\end{document}
