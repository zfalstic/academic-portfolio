\documentclass[11pt]{article}

\usepackage[utf8]{inputenc}
\usepackage[T1]{fontenc}
\usepackage{lmodern}
\usepackage{microtype}

\usepackage[sexy]{evan}

\usepackage{fancyhdr}
\pagestyle{fancy}
\fancyhf{}

\usepackage{amsmath, amssymb, amsthm, mathtools}
\usepackage{siunitx}
\usepackage{graphicx}
\usepackage{float}

\usepackage{pgfplots}
\pgfplotsset{compat=1.18}
\usepgfplotslibrary{statistics}

\newcommand{\coursename}{\textbf{Introduction to Electrical Engineering}}
\newcommand{\coursecode}{ECE 302H}
\newcommand{\term}{Fall 2025}
\newcommand{\instructor}{Dr.\ Hanson}
\newcommand{\notetaker}{Dawson Zhang}
\newcommand{\lecturetitle}{Homework 4}

\fancyhead[L]{\coursecode}
\fancyhead[C]{\lecturetitle}
\fancyhead[R]{\term}
\fancyfoot[R]{\thepage}

\title{\coursename~(\coursecode) -- \lecturetitle}
\author{\notetaker~~|~~Instructor: \instructor}
\date{\term}

\begin{document}
  
\maketitle
\pagebreak

\setcounter{section}{4}

\begin{problem}

The most commonly-used nonlinear circuit component is the MOSFET. The I-V characteristic of the
MOSFET (ignoring channel length modulation) is given by

\begin{equation*}
i = \frac{1}{2} \mu_n C _{ox} \frac{W}{L} (v _{gs} - V _{TH})^{2}
\end{equation*}

\begin{enumerate}[\alph*)]

\item

Formally show that this equation is nonlinear by using the definition of linearity.

\item

It is common for $v _{gs}$ to be the sum of multiple sine waves. Let $v _{gs} = V_1 \sin(\omega t) + V_3 \sin(3 \omega t)$,
which can be modeled as two voltage sources in series, one of value $V_1 \sin(\omega t)$ and the other of
value $V_3 \sin(3 \omega t)$. With two sources, it may be tempting to apply superposition to the circuit to
solve for the current, i.e. to solve for the current at $1 \omega$ and the current at $3 \omega$ separately. Show
that this can not be done by calculating the current that you would get if you applied
superposition and comparing it to the actual current. The fact that superposition does not work
for nonlinear circuits will seriously hamper us later on. (You will learn later in the course that
any periodic function can be represented by a sum of sine waves, one of the most impactful and
remarkable observations of the 19th century)

\item

In HW3, we linearized the MOSFET (aka, we found the small-signal model for the MOSFET) by
saying that each variable could be represented by an operating point plus a deviation. For
example, $i \rightarrow I + \Delta i$. We then saw that the operating point terms on both sides cancelled and
then we approximated any terms with a product of two deviations (e.g., $\Delta v _{gs} ^{2}$) as zero. This
time, linearize the equation above by taking the derivative of $i$ with respect to $v _{gs}$ and
constructing a line that is tangent to the true $i(v _{gs})$ at an operating point $(V _{gs}, I)$.

\end{enumerate}

\end{problem}

\begin{soln}
\end{soln}

\begin{enumerate}[\alph*)]

\item

Notice that there are a lot of constants in this equation that can be represented as a simple
$C$ variable for clarity.

Let,

\begin{align*}
C_1 &= \frac{1}{2} \mu_n C _{ox} \frac{W}{L} \\
C_2 &= V _{TH}
\end{align*}

Rewrite the equation by substitution in $C_1$ and $C_2$,

\begin{align*}
i &= \frac{1}{2} \mu_n C _{ox} \frac{W}{L} (v _{gs} - V _{TH})^{2} \\
i &= C_1(v _{gs} - C_2)^{2} \\
\end{align*}

For the purposes of using the definition of linearity, I will represent $i(v _{gs})$ as
$f(x)$. Rewrite the equation,

\begin{align*}
i(v _{gs}) &= C_1(v _{gs} - C_2)^{2} \\
f(x) &= C_1(x - C_2)^{2}
\end{align*}

Recall that by the definition of linearity, the function $f$ is linear if it satisfies,

\begin{align*}
f(au_1 + bu_2) &= af(u_1) + bf(u_2)
\end{align*}

Let $x = au_1 + bu_2$,

\begin{align*}
f(x) &= C_1(x - C_2)^{2} \\
f(au_1 + bu_2) &= C_1(au_1 + bu_2 - C_2)^{2} \\
\end{align*}

Let $x = u_1$ and $x = u_2$,

\begin{align*}
f(x) &= C_1(x - C_2)^{2} \\
f(u_1) &= C_1(u_1 - C_2)^{2} \\
f(u_2) &= C_1(u_2 - C_2)^{2} \\
\end{align*}

Test linearity,

\begin{align*}
f(au_1 + bu_2) &= C_1(au_1 + bu_2 - C_2)^{2} \\
af(u_1) + bf(u_2) &= a C_1(u_1 - C_2)^{2} + b C_1(u_2 - C_2)^{2} \\
\end{align*}

\begin{remark*}
You could distribute this out more to fully make sure that these aren't equal, but I think that
it is pretty obvious these two equations can't possibly be equal. If you pay attention to
the $a$ and $b$, notice how in the first equation you will eventuall get some $a^2$ and 
$b^2$ coefficient which is impossible to get in the second equation.
\end{remark*}

\begin{align*}
C_1(au_1 + bu_2 - C_2)^{2} &\neq a C_1(u_1 - C_2)^{2} + b C_1(u_2 - C_2)^{2} \\
f(au_1 + bu_2) &\neq af(u_1) + bf(u_2) \\
\end{align*}

Not linear.

\item

Apply the same constant substitution as in part a,

\begin{align*}
i &= \frac{1}{2} \mu_n C _{ox} \frac{W}{L} (v _{gs} - V _{TH})^{2} \\
i &= C_1(v _{gs} - C_2)^{2} \\
\end{align*}

Consider the case when $v _{gs} = V_1 \sin(\omega t) + V_3 \sin(3 \omega t)$,

\begin{align*}
i &= C_1(v _{gs} - C_2)^{2} \\
i &= C_1(V_1 \sin(\omega t) + V_3 \sin(3 \omega t) - C_2)^{2}
\end{align*}

This equation models the ACTUAL current that you would get without applying
superposition.

Apply superposition $V_3 \sin(3 \omega t) = 0$,

\begin{align*}
i &= C_1(v _{gs} - C_2)^{2} \\
i_1 &= C_1(V_1 \sin(\omega t) - C_2)^{2} \\
\end{align*}

Apply superposition $V_1 \sin(\omega t) = 0$,

\begin{align*}
i &= C_1(v _{gs} - C_2)^{2} \\
i_2 &= C_1(V_3 \sin(3 \omega t) - C_2)^{2} \\
\end{align*}

Sum $i_1 + i_2$,

\begin{align*}
i &= i_1 + i_2 \\
i &= C_1(V_1 \sin(\omega t) - C_2)^{2} + C_1(V_3 \sin(3 \omega t) - C_2)^{2} \\
\end{align*}

Compare ACTUAL current and current calculated via superposition,

\begin{align*}
i _{\text{actual}} &= C_1(V_1 \sin(\omega t) + V_3 \sin(3 \omega t) - C_2)^{2} \\
i _{\text{superposition}} &= C_1(V_1 \sin(\omega t) - C_2)^{2} + C_1(V_3 \sin(3 \omega t) - C_2)^{2} \\
i _{\text{actual}} &\neq i _{\text{superposition}}
\end{align*}

Thus because the equation is not linear, we cannot apply superposition in this scenario.

\item

Apply the same constant substitution as in part a and b,

\begin{align*}
i &= \frac{1}{2} \mu_n C _{ox} \frac{W}{L} (v _{gs} - V _{TH})^{2} \\
i &= C_1(v _{gs} - C_2)^{2} \\
\end{align*}

Take the derivative of $i(v _{gs})$,

\begin{align*}
i(v _{gs}) &= C_1(v _{gs} - C_2)^{2} \\
i ' (v _{gs}) &= 2 C_1(v _{gs} - C_2)
\end{align*}

Slope of tangent line at $V _{gs}$,

\begin{align*}
i ' (v _{gs}) &= 2 C_1(v _{gs} - C_2) \\
i ' (V _{gs}) &= 2 C_1(V _{gs} - C_2) \\
\end{align*}

Equation for tangent line at $(V _{gs}, I)$,

\begin{align*}
i - i_1 &= i'(v _{gs_1})(v _{gs} - v _{gs_1}) \\
i - I &= 2 C_1(V _{gs} - C_2) (v _{gs} - V _{gs})
\end{align*}

Substitute back in $C_1 = \frac{1}{2} \mu_n C _{ox} \frac{W}{L}$ and $C_2 = V _{TH}$,

\begin{align*}
i - I &= 2 C_1(V _{gs} - C_2) (v _{gs} - V _{gs}) \\
i - I &= \mu_n C _{ox} \frac{W}{L} (V _{gs} - V _{TH}) (v _{gs} - V _{gs}) \\
\end{align*}

\end{enumerate}

\pagebreak

\begin{problem}

Consider the circuit in the Figure composed of some resistors of value $R$ and others with
value $2R$.

\begin{enumerate}[\alph*)]

\item

Use superposition to calculate the output voltage as a function of the inputs.

\item

Suppose that the inputs $V_a - V_c$ are digital values, i.e. they can only be equal to $0$ or $V _{DD}$. This
circuit is then a digital-to-analog converter (DAC). Explain how the DAC works in conceptual
terms. How many different voltages can be produced in this DAC?

\item

Calculate the output if the digital input [$V_c$ $V_b$ $V_a$] = [$101$].

\item

How would you create a similar DAC with twice as much resolution? (you do not need to show
that it works; just draw the circuit)

\begin{figure}[H]
\centering
\includegraphics[width=0.4\textwidth]{2 circuit.png}
\end{figure}

\end{enumerate}

\end{problem}

\begin{soln}
\end{soln}

\begin{enumerate}[\alph*)]

\item

Superposition tells us that we need to solve for $v _{out}$ by isolating each independent source.

Begin by isolating $v_A$, AKA setting $v_B = 0$ and $v_C = 0$. Here's a redrawn circuit of what
that will look like,

\begin{figure}[H]
\centering
\includegraphics[width=0.5\textwidth]{2a1.jpg}
\end{figure}

From here we can use node analysis to solve for $v _{out}$,

KCLs:

\begin{align*}
\sum i = \frac{v_A - v_1}{2R} - \frac{v_1 - 0}{2R} - \frac{v_1 - v_2}{R} &= 0 \\
\sum i = \frac{v_1 - v_2}{R} - \frac{v_2 - 0}{2R} - \frac{v_2 - v _{out}}{R} &= 0 \\
\sum i = \frac{v_2 - v _{out}}{R} - \frac{v _{out} - 0}{2R} &= 0 \\
\end{align*}

Rearrange first KCL:

\begin{align*}
\frac{v_A - v_1}{2R} - \frac{v_1 - 0}{2R} - \frac{v_1 - v_2}{R} &= 0 \\
v_A - v_1 - v_1 - 2 v_1 + 2 v_2 &= 0 \\
v_A + 2 v_2 - 4 v_1 &= 0 \\
4 v_1 &= v_A + 2 v_2 \\
v_1 &= \frac{v_A + 2 v_2}{4}
\end{align*}

Rearrange third KCL:

\begin{align*}
\frac{v_2 - v _{out}}{R} - \frac{v _{out} - 0}{2R} &= 0 \\
2 v_2 - 2 v _{out} - v _{out} &= 0 \\
2 v_2 - 3 v _{out} &= 0 \\
v_2 &= \frac{3}{2} v _{out} \\
\end{align*}

Substitute in $v_2$ into the first KCL:

\begin{align*}
v_1 &= \frac{v_A + 2 v_2}{4} \\
v_1 &= \frac{v_A + 2 \left( \frac{3}{2} v _{out} \right)}{4} \\
v_1 &= \frac{v_A + 3 v _{out}}{4}
\end{align*}

Substitute $v_1$ and $v_2$ into second KCL:

\begin{align*}
\frac{v_1 - v_2}{R} - \frac{v_2 - 0}{2R} - \frac{v_2 - v _{out}}{R} &= 0 \\
2 v_1 - 2 v_2 - v_2 - 2 v_2 + 2 v _{out} &= 0 \\
2 v _{out} + 2 v_1 - 5 v_2 &= 0 \\
2 v _{out} + 2 \left( \frac{v_A + 3 v _{out}}{4} \right) - 5 \left( \frac{3}{2} v _{out} \right) &= 0 \\
2 v _{out} + \frac{1}{2} v_A + \frac{3}{2} v _{out} - \frac{15}{2} v _{out} &= 0 \\
\frac{1}{2} v_A &= 4 v _{out} \\
v _{out}' &= \frac{1}{8} v_A
\end{align*}

Isolate $v_B$, ($v_A = v_C = 0$) and redraw,

\begin{figure}[H]
\centering
\includegraphics[width=0.5\textwidth]{2a2.jpg}
\end{figure}

KCLs:

\begin{align*}
\sum i = \frac{v_2 - v_1}{R} - \frac{v_1 - 0}{2R} - \frac{v_1 - 0}{2R} &= 0 \\
\sum i = \frac{v_B - v_2}{2R} - \frac{v_2 - v_1}{R} - \frac{v_2 - v _{out}}{R} &= 0 \\
\sum i = \frac{v_2 - v _{out}}{R} - \frac{v _{out} - 0}{2R} &= 0
\end{align*}

Rearrange first KCL:

\begin{align*}
\frac{v_2 - v_1}{R} - \frac{v_1 - 0}{2R} - \frac{v_1 - 0}{2R} &= 0 \\
2 v_2 - 2 v_1 - v_1 - v_1 &= 0 \\
2 v_2 - 4 v_1 &= 0 \\
v_2 &= 2 v_1
\end{align*}

Rearrange third KCL:

\begin{align*}
\frac{v_2 - v _{out}}{R} - \frac{v _{out} - 0}{2R} &= 0 \\
2 v_2 - 2 v _{out} - v _{out} &= 0 \\
2 v_2 - 3 v _{out} &= 0 \\
v_2 &= \frac{3}{2} v _{out}
\end{align*}

Substitute in $v_2$ into the first KCL:

\begin{align*}
v_2 &= 2 v_1 \\
\frac{3}{2} v _{out} &= 2 v_1 \\
v_1 &= \frac{3}{4} v _{out}
\end{align*}

Substitute $v_1$ and $v_2$ into the second KCL:

\begin{align*}
\frac{v_B - v_2}{2R} - \frac{v_2 - v_1}{R} - \frac{v_2 - v _{out}}{R} &= 0 \\
v_B - v_2 - 2 v_2 + 2 v_1 - 2 v_2 + 2 v _{out} &= 0 \\
v_B + 2 v _{out} + 2 v_1 - 5 v_2 &= 0 \\
v_B + 2 v _{out} + 2 \left( \frac{3}{4} v _{out} \right) - 5 \left( \frac{3}{2} v _{out} \right) &= 0 \\
v_B + 2 v _{out} + \frac{3}{2} v _{out} - \frac{15}{2} v _{out} &= 0 \\
v_B &= 4 v _{out} \\
v _{out}'' &= \frac{1}{4} v_B
\end{align*}

Isolate $v_C$, ($v_A = v_B = 0$) and redraw,

\begin{figure}[H]
\centering
\includegraphics[width=0.5\textwidth]{2a3.jpg}
\end{figure}

KCLs:

\begin{align*}
\sum i = \frac{v_2 - v_1}{R} - \frac{v_1 - 0}{2R} - \frac{v_1 - 0}{2R} &= 0 \\
\sum i = \frac{v _{out} - v_2}{R} - \frac{v_2 - v_1}{R} - \frac{v_2 - 0}{2R} &= 0 \\
\sum i = \frac{v_C - v _{out}}{2R} - \frac{v _{out} - v_2}{R} &= 0
\end{align*}

Rearrange first KCL:

\begin{align*}
\frac{v_2 - v_1}{R} - \frac{v_1 - 0}{2R} - \frac{v_1 - 0}{2R} &= 0 \\
2 v_2 - 2 v_1 - v_1 - v_1 &= 0 \\
2 v_2 - 4 v_1 &= 0 \\
v_1 &= \frac{1}{2} v_2
\end{align*}

Substitute in $v_1$ into the second KCL:

\begin{align*}
\frac{v _{out} - v_2}{R} - \frac{v_2 - v_1}{R} - \frac{v_2 - 0}{2R} &= 0 \\
2 v _{out} - 2 v_2 - 2 v_2 + 2 v_1 - v_2 &= 0 \\
2 v _{out} + 2 v_1 - 5 v_2 &= 0 \\
2 v _{out} + 2 \left( \frac{1}{2} v_2 \right) - 5 v_2 &= 0 \\
2 v _{out} + v_2 - 5 v_2 &= 0 \\
2 v _{out} - 4 v_2 &= 0 \\
v_2 &= \frac{1}{2} v _{out}
\end{align*}

Substitute in $v_2$ into the third KCL:

\begin{align*}
\frac{v_C - v _{out}}{2R} - \frac{v _{out} - v_2}{R} &= 0 \\
v_C - v _{out} - 2 v _{out} + 2 v_2 &= 0 \\
v_C + 2 v_2 - 3 v _{out} &= 0 \\
v_C + 2 \left( \frac{1}{2} v _{out} \right) - 3 v _{out} &= 0 \\
v_C + v _{out} - 3 v _{out} &= 0 \\
v_C - 2 v _{out} &= 0 \\
v _{out}''' &= \frac{1}{2} v_C
\end{align*}

Sum together $v _{out}$ superpositions,

\begin{align*}
v _{out} &= v _{out}' + v _{out}'' + v _{out}''' \\
v _{out} &= \boxed{\frac{1}{8} v_A + \frac{1}{4} v_B + \frac{1}{2} v_C}
\end{align*}

\item

The DAC works by taking digital inputs from [$v_A$, $v_B$, $v_C$]
and outputting an analog conversion of binary instruction since
the coefficients of the inputs are consecutive multiples of $2$ apart.

With this DAC, we can expect $\boxed{2^3 = 8}$ different voltage
levels since we have three non identical inputs which can each be
$0$ or $1$.

\item

\begin{align*}
v_A = v_C &= V _{DD} \\
v_B &= 0
\end{align*}

\begin{align*}
v _{out} &= \frac{1}{8} v_A + \frac{1}{4} v_B + \frac{1}{2} v_C \\
v _{out} &= \frac{1}{8} v _{DD} + \frac{1}{4} (0) + \frac{1}{2} v _{DD} \\
v _{out} &= \boxed{\frac{5}{8} v _{DD}}
\end{align*}

\item

We could double the number of voltages by just adding a bit,

\begin{figure}[H]
\centering
\includegraphics[width=0.65\textwidth]{2d1.jpg}
\end{figure}

\end{enumerate}

\pagebreak

\begin{problem}

Recall the two circuits that we simulated in HW1 and built in Lab 1: the resistive voltage divider, and the
voltage divider plus op-amp, as recreated in the Figure.

\begin{enumerate}[\alph*)]

\item

Find the Thevenin equivalent circuit for the voltage divider. Then, imagine that $V _{out}$ is loaded
with a current source of value $i _{out}$ and plot $V _{out}$ vs $i _{out}$ using the Thevenin equivalent.

\item

Find the Thevenin equivalent circuit for the voltage divider plus op-amp circuit (the equivalent
circuit model for the op-amp is given on the right; you may assume an ideal op-amp, namely $R_i = \infty$ and
$R_0 = 0$ and when you have the final answers, take the limit as $A \rightarrow \infty$). Then,
imagine that $V _{out2}$ is loaded with a current source of value $i _{out}$ and plot $V _{out}$ vs $i _{out}$ using the
Thevenin equivalent.

\item

Show that your results match what you found in HW1 and Lab 1.

\item

What advantage does the second circuit provide over the first circuit? State your answer in
terms of Thevenin Equivalent Circuit parameters.

\end{enumerate}

\begin{figure}[H]
\centering
\includegraphics[width=\textwidth]{3 circuit.png}
\end{figure}

\end{problem}

\begin{soln}
\end{soln}

\begin{enumerate}[\alph*)]

\item

Apply test current source to voltage divider circuit,

\begin{figure}[H]
\centering
\includegraphics[width=0.4\textwidth]{3a1.jpg}
\end{figure}

\begin{remark*}
In class we covered this problem by applying the superposition
technique to solve for $V _{test}$ so this will be mostly review.
\end{remark*}

Set up $i _{test}$ superposition by setting $V_1 = 0$,

\begin{figure}[H]
\centering
\includegraphics[width=0.38\textwidth]{3a2.jpg}
\end{figure}

Calculate $R _{eq}$ in parallel,

\begin{align*}
\frac{1}{R _{eq}} &= \frac{1}{R_1} + \frac{1}{R_2} \\
\frac{1}{R _{eq}} &= \frac{R_2}{R_1 R_2} + \frac{R_1}{R_1 R_2} \\
\frac{1}{R _{eq}} &= \frac{R_1 + R_2}{R_1 R_2} \\
R _{eq} &= \frac{R_1 R_2}{R_1 + R_2}
\end{align*}

Calculate voltage difference ($V _{test}$) through $R _{eq}$,

\begin{align*}
V &= i R \\
V _{test}' &= \left( \frac{R_1 R_2}{R_1 + R_2} \right) i _{test}
\end{align*}

Set up $V_1$ superposition by setting $i _{test} = 0$,

\begin{figure}[H]
\centering
\includegraphics[width=0.23\textwidth]{3a3.jpg}
\end{figure}

Perform clockwise current node analysis on highlighted node,

\begin{align*}
\sum i = \frac{V_1 - V _{test}}{R_1} - \frac{V _{test} - 0}{R_2} &= 0 \\
\frac{V_1}{R_1} - \frac{V _{test}}{R_1} - \frac{V _{test}}{R_2} &= 0 \\
\frac{V_1}{R_1} &= \frac{V _{test}}{R_1} + \frac{V _{test}}{R_2} \\
\frac{V_1}{R_1} &= V _{test} \left( \frac{1}{R_1} + \frac{1}{R_2} \right) \\
\frac{V_1}{R_1} &= V _{test} \left( \frac{R_1 + R_2}{R_1 R_2} \right) \\
V _{test}'' &= \frac{V_1}{R_1} \left( \frac{R_1 R_2}{R_1 + R_2} \right) \\
V _{test}'' &= \left( \frac{R_2}{R_1 + R_2} \right) V_1
\end{align*}

Combine superpositions,

\begin{align*}
V _{test} &= V _{test}' + V _{test}'' \\
V _{test} &= \left( \frac{R_2}{R_1 + R_2} \right) V_1 + \left( \frac{R_1 R_2}{R_1 + R_2} \right) i _{test}
\end{align*}

Thevenin equivalent,

\begin{align*}
V _{test} &= V _{th} + R _{th} i _{test} \\
V _{test} &= \boxed{\frac{R_2 V_1}{R_1 + R_2} + \left( \frac{R_1 R_2}{R_1 + R_2} \right) i _{test}}
\end{align*}

Plot $V _{test}$ vs. $i _{test}$,

\begin{figure}[H]
\centering
\includegraphics[width=0.35\textwidth]{3a4.jpg}
\end{figure}

\item

Redraw the circuit with the voltage divider plus equivalent op-amp circuit,

\begin{figure}[H]
\centering
\includegraphics[width=0.6\textwidth]{3b1.jpg}
\end{figure}

Perform node analysis to solve for $v _{test}$,

KCLs,

\begin{align*}
\sum i = \frac{V_1 - V_a}{R_1} - \frac{V_a}{R_2} &= 0 \\
\sum i = i _{test} - \frac{V _{test} - A V_d}{R_0} &= 0
\end{align*}

Rearrange first equation,

\begin{align*}
\frac{V_1 - V_a}{R_1} - \frac{V_a}{R_2} &= 0 \\
\frac{V_1}{R_1} - \frac{V_a}{R_1} - \frac{V_a}{R_2} &= 0 \\
V_a \left( \frac{R_1 + R_2}{R_1 R_2} \right) &= \frac{V_1}{R_1} \\
V_a &= \left( \frac{R_2}{R_1 + R_2} \right) V_1
\end{align*}

Rearrange second equation,

\begin{align*}
i _{test} - \frac{V _{test} - A V_d}{R_0} &= 0 \\
i _{test} - \frac{V _{test}}{R_0} + \frac{A V_d}{R_0} &= 0 \\
\frac{V _{test}}{R_0} &= i _{test} + \frac{A V_d}{R_0} \\
V _{test} &= R_0 i _{test} + A V_d \\
V _{test} &= (0) i _{test} + A V_d \\
V _{test} &= A V_d \\
\end{align*}

From the the circuit,

\begin{align*}
V_a - V_d &= V _{test} \\
V_d &= V_a - V _{test} \\
V_d &= \left( \frac{R_2}{R_1 + R_2} \right) V_1 - V _{test}
\end{align*}

Plug back into second equation,

\begin{align*}
V _{test} &= A V_d \\
V _{test} &= A \left( \left( \frac{R_2}{R_1 + R_2} \right) V_1 - V _{test} \right) \\
\frac{V _{test}}{A} &= \left( \frac{R_2}{R_1 + R_2} \right) V_1 - V _{test} \\
\lim _{A \to \infty} \frac{V _{test}}{A} &= \lim _{A \to \infty} \left( \frac{R_2}{R_1 + R_2} \right) V_1 - V _{test} \\
0 &= \left( \frac{R_2}{R_1 + R_2} \right) V_1 - V _{test} \\
V _{test} &= \left( \frac{R_2}{R_1 + R_2} \right) V_1
\end{align*}

Thevenin equivalent,

\begin{align*}
V _{test} &= V _{th} + R _{th} i _{test} \\
V _{test} &= \boxed{\frac{R_2 V_1}{R_1 + R_2} + (0) i _{test}}
\end{align*}

Plot $V _{test}$ vs. $i _{test}$,

\begin{figure}[H]
\centering
\includegraphics[width=0.35\textwidth]{3a5.jpg}
\end{figure}

\item

Recall that Thevenin is in the form of

\begin{align*}
V _{test} &= V _{th} + R _{th} i _{test}
\end{align*}

In HW1 and Lab 1, we graphed the i-v relationship of the voltage divider and those results
match what I found here. The voltage divider by itself yeilded a linear i-v relationship
which is consistent with what we found previously. The voltage divider plus op-amp circuit
produces a constant i-v relationship which can be represented by a horizontal/vertical line and is
also consistent with previous results.

\item

In the second circuit with ideal conditions $R _{th} \approx 0$, $V _{test}$ barely changes with any load current.
That means that it can maintain more load without affecting $V _{test}$. This relationship is reflected by the
constant line on the i-v graph of the second circuit.

\end{enumerate}

\pagebreak

\begin{problem}
The Moku:Go has a signal generator that you want to use to excite a resistive circuit. The signal
generator’s Thevenin equivalent circuit is a voltage source with an output impedance of $\SI{200}{\ohm}$.

\begin{enumerate}[\alph*)]

\item

Use the concept of Thevenin/Norton equivalent circuits to explain why the resistive circuit can
be modeled as a single resistor.

\item

You program the signal generator to output $\SI{5}{\volt}$. What voltage will you see if the resistive circuit’s
equivalent resistance is the following? $R _{eq} \ll \SI{200}{\ohm}$, $R _{eq} \gg \SI{200}{\ohm}$, and $R _{eq} = \SI{200}{\ohm}$

\item

What resistive circuit equivalent resistance is required so that the actual voltage it experiences is
within 1\% of the voltage you programmed? Is that resistance a minimum or a maximum?

\item

A competing product, the Analog Discovery 2, has an output impedance of 20 ohms. Is this
better or worse than the Moku:Go? Explain your answer, including what you mean by “better.”

\item

In Lab 1, you programmed the signal generator to produce $\SI{2}{\volt}$ but when we measured the
output, it really produced $\SI{4}{\volt}$. If you asked an experienced engineer what was happening, they
might say “The signal generator’s output resistance in \SI{50}{\ohm} and it’s expecting to drive a
matched load.” Explain how this statement justifies why you saw $\SI{4}{\volt}$ in Lab 1 instead of $\SI{2}{\volt}$.

\end{enumerate}

\end{problem}

\begin{soln}
\end{soln}

\begin{enumerate}[\alph*)]

\item

For a purely resistive circuit without any independent sources, its Thevenin equivalent would just be
equivalent to a resistance $R _{eq}$, and can be modeled by a single resistor.

\item

\begin{align*}
V &= \frac{R _{eq}}{R_s + R _{eq}} V_s \\
V &= 5 \left( \frac{R _{eq}}{200 + R _{eq}} \right) \\
\end{align*}

For the case $R _{eq} \ll \SI{200}{\ohm}$,

\begin{align*}
V &\approx \lim _{R _{eq} \to 0} 5 \left( \frac{R _{eq}}{200 + R _{eq}} \right) \\
V &\approx 5 \left( \frac{0}{200 + 0} \right) \\
V &\approx \boxed{\SI{0}{\volt}}
\end{align*}

For the case $R _{eq} \gg \SI{200}{\ohm}$,

\begin{align*}
V &\approx \lim _{R _{eq} \to \infty} 5 \left( \frac{R _{eq}}{200 + R _{eq}} \right) \\
V &\approx \lim _{R _{eq} \to \infty} 5 \left( \frac{1}{\frac{200}{R _{eq}} + 1} \right) \\
V &\approx 5 \left( \frac{1}{0 + 1} \right) \\
V &\approx \boxed{\SI{5}{\volt}}
\end{align*}

For the case $R _{eq} = \SI{200}{\ohm}$,

\begin{align*}
V &= 5 \left( \frac{200}{200 + 200} \right) \\
V &= 5 \left( \frac{1}{2} \right) \\
V &= \boxed{\SI{2.5}{\volt}}
\end{align*}

\item

\begin{align*}
V &= \frac{R _{eq}}{R_s + R _{eq}} V_s \\
\frac{V}{V_s} &= \frac{R _{eq}}{R_s + R _{eq}} \geq 0.99 \\
\end{align*}

\begin{remark*}
We only care about $1\%$ in the direction from $100\% \to 99\%$ and not from $100\% \to 101\%$
bceause the output voltage cannot physically be higher than the programmed voltage after
passing through resistance.
\end{remark*}

\begin{align*}
\frac{R _{eq}}{R_s + R _{eq}} &\geq 0.99 \\
R _{eq} &\geq 0.99(R_s + R _{eq}) \\
R _{eq} &\geq 0.99 R_s + 0.99 R _{eq} \\
R _{eq} - 0.99 R _{eq} &\geq 0.99 R_s \\
R _{eq}(1 - 0.99) &\geq 0.99 R_s \\
R _{eq} &\geq \frac{0.99}{1 - 0.99} R_s \\
R _{eq} &\geq 99 R_s
\end{align*}

For $R_s = \SI{200}{\ohm}$,

\begin{align*}
R _{eq} &\geq 99 R_s \\
R _{eq} &\geq 99 (200) \\
R _{eq} &\geq 19800 \\
R _{eq} &\geq \boxed{\SI{19.8}{\kilo \ohm}}
\end{align*}

This is a mimimum because $V \to V_s$ as $R _{eq} \to \infty$. This realtionship was modeled in the
previous part (b). $\SI{19.8}{\kilo \ohm}$ just represents the minimum resistance to remain within
1\% of the programed voltage when matched with an impedence of $\SI{200}{\ohm}$.

\item

"Better" usually refers to less difference between the programmed voltage and actual voltage. In the case of the competing product with a lower
$R_s$ of $\SI{20}{\ohm}$. The ratio

\begin{align*}
\frac{R _{eq}}{R_s + R _{eq}}
\end{align*}

is closer to $1$ than when $R_s = \SI{200}{\ohm}$, so the competitor would be better.

Another way to think about this is that the competitor would ``need'' a  lesser resistive
circuit equivalent to be within the same difference of voltage.

For less than 1\% difference,

\begin{align*}
R _{eq} &\geq 99 (20) \\
R _{eq} &\geq 1980 \\
R _{eq} &\geq \SI{1.98}{\kilo \ohm}
\end{align*}

\begin{align*}
\SI{1.98}{\kilo \ohm} < \SI{19.8}{\kilo \ohm}
\end{align*}

\item

\begin{align*}
V &= \frac{R _{eq}}{R_s + R _{eq}} V_s \\
V &= \frac{50}{50 + 50} V_s \\
V &= \frac{1}{2} V_s \\
V_s &= 2 V \\
\end{align*}

When $V$ = $\SI{2}{\volt}$,

\begin{align*}
V_s &= 2 (\SI{2}{\volt}) \\
V_s &= \SI{4}{\volt}
\end{align*}

If the load is matched $R _{eq} = R_s$, then the ration between $V_s$ and $V$ would be 1:2. That's
why we saw $\SI{4}{\volt}$ in Lab 1.

\end{enumerate}

\pagebreak

\begin{problem}
Suppose you have an unknown system (e.g., in a box). It is meant to take two voltages and one current
as inputs $v_1$, $v_2$, $i_3$ and produce a single voltage as output $v_o$. All you know about the box is that the
system inside is linear and the results of the following experiments, with voltages in volts and currents in
amps.

\begin{figure}[H]
\centering
\includegraphics[width=0.4\textwidth]{5 table.png}
\end{figure}

Find the system equation, i.e. the equation that relates the inputs and the output. Be sure to include
the units for each constant.
\end{problem}

\begin{soln}
\end{soln}

Because the system is linear, we can model the equation with,

\begin{align*}
v_o = A v_1 + B v_2 + C i_3
\end{align*}

where $A$ and $B$ have no units and $C$ has units of ohms.

Convert the table into a system of three equations,

\begin{align*}
3A + 2B + C &= 37 \\
2A + 2B + 2C &= 36 \\
A + 5B + 3C &= 44
\end{align*}

\[
\left\{
\begin{aligned}
3A + 2B + C &= 37 \\
-2A - 2B - 2C &= - 36
\end{aligned}
\right.
\quad \Rightarrow \quad
A - C = 1
\quad \Rightarrow \quad
A = C + 1
\]

\begin{align*}
3 (C + 1) + 2B + C &= 37 \\
3C + 3 + 2B + C &= 37 \\
4C + 2B + 3 &= 37
\end{align*}

\begin{align*}
(C + 1) + 5B + 3C &= 44 \\
4C + 5B + 1 &= 44
\end{align*}

\[
\left\{
\begin{aligned}
4C + 5B + 1 &= 44 \\
-4C -2B - 3 &= 37
\end{aligned}
\right.
\quad \Rightarrow \quad
3B - 2 = 7
\quad \Rightarrow \quad
B = 3
\]

\begin{align*}
4C + 2(3) + 3 &= 37 \\
4C + 9 &= 37 \\
4C &= 28 \\
C &= 7
\end{align*}

\begin{align*}
A &= C + 1 \\
A &= 7 + 1 \\
A &= 8
\end{align*}

\begin{align*}
\boxed{v_o = 8 v_1 + 3 v_2 + \SI{7}{\ohm} i_3}
\end{align*}

\pagebreak

\begin{problem}

Find the Thevenin equivalent for the circuit below at the port consisting of terminals a and b.
(Hint: as always, apply $v _{test}$ and solve for $i _{test}$ or apply $i _{test}$ and solve for $v _{test}$. When solving, you may
want/need to apply other techniques, such as node analysis or equivalent circuits.)

\begin{figure}[H]
\centering
\includegraphics[width=0.5\textwidth]{6 circuit.png}
\end{figure}

\end{problem}

\begin{soln}
\end{soln}

Begin by drawing the circuit with a an $i _{test}$ current source,

\begin{figure}[H]
\centering
\includegraphics[width=\textwidth]{61.jpg}
\end{figure}

By performing superposition on both indepdendent sources, we can
solve for $v _{out} = v _{out}' + v _{out}''$

Isolate the voltage source by setting $i _{test} = 0$,

\begin{figure}[H]
\centering
\includegraphics[width=0.95\textwidth]{62.jpg}
\end{figure}

KCLs,

\begin{align*}
\sum i = \frac{19 - v_a}{10} - \frac{v_a - 0}{4} - \frac{v_a - v_b}{20} &= 0 \\
\sum i = 2 I_x + \frac{v_a - v_b}{20} + \frac{19 - v_b}{5} - \frac{v_b - 0}{8} - \frac{v_b - v _{out}}{2} &= 0 \\
\sum i = \frac{v_b - v _{out}}{2} &= 0
\end{align*}

i-v's,

\begin{align*}
I_x = \frac{19 - v_a}{10}
\end{align*}

Solve for $v _{test}'$,

\begin{align*}
\frac{v_b - v _{out}}{2} &= 0 \\
v_b - v _{out} &= 0 \\
v_b &= v _{out}
\end{align*}

\begin{align*}
\frac{v_b - v_a}{20} - \frac{v_a - 0}{4} - \frac{v_a - 0}{10} &= 0 \\
38 - 2 v_a - 5 v_a - v_a + v_b &= 0 \\
38 + v_b - 8 v_a &= 0 \\
38 + v _{test} - 8 v_a &= 0 \\
v_a &= \frac{19}{4} + \frac{1}{8} v _{test}
\end{align*}

\begin{align*}
2 I_x + \frac{v_a - v_b}{20} + \frac{19 - v_b}{5} - \frac{v_b - 0}{8} - \frac{v_b - v _{out}}{2} &= 0 \\
40 I_x + v_a - v_b + 76 - 4 v_b - \frac{5}{2} v_b - 10 v_b + 10 v _{test} &= 0 \\
76 - 4 v_a + v_a - v_b + 76 - 4 v_b - \frac{5}{2} v_b - 10 v_b + 10 v _{test} &= 0 \\
76 - 4 v_a + v_a - v _{test} + 76 - 4 v _{test} - \frac{5}{2} v _{test} - 10 v _{test} + 10 v _{test} &= 0 \\
152 - 3 v_a - \frac{15}{2} v _{test} &= 0
\end{align*}

\begin{align*}
152 - 3 v_a - \frac{15}{2} v _{test} &= 0 \\
152 - 3 \left( \frac{19}{4} + \frac{1}{8} v _{test} \right) - \frac{15}{2} v _{test} &= 0 \\
152 - \frac{57}{4} - \frac{3}{8} v _{test} - \frac{15}{2} v _{test} &= 0 \\
\frac{551}{4} - \frac{63}{8} v _{test} &= 0 \\
v _{test}' &= \frac{1102}{63}
\end{align*}

Isolate the current source by setting the voltage source to $0$,

\begin{figure}[H]
\centering
\includegraphics[width=0.95\textwidth]{63.jpg}
\end{figure}

KCLs,

\begin{align*}
\sum i = \frac{v_b - v_a}{20} - \frac{v_a - 0}{4} - \frac{v_a - 0}{10} &= 0 \\
\sum i = \frac{v _{test} - v_b}{2} + 2 I_x - \frac{v_b - v_a}{20} - \frac{v_b - 0}{8} - \frac{v_b - 0}{5} &= 0 \\
\sum i = i _{test} - \frac{v _{test} - v_b}{2} &= 0
\end{align*}

i-v's,

\begin{align*}
I_x = - \frac{v_a - 0}{10}
\end{align*}

Solve for $v _{test}''$,

\begin{align*}
\frac{v_b - v_a}{20} - \frac{v_a - 0}{4} - \frac{v_a - 0}{10} &= 0 \\
v_b - v_a - 5 v_a - 2 v_a &= 0 \\
v_b - 8 v_a &= 0 \\
v_b &= 8 v_a
\end{align*}

\begin{align*}
\frac{v _{test} - v_b}{2} + 2 I_x - \frac{v_b - v_a}{20} - \frac{v_b - 0}{8} - \frac{v_b - 0}{5} &= 0 \\
10 v _{test} - 10 v_b + 40 I_x - v_b + v_a - \frac{5}{2} v_b - 4 v_b &= 0 \\
10 v _{test} - 10 v_b - 4 v_a - v_b + v_a - \frac{5}{2} v_b - 4 v_b &= 0 \\
10 v _{test} - 80 v_a - 4 v_a - 8 v_a + v_a - 20 v_a - 32 v_a &= 0 \\
10 v _{test} - 143 v_a &= 0 \\
10 v _{test} &= 143 v_a \\
v_a &= \frac{10}{143} v _{test}
\end{align*}

\begin{align*}
i _{test} - \frac{v _{test} - v_b}{2} &= 0 \\
2 i _{test} - v _{test} + v_b &= 0 \\
2 i _{test} - v _{test} + 8 v_a &= 0 \\
2 i _{test} - v _{test} + \frac{80}{143} v _{test} &= 0 \\
2 i _{test} - \frac{63}{143} v _{test} &= 0 \\
v _{test}'' &= \frac{286}{63} i _{test}
\end{align*}

Plug back into $v _{test}$,

\begin{align*}
v _{test} &= v _{test}' + v _{test}'' \\
v _{test} &= \frac{1102}{63} + \frac{286}{63} i _{test}
\end{align*}

Thevenin equivalent,

\begin{align*}
v _{test} &= v _{th} + R _{th} i _{test} \\
v _{test} &= \frac{1102}{63} + \frac{286}{63} i _{test} \\
v _{test} &= \boxed{\SI{17.492}{\volt} + \SI{4.540}{\ohm} i _{test}}
\end{align*}

\end{document}
